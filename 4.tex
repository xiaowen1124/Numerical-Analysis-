\newpage
\section{解线性方程组的迭代法及非线性方程迭代法}
\subsection{课后习题}

\begin{tcolorbox}[breakable,enhanced,arc=0mm,outer arc=0mm,
		boxrule=0pt,toprule=1pt,leftrule=0pt,bottomrule=1pt, rightrule=0pt,left=0.2cm,right=0.2cm,
		titlerule=0.5em,toptitle=0.1cm,bottomtitle=-0.1cm,top=0.2cm,
		colframe=white!10!biru,colback=white!90!biru,coltitle=white,
            coltext=black,title =2024-04, title style={white!10!biru}, before skip=8pt, after skip=8pt,before upper=\hspace{2em},
		fonttitle=\bfseries,fontupper=\normalsize]
  
1. 设有方程组 $ A x=b $, 其中 $ A $ 为对称正定矩阵, 迭代公式
$$
x^{(k+1)}=x^{(k)}+\omega\left(b-A x^{(k)}\right)(k=0,1,2, \cdots) .
$$
试证: 当 $ 0<\omega<\frac{2}{\beta} $ 时, 上述迭代法收敛
 (其中  $0<\alpha \leqslant \lambda(A) \leqslant \beta$  ). 
\tcblower

将迭代格式改写成
$$
\boldsymbol{x}^{(k+1)}=(\boldsymbol{I}-\omega \boldsymbol{A}) \boldsymbol{x}^{(k)}+\omega \boldsymbol{b} \quad(k=0,1,2, \cdots)
$$
即迭代矩阵 $ \boldsymbol{B}=\boldsymbol{I}-\omega \boldsymbol{A} $,设迭代矩阵的特征值为$\mu$, 对应的特征向量为$\boldsymbol{v}$, 因此 $ \boldsymbol{Bv} = \mu \boldsymbol{v} $,于是 $ (\boldsymbol{I}-\omega \boldsymbol{A})\boldsymbol{v} = \mu \boldsymbol{v} $,整理可得 $ \boldsymbol{v} - \omega \boldsymbol{A}\boldsymbol{v} = \mu \boldsymbol{v} $,进一步整理即可得到迭代矩阵 $ \boldsymbol{B}$的特征值 $ \mu = 1 - \omega \lambda(\boldsymbol{A}) $.

由 $ |\mu|<1$, 即$|1-{\omega \lambda}(\boldsymbol{A})|<1 $, 得$0<\omega<\frac{2}{\lambda(\boldsymbol{A})}$,
而$0<\alpha \leq \lambda(A) \leq \beta$,所以$ 0<\omega<\frac{2}{\beta}<\frac{2}{\lambda(\boldsymbol{A})}$.
故当 $0<\omega<\frac{2}{\beta} $ 时,有 $|\mu|<1, \rho(\boldsymbol{B})<1 $, 因此迭代格式收敛.
\end{tcolorbox}


\begin{tcolorbox}[breakable,enhanced,arc=0mm,outer arc=0mm,
		boxrule=0pt,toprule=1pt,leftrule=0pt,bottomrule=1pt, rightrule=0pt,left=0.2cm,right=0.2cm,
		titlerule=0.5em,toptitle=0.1cm,bottomtitle=-0.1cm,top=0.2cm,
		colframe=white!10!biru,colback=white!90!biru,coltitle=white,
            coltext=black,title =2024-04, title style={white!10!biru}, before skip=8pt, after skip=8pt,before upper=\hspace{2em},
		fonttitle=\bfseries,fontupper=\normalsize]
  
2. 设方程组 $ \left\{\begin{array}{l}a_{11} x_{1}+a_{12} x_{2}=b_{1} \\ a_{21} x_{1}+a_{22} x_{2}=b_{2}\end{array}, a_{11} a_{22} \neq 0\right. $.
求证: 

(1). 用 Jacobi 迭代法与 G-S 迭代法解此方程组
收敛的充要条件为 $\left|\dfrac{a_{12} a_{21}}{a_{11} a_{22}}\right|<1 ;$

(2). Jacobi 方法和 Gauss-Seidel 方法同时收敛或同时发散.
\tcblower

 (1) 由题意可知, Jacobi 迭代法的迭代矩阵
$$
B_{J}=D^{-1}(L+U)=\left[\begin{array}{cc}
a_{11} & 0 \\
0 & a_{22}
\end{array}\right]^{-1}\left[\begin{array}{cc}
0 & -a_{12} \\
-a_{21} & 0
\end{array}\right]=\left[\begin{array}{cc}
0 & -\frac{a_{12}}{a_{11}} \\
-\frac{a_{21}}{a_{22}} & 0
\end{array}\right]
$$

由 $ \operatorname{det}\left(\lambda I-B_{J}\right)=\lambda^{2}-\frac{a_{12} a_{21}}{a_{11} a_{21}} $, 计算其特征值 $ \lambda_{1,2}= \pm \sqrt{\left|\frac{a_{12} a_{21}}{a_{11} a_{22}}\right|} $, 因此Jacobi 迭代法收敛满足:
$$
\rho\left(B_{J}\right)=\sqrt{\left|\frac{a_{12} a_{21}}{a_{11} a_{22}}\right|}<1\iff \left|\dfrac{a_{12} a_{21}}{a_{11} a_{22}}\right|<1
$$

同理, Gauss-Seidel 迭代法的迭代矩阵为  
$$
B_{G}=(D-L)^{-1}U=\left[\begin{array}{cc}
a_{11} & 0 \\
a_{21} & a_{22}
\end{array}\right]^{-1}\left[\begin{array}{cc}
0 & -a_{12} \\
0 & 0
\end{array}\right]=\left[\begin{array}{cc}
0 & -\frac{a_{12}}{a_{11}} \\
0 & \frac{a_{12} a_{21}}{a_{11} a_{12}}
\end{array}\right]
$$
其中$\left[\begin{array}{cc}
a_{11} & 0 \\
a_{21} & a_{22}
\end{array}\right]^{-1}=\frac{1}{a_{11}a_{22}}\left[\begin{array}{cc}
a_{22} & 0 \\
-a_{21} & a_{11}
\end{array}\right]$. 由 $ \operatorname{det}\left(\lambda I-B_{G}\right)=\lambda\left(\lambda-\frac{a_{12} a_{21}}{a_{11} a_{12}}\right) $, 计算其特征值 $ \lambda_{1}=0, \lambda_{2}=\frac{a_{12} a_{21}}{a_{11} a_{22}} $, 因此
$$
\rho\left(B_{G}\right)<1\iff \left|\frac{a_{12} a_{21}}{a_{11} a_{22}}\right|<1
$$

(2) 当 $|\frac{a_{12} a_{21}}{a_{11} a_{22}}|<1$ 时两种方法同时收敛; 当 $|\frac{a_{12} a_{21}}{a_{11} a_{22}}| \geqslant 1$ 时两种方法同时发散.
\end{tcolorbox}

\begin{tcolorbox}[breakable,enhanced,arc=0mm,outer arc=0mm,
		boxrule=0pt,toprule=1pt,leftrule=0pt,bottomrule=1pt, rightrule=0pt,left=0.2cm,right=0.2cm,
		titlerule=0.5em,toptitle=0.1cm,bottomtitle=-0.1cm,top=0.2cm,
		colframe=white!10!biru,colback=white!90!biru,coltitle=white,
            coltext=black,title =2024-04, title style={white!10!biru}, before skip=8pt, after skip=8pt,before upper=\hspace{2em},
		fonttitle=\bfseries,fontupper=\normalsize]
  
3. 设
$
A=\left[\begin{array}{ccc}
3 & 7 & 1 \\
0 & 4 & t+1 \\
0 & -t+1 & -1
\end{array}\right], \quad b=\left[\begin{array}{l}
1 \\
1 \\
0
\end{array}\right], \quad A x=b,
$
其中 $ t $ 为实参数.

(1). 求用 Jacobi 法解 $ A x=b $ 时的迭代矩阵;

(2). $ t $ 在什么范围内 Jacobi 迭代法收敛.
\tcblower
(1)
$$
\boldsymbol{B}_{J}=\boldsymbol{D}^{-1}(\boldsymbol{L}+\boldsymbol{U})=\left[\begin{array}{ccc}
\frac{1}{3} &  &  \\
 & \frac{1}{4} &  \\
 &  & -1
\end{array}\right]\left[\begin{array}{ccc}
0 & -7 & -1 \\
0 & 0 & -(t+1) \\
0 & t-1 & 0
\end{array}\right]=\left[\begin{array}{ccc}
0 & -\frac{7}{3} & -\frac{1}{3} \\
0 & 0 & -\frac{1}{4}(t+1) \\
0 & 1-t & 0
\end{array}\right]
$$

(2)
$$\operatorname{det}\left(\lambda \boldsymbol{I}-\boldsymbol{B}_{J}\right)=\left|\begin{array}{ccc}
\lambda & \frac{7}{3} & \frac{1}{3} \\
0 & \lambda & \frac{1}{4}(t+1) \\
0 & t-1 & \lambda
\end{array}\right|=\lambda^{3}-\frac{\lambda}{4}\left(t^{2}-1\right)=0$$
所以 $ \lambda_1=0 $, $ \lambda_{2,3}=\pm\frac{1}{2} \sqrt{t^{2}-1} $, 由 $ \rho\left(\boldsymbol{B}_{J}\right)=\frac{1}{2} |\sqrt{t^{2}-1}|<1 $, 解得 $  -\sqrt{5}<t<\sqrt{5} $.
\end{tcolorbox}


\begin{tcolorbox}[breakable,enhanced,arc=0mm,outer arc=0mm,
		boxrule=0pt,toprule=1pt,leftrule=0pt,bottomrule=1pt, rightrule=0pt,left=0.2cm,right=0.2cm,
		titlerule=0.5em,toptitle=0.1cm,bottomtitle=-0.1cm,top=0.2cm,
		colframe=white!10!biru,colback=white!90!biru,coltitle=white,
            coltext=black,title =2024-04, title style={white!10!biru}, before skip=8pt, after skip=8pt,before upper=\hspace{2em},
		fonttitle=\bfseries,fontupper=\normalsize]
  
4. 设 $ A=\left[\begin{array}{ccc}t & 1 & 1 \\ \frac{1}{t} & t & 0 \\ \frac{1}{t} & 0 & t\end{array}\right], b=\left[\begin{array}{l}0 \\ 1 \\ 2\end{array}\right], \quad A x=b $,
试问用 Gauss-Seidel 迭代法解 $ A x=b $ 时, 实参数 $ t $ 在什么
范围内上述迭代法收敛.

\tcblower

$$
\boldsymbol{B}_{G}=(\boldsymbol{D}-\boldsymbol{L})^{-1} \boldsymbol{U}=\left[\begin{array}{rrr}
0 & -\frac{1}{t} & -\frac{1}{t} \\
0 & \frac{1}{t^{3}} & \frac{1}{t^{3}} \\
0 & \frac{1}{t^{3}} & \frac{1}{t^{3}}
\end{array}\right]
$$

所以
$$
\operatorname{det}\left(\lambda \boldsymbol{I}-\boldsymbol{B}_{G}\right)=\left|\begin{array}{ccc}
\lambda & \frac{1}{t} & \frac{1}{t} \\
0 & \lambda-\frac{1}{t^{3}} & -\frac{1}{t^{3}} \\
0 & -\frac{1}{t^{3}} & \lambda-\frac{1}{t^{3}}
\end{array}\right|=\lambda\left(\lambda-\frac{1}{t^{3}}\right)^{2}-\lambda\left(\frac{1}{t^{6}}\right)=\lambda^2(\lambda-\frac{2}{t^3})=0
$$

解得 $ \lambda_{1,2}=0 $ , $ \lambda_3={\frac{2}{t^{3}}} $, 即 $ \rho\left(\boldsymbol{B}_{G}\right)=|\frac{2}{t^{3}}|<1 $, 解得 $ t>\sqrt[3]{2} $ 或 $ t<-\sqrt[3]{2} $ .
\end{tcolorbox}

\begin{tcolorbox}[breakable,enhanced,arc=0mm,outer arc=0mm,
		boxrule=0pt,toprule=1pt,leftrule=0pt,bottomrule=1pt, rightrule=0pt,left=0.2cm,right=0.2cm,
		titlerule=0.5em,toptitle=0.1cm,bottomtitle=-0.1cm,top=0.2cm,
		colframe=white!10!biru,colback=white!90!biru,coltitle=white,
            coltext=black,title =2024-04, title style={white!10!biru}, before skip=8pt, after skip=8pt,before upper=\hspace{2em},
		fonttitle=\bfseries,fontupper=\normalsize]
  
5. 利用非线性方程迭代方法求根的思想, 证明:
$$
\sqrt{2+\sqrt{2+\sqrt{2+\cdots}}}=2
$$
\tcblower

首先建立迭代公式. 令
$$
x_{k}=\sqrt{2+\sqrt{2+\sqrt{2+\cdots}}}
$$
则有迭代公式
$$
\left\{\begin{array}{l}
x_{k+1}=\sqrt{2+x_{k}}, \quad k=0,1,2, \cdots \\
x_{0}=0
\end{array}\right.
$$
其中迭代函数
$$
\varphi(x)=\sqrt{2+x}, \quad \varphi^{\prime}(x)=\frac{1}{2 \sqrt{2+x}}
$$
显然当 $ x \in[0,2] $ 时 $ \varphi(x) \in[0,2] $, 且成立
$$
\left|\varphi^{\prime}(x)\right| \leqslant \frac{1}{2 \sqrt{2}}<1
$$
因此这一迭代过程收敛于方程$x^{2}-x-2=0$的正根 $ x^{*}=2 $.
\end{tcolorbox}


\begin{tcolorbox}[breakable,enhanced,arc=0mm,outer arc=0mm,
		boxrule=0pt,toprule=1pt,leftrule=0pt,bottomrule=1pt, rightrule=0pt,left=0.2cm,right=0.2cm,
		titlerule=0.5em,toptitle=0.1cm,bottomtitle=-0.1cm,top=0.2cm,
		colframe=white!10!biru,colback=white!90!biru,coltitle=white,
            coltext=black,title =2024-04, title style={white!10!biru}, before skip=8pt, after skip=8pt,before upper=\hspace{2em},
		fonttitle=\bfseries,fontupper=\normalsize]
  
6. 给定线性方程组
$
\left[\begin{array}{ccc}
4 & -1 & 0 \\
-1 & a & 1 \\
0 & 1 & 4
\end{array}\right]\left[\begin{array}{l}
x_{1} \\
x_{2} \\
x_{3}
\end{array}\right]=\left[\begin{array}{l}
2 \\
6 \\
2
\end{array}\right]
$
其中 $ a $ 为非零常数, 分析 $ a $ 在什么范围取值时, Gauss-Seide 迭代格式收敛.

\tcblower

G-S迭代矩阵 $ B_{G} $计算如下:
$$
\begin{array}{l}
B_{G}=(D-L)^{-1}U=\left[\begin{array}{ccc}
4 & 0& 0\\
-1& {a} &0 \\
0&1 &4
\end{array}\right]^{-1}\left[\begin{array}{ccc}
0 &1 & 0\\
 & 0 & -1\\
&  & 0
\end{array}\right] \\
=\left[\begin{array}{ccc}
\frac{1}{4} & 0 & 0 \\
\frac{1}{4a} & \frac{1}{a} &0 \\
-\frac{1}{16a} & -\frac{1}{4a} & \frac{1}{4}
\end{array}\right]\left[\begin{array}{ccc}
0 &1 & 0\\
 0& 0 & -1\\
0& 0 & 0
\end{array}\right] =\left[\begin{array}{ccc}
 0& \frac{1}{4} & 0 \\
0 & \frac{1}{4a} &-\frac{1}{a} \\
0 & -\frac{1}{16a} & \frac{1}{4a}
\end{array}\right] \\
\end{array}
$$
 于是 Gauss-Seide 迭代矩阵 $B_G$ 的特征方程为
$$
\operatorname{det}\left(\lambda I-B_{G}\right)=\left|\begin{array}{ccc}
\lambda& -\frac{1}{4} & 0 \\
0 & \lambda-\frac{1}{4a} &\frac{1}{a} \\
0 & \frac{1}{16a} & \lambda-\frac{1}{4a}
\end{array}\right|=\lambda^2 \cdot\left( \lambda-\frac{1}{2a}\right) 
$$
求得特征值
$$
\lambda_{1,2}=0, \quad \lambda_{3}= \frac{1}{2a},
$$
所以 $ \rho(\boldsymbol{B_G})=\left|\frac{1}{2a}\right| $. 由$\rho(\boldsymbol{B_G})<1$得$ |a|>2 $. 因此当 $ |a|>\frac 12 $时,原方程组 Gauss-Seide 迭代格式收敛.
\end{tcolorbox}


\begin{tcolorbox}[breakable,enhanced,arc=0mm,outer arc=0mm,
		boxrule=0pt,toprule=1pt,leftrule=0pt,bottomrule=1pt, rightrule=0pt,left=0.2cm,right=0.2cm,
		titlerule=0.5em,toptitle=0.1cm,bottomtitle=-0.1cm,top=0.2cm,
		colframe=white!10!biru,colback=white!90!biru,coltitle=white,
            coltext=black,title =2024-04, title style={white!10!biru}, before skip=8pt, after skip=8pt,before upper=\hspace{2em},
		fonttitle=\bfseries,fontupper=\normalsize]
  
7. 试确定 $ a(a \neq 0) $ 的取值范围, 使得求解方程组
$
\left[\begin{array}{ccc}
a & 1 & 3 \\
1 & a & 2 \\
-3 & 2 & a
\end{array}\right]\left[\begin{array}{l}
x \\
y \\
z
\end{array}\right]=\left[\begin{array}{l}
b_{1} \\
b_{2} \\
b_{3}
\end{array}\right] 
$
 的 Jacobi 迭代格式收敛. 
\tcblower

系数矩阵
$$
\begin{aligned}
A=\left[\begin{array}{ccc}
a & 1 & 3 \\
1 & a & 2 \\
-3 & 2 & a
\end{array}\right] & =D-L-U  =\left[\begin{array}{lll}
a & & \\
& a & \\
& & a
\end{array}\right]+\left[\begin{array}{ccc}
0 & \\
1 & 0 \\
-3 & 2 & 0
\end{array}\right]+\left[\begin{array}{lll}
0 & 1 & 3 \\
& 0 & 2 \\
& & 0
\end{array}\right]
\end{aligned}
$$

于是雅可比迭代矩阵 $ B_{J} $计算如下:
$$
\begin{array}{l}
B_{J}=D^{-1}(L+U)=\left[\begin{array}{lll}
\frac{1}{a} & & \\
& \frac{1}{a} & \\
& & \frac{1}{a}
\end{array}\right]\left(\left[\begin{array}{ccc}
0 & & \\
-1 & 0 & \\
3 & -2 & 0
\end{array}\right]+\left[\begin{array}{ccc}
0 & -1 & -3 \\
& 0 & -2 \\
& & 0
\end{array}\right]\right) \\
=\left[\begin{array}{lll}
\frac{1}{a} & & \\
& \frac{1}{a} & \\
& & \frac{1}{a}
\end{array}\right]\left[\begin{array}{ccc}
0 & -1 & -3 \\
-1 & 0 & -2 \\
3 & -2 & 0
\end{array}\right] =\left[\begin{array}{ccc}
0 & -\frac{1}{a} & -\frac{3}{a} \\
-\frac{1}{a} & 0 & -\frac{2}{a} \\
\frac{3}{a} & -\frac{2}{a} & 0
\end{array}\right] \\
\end{array}
$$
 于是Jacobi 迭代矩阵 $ B_J $ 的特征方程为
$$
\operatorname{det}\left(\lambda I-B_{J}\right)=\left|\begin{array}{ccc}
\lambda & \frac{1}{a} & \frac{3}{a} \\
\frac{1}{a} & \lambda & \frac{2}{a} \\
-\frac{3}{a} & \frac{2}{a} & \lambda
\end{array}\right|=\frac{\lambda \cdot\left(a^{2} \lambda^{2}+4\right)}{a^{2}} 
$$
求得特征值
$$
\lambda_{1}=0, \quad \lambda_{2,3}= \pm \frac{2}{|a|} \mathrm{i},
$$
所以 $ \rho(\boldsymbol{B_J})=\left|\frac{2}{a}\right| $. 由$\rho(\boldsymbol{B_J})<1$得$ |a|>2 $. 因此当 $ |a|>2 $时,原方程组 Jacobi 迭代格式收敛.
\end{tcolorbox}


\begin{tcolorbox}[breakable,enhanced,arc=0mm,outer arc=0mm,
		boxrule=0pt,toprule=1pt,leftrule=0pt,bottomrule=1pt, rightrule=0pt,left=0.2cm,right=0.2cm,
		titlerule=0.5em,toptitle=0.1cm,bottomtitle=-0.1cm,top=0.2cm,
		colframe=white!10!biru,colback=white!90!biru,coltitle=white,
            coltext=black,title =2024-04, title style={white!10!biru}, before skip=8pt, after skip=8pt,before upper=\hspace{2em},
		fonttitle=\bfseries,fontupper=\normalsize]
  
8. 设 $ n \geq 2 $ 为正整数, $ c $ 为正常数, 记 $ f(x)=x^{n}-c=0 $ 的根为 $ x^{*} $,

(1) 假设 $ 0<x_{k}<\sqrt[n]{c} $, 试说明迭代格式 $ x_{k+1}=c x_{k}^{1-n},(k= $ $ 0,1,2 \cdots) $ 不能用来计算 $ x^{*} $ 的近似值

(2) 试写出求解 $ x^{*} $ 近似值的牛顿迭代格式, 并计算迭代格式的收敛阶数.


\tcblower
 (1) 记 $ \varphi(x)=c x^{1-n} $, 则 $ \varphi^{\prime}(x)=c(1-n) x^{-n}, \varphi^{\prime}\left(x^{*}\right)=1-n $.
 
(a) 当 $ n \geqslant 3 $ 时, $ \left|\varphi^{\prime}\left(x^{*}\right)\right|=n-1 \geqslant 2 $, 迭代格式发散.

(b) 当 $ n=2 $ 时, 有$x_{k+1}=\frac{c}{x_{k}}, \quad k=0,1,2, \cdots,$ 

设 $x_{0} \neq x^{*}$, 则有 $x_{1}=\frac{c}{x_{0}} \neq x^{*}$, 且$x_{k} x_{k+1}=c, \quad k=0,1,2, \cdots$,
$$
x_{k+1}-\sqrt{c}=\frac{c}{x_{k}}-\sqrt{c}=-\frac{\sqrt{c}}{x_{k}}\left(x_{k}-\sqrt{c}\right) 
=\left(-\frac{\sqrt{c}}{x_{k}}\right)\left(-\frac{\sqrt{c}}{x_{k-1}}\right)\left(x_{k-1}-\sqrt{c}\right) 
=x_{k-1}-\sqrt{c},  k=1,2, \cdots,
$$
即$x_{k+1}=x_{k-1}, \quad k=1,2, \cdots,$于是$x_{2 m} \equiv x_{0}, \quad x_{2 m+1} \equiv x_{1}, \quad m=0,1,2, \cdots,$即迭代格式不收敛.

(2) 考虑方程 $ f(x) \equiv x^{n}-c=0 $,$f(x^*)=0, f^{\prime}(x^*)\neq0$,则 $ x^{*} $ 为其单根. 用 Newton 迭代格式:
$$
x_{k+1}=x_{k}-\frac{f\left(x_{k}\right)}{f^{\prime}\left(x_{k}\right)}=\left(1-\frac{1}{n}\right) x_{k}+\frac{c}{n} x_{k}^{1-n}, k=0,1,2, \cdots
$$
求解. 由于 Newton 迭代格式对单根是 2 阶局部收敛的, 所以迭代格式当 $ x_{0} $ 比较靠近 $ x^{*} $ 时是收敛的, 且收敛阶数为 2 .

\begin{tcolorbox}
    定理 :对于迭代过程 $ x_{k+1}=\varphi\left(x_{k}\right) $ 及正整数 $ p $, 如果 $ \varphi^{(p)}(x) $ 在所求根 $ x^{*} $ 的邻近连续,并且
$$
\varphi^{\prime}\left(x^{*}\right)=\varphi^{\prime \prime}\left(x^{*}\right)=\cdots=\varphi^{(p-1)}\left(x^{*}\right)=0, 
\varphi^{(p)}\left(x^{*}\right) \neq 0,
$$
则该迭代过程在点 $ x^{*} $ 邻近是 $ p $ 阶收敛的.
\end{tcolorbox}
令$\varphi(x)=(1-\frac 1n)x+\frac{c}{n}x^{1-n}$,则 $\varphi^{\prime}(x)=(1-\frac 1n)+\frac{c(1-n)}{n}x^{-n},\varphi^{\prime \prime}(x)=c(n-1)x^{-n-1}$.由于
$x^{*}=\sqrt{n}$,且$\varphi^{\prime}\left(x^{*}\right)=0,\varphi^{\prime \prime}\left(x^{*}\right) \neq 0$.所以迭代格式当 $ x_{0} $ 比较靠近 $ x^{*} $ 时是二阶收敛的.

\end{tcolorbox}


\begin{tcolorbox}[breakable,enhanced,arc=0mm,outer arc=0mm,
		boxrule=0pt,toprule=1pt,leftrule=0pt,bottomrule=1pt, rightrule=0pt,left=0.2cm,right=0.2cm,
		titlerule=0.5em,toptitle=0.1cm,bottomtitle=-0.1cm,top=0.2cm,
		colframe=white!10!biru,colback=white!90!biru,coltitle=white,
            coltext=black,title =2024-04, title style={white!10!biru}, before skip=8pt, after skip=8pt,before upper=\hspace{2em},
		fonttitle=\bfseries,fontupper=\normalsize]
  
9. 设 $ f(x)=0 $ 有根, 且 $ M \geqslant f^{\prime}(x) \geqslant m>0 $ 求证: 用迭代格式
$$
x_{i+1}=x_{i}-\lambda f\left(x_{i}\right), \quad i=0,1, \cdots,
$$
取任意初值 $ x_{0} $, 当 $ \lambda $ 满足 $ 0<\lambda<\frac{2}{M} $ 时, 迭代序列 $ \left\{x_{i}\right\} $ 收敛于 $ f(x)=0 $ 的根.
\tcblower

根据迭代公式构造迭代函数$\varphi(x)=x-\lambda f(x)$,因为 $ \varphi\left(x^{*}\right)=x^{*}-\lambda f\left(x^{*}\right)=x^{*} $, 故 $ f(x)=0 $ 的根 $ x^{*} $ 是迭代函数 $ \varphi(x) $ 的一个不动点. 显然迭代函数 $ \varphi(x) $ 的导数存在且为
$$
\varphi^{\prime}(x)=1-\lambda f^{\prime}(x)
$$

注意到 $ 0<m \leqslant f^{\prime}(x) \leqslant M $ 及 $ 0<\lambda<\frac{2}{M} $, 有
$$
0<\lambda m \leqslant \lambda f^{\prime}(x) \leqslant \lambda M<2
$$

以上不等式所有项同乘以 $-1$ , 然后再都加 $1$ , 由不等式运算规则, 有
$$
-1<1-\lambda M \leqslant 1-\lambda f^{\prime}(x) \leqslant 1-\lambda m<1
$$

注意到
$$
\left|1-\lambda f^{\prime}(x)\right| \leqslant \max \{|1-\lambda m|,|1-\lambda M|\}<1, \quad x \in \mathbf{R}
$$

取 $ L=\max \{|1-\lambda m|,|1-\lambda M|\} $, 则有
$$
\left|\varphi^{\prime}(x)\right| \leqslant L<1, x \in \mathbf{R}
$$

此外, 显然有任取 $ x \in \mathbf{R} $ 可得 $ \varphi(x) \in \mathbf{R} $. 因此迭代公式 $ x_{k+1}=x_{k}-\lambda f\left(x_{k}\right) $ 对任意初值 $ x_{0} $ 均收敛于 $ \varphi(x) $ 的唯一不动点 $ x^{*} $, 即它收敛于 $ f(x)=0 $ 的根 $ x^{*} $.


\begin{tcolorbox}
    $$
\left|x_{k}-x^{*}\right| \leqslant L\left|x_{k-1}-x^{*}\right| \leqslant \cdots \leqslant L^{k}\left|x_{0}-x^{*}\right| \rightarrow 0(k \rightarrow \infty) 
$$
即$ \lim\limits _{k \rightarrow \infty} x_{k}=x^{*} $
\end{tcolorbox}

\end{tcolorbox}


\begin{tcolorbox}[breakable,enhanced,arc=0mm,outer arc=0mm,
		boxrule=0pt,toprule=1pt,leftrule=0pt,bottomrule=1pt, rightrule=0pt,left=0.2cm,right=0.2cm,
		titlerule=0.5em,toptitle=0.1cm,bottomtitle=-0.1cm,top=0.2cm,
		colframe=white!10!biru,colback=white!90!biru,coltitle=white,
            coltext=black,title =2024-04, title style={white!10!biru}, before skip=8pt, after skip=8pt,before upper=\hspace{2em},
		fonttitle=\bfseries,fontupper=\normalsize]
  
10. 证明: 当 $ x_{0}=1.5 $ 时, 迭代法 $ x_{k+1}=\sqrt{\frac{10}{4+x_{k}}} $ 收敛于方程 $ f(x)=x^{3}+4 x^{2}-10=0 $ 在区间 $ [1,2] $ 内唯一实根 $ x^{*} $.

\tcblower

首先,我们建立迭代公式:
$$
\left\{
\begin{array}{l}
x_{k+1}=\sqrt{\frac{10}{4+x_{k}}} \\
x_{0}=1.5
\end{array}
\right.
$$
设迭代函数为$\varphi(x)=\sqrt{\frac{10}{4+x}}$,我们很容易验证$x=\varphi(x)$与$f(x)=x^{3}+4x^{2}-10=0$是等价的方程.

显然$\varphi(x)$在区间$[1,2]$上是单调递减的,当 $ {x}=1 $ 时, $ \varphi(1)=\sqrt{2} $, 当 $ {x}=2 $ 时, $ \varphi(2)=\sqrt{\frac{5}{3}} $ .所以当 $ {x} \in[1,2] $ 时, $ 1<\varphi(2) \leqslant \varphi({x}) \leqslant \varphi(1)<2 $, 即 $ \varphi({x}) \in[1,2] $ .而$ \varphi ^{\prime}(x)=-\dfrac{\sqrt{10}}{2\sqrt{(4+x)^{3}}}$.
易知$ \varphi ^{\prime}(x)$是一个增函数, 则有(注意添了绝对值)
$$
\max _{1 \leqslant x \leqslant 2}\left|\varphi^{\prime}(x)\right|=|\varphi^{\prime}(1)|=\left|- \frac{\sqrt{10}}{2\sqrt{(4+1)^{3}}}\right|=\left|-\sqrt{\frac{1}{50}}\right|<1
$$

 所以 $|\varphi^{\prime}({x})|<1 $.依照收敛性定理, 迭代法 $ {x}_{{k}+1}=\sqrt{\frac{10}{4+{x}_{{k}}}} $ 收敛于方程 $ f(x)=x^{3}+4 x^{2}-10=0 $ 在区间 $ [1,2] $ 内唯一实根 $ x^{*} $.
\end{tcolorbox}


\begin{tcolorbox}[breakable,enhanced,arc=0mm,outer arc=0mm,
		boxrule=0pt,toprule=1pt,leftrule=0pt,bottomrule=1pt, rightrule=0pt,left=0.2cm,right=0.2cm,
		titlerule=0.5em,toptitle=0.1cm,bottomtitle=-0.1cm,top=0.2cm,
		colframe=white!10!biru,colback=white!90!biru,coltitle=white,
            coltext=black,title =2024-04, title style={white!10!biru}, before skip=8pt, after skip=8pt,before upper=\hspace{2em},
		fonttitle=\bfseries,fontupper=\normalsize]
  
11. 已知求解线性方程组 $ \boldsymbol{A x}=\boldsymbol{b} $ 的迭代格式:
$$
x_{i}^{(k+1)}=x_{i}^{(k)}+\frac{\mu}{a_{i i}}\left(b_{i}-\sum_{j=1}^{n} a_{i j} x_{j}^{(k)}\right), \quad i=1,2, \ldots, n
$$
(1) 求此迭代法的迭代矩阵 $ \boldsymbol{M}(\boldsymbol{A}=\boldsymbol{D}-\boldsymbol{L}-\boldsymbol{U}) $; 

(2) 当 $ A $ 是严格行对角占优矩阵, $ 0<\mu \leq 1 $ 时, 给出 $ \|M\|_{\infty} $ 表达式, 并证明此时迭代格式收敛.
\tcblower

(1) 要求迭代矩阵 $ M $ ,我们首先需要明确分解系数矩阵 $ \boldsymbol{A} $ 为 $ \boldsymbol{D}-\boldsymbol{L}-\boldsymbol{U} $ ,其中 $ \boldsymbol{D} $ 是 $ \boldsymbol{A} $ 的对角部分, $ \boldsymbol{L} $ 是严格下三角部分 (所有上三角元素为0), $ \boldsymbol{U} $ 是严格上三角部分(所有下三角元素为 $ 0) $ .

将迭代格式重写为更符合矩阵运算的形式:
$$
\boldsymbol{x}^{(k+1)}=\boldsymbol{x}^{(k)}+\mu \boldsymbol{D}^{-1}\left(\boldsymbol{b}-\boldsymbol{A} \boldsymbol{x}^{(k)}\right)
$$
展开后得到:
$$
\boldsymbol{x}^{(k+1)}=(\boldsymbol{I}-\mu \boldsymbol{D}^{-1}\boldsymbol{A})\boldsymbol{x}^{(k)}+\mu \boldsymbol{D}^{-1}\boldsymbol{b}
$$
因此$\boldsymbol{M}=\boldsymbol{I}-\mu \boldsymbol{D}^{-1}\boldsymbol{A}$.


(2) 将 $ \boldsymbol{A}=\boldsymbol{D}-\boldsymbol{L}-\boldsymbol{U} $ 代入$\boldsymbol{M}$得:
$$
\begin{aligned}
\boldsymbol{M}&=\boldsymbol{I}-\mu \boldsymbol{D}^{-1}(\boldsymbol{D}-\boldsymbol{L}-\boldsymbol{U}) \\
&=\boldsymbol{I}-\mu \boldsymbol{D}^{-1} \boldsymbol{D}+\mu \boldsymbol{D}^{-1}(\boldsymbol{L}+\boldsymbol{U})\\
&=I-\mu I+\mu \boldsymbol{D}^{-1}(\boldsymbol{D}-\boldsymbol{A}) \\
&=(1-\mu) \boldsymbol{I}+\mu( \boldsymbol{I}-\boldsymbol{D}^{-1}\boldsymbol{A}) 
\end{aligned}
$$
其中矩阵
$$
\boldsymbol{I}-\boldsymbol{D}^{-1} \boldsymbol{A}=\left(\begin{array}{cccc}
0 & -\frac{a_{12}}{a_{11}} & \cdots & -\frac{a_{1 n}}{a_{11}} \\
-\frac{a_{21}}{a_{22}} & 0 & \cdots & -\frac{a_{2 n}}{a_{22}} \\
\vdots & \vdots & & \vdots \\
-\frac{a_{n 1}}{a_{n n}} & -\frac{a_{n 2}}{a_{n n}} & \cdots & 0
\end{array}\right)
$$

$$\|\boldsymbol{M}\|_{\infty}=\|(1-\mu) \boldsymbol{I}+\mu( \boldsymbol{I}-\boldsymbol{D}^{-1}\boldsymbol{A})\|_{\infty} =|1-\mu|+|\mu| \max _{1 \leqslant i \leqslant n} \sum_{\substack{j=1 \\ j \neq i}} \frac{\left|a_{i j}\right|}{\left|a_{i i}\right|}=|\mu|\max _{1 \leqslant i \leqslant n} \frac{\sum\limits_{\substack{j=1 \\ j \neq i}}^{n}\left|a_{i j}\right|}{\left|a_{i i}\right|}+|1-\mu|$$

由$\boldsymbol{A}$是严格行对角占优知$|a_{i i}|>\sum\limits_{\substack{j=1 \\ j \neq i}}^{n}\left|a_{i j}\right|$,所以
$$\|\boldsymbol{M}\|_{\infty}<|\mu|\cdot1+|1-\mu|=|\mu|+|1-\mu|$$

当$ 0<\mu \leq 1 $ 时,$\|\boldsymbol{M}\|_{\infty}<|\mu|+|1-\mu|=\mu+1-\mu=1$.
 此时迭代格式收敛得证.
\end{tcolorbox}

\subsection{补充习题}
\begin{tcolorbox}[enhanced,colback=10,colframe=9,breakable,coltitle=green!25!black,title=2024]
  
利用非线性方程迭代方法求根的思想证明
$$
1+\frac{1}{1+\frac{1}{1+\cdots}}=\frac{1+\sqrt{5}}{2} .
$$
\tcblower
 记
$$
x_{k}=1+\frac{1}{1+\frac{1}{1+\cdots}},
$$
则有递推式
$$
x_{k+1}=1+\frac{1}{x_{k}}, \quad k=0,1, \cdots .
$$
令 $ \varphi(x)=1+\frac{1}{x} $,则 $ \varphi^{\prime}(x)=-\frac{1}{x^{2}} $. 设 $ \varphi(x) $ 有不动点 $ x^{*} $, 即 $ x^{*}=1+\frac{1}{x^{*}} $, 解之得$x^{*}=\frac{1+\sqrt{5}}{2}.$
另一方面, 因
$$
\left|\varphi^{\prime}\left(x^{*}\right)\right|=\frac{1}{\left(\frac{1+\sqrt{5}}{2}\right)^{2}}<1,
$$

故由定理知, $ \left\{x_{k}\right\} $ 局部收敛于 $ x^{*} $.
\end{tcolorbox}


\begin{tcolorbox}[enhanced,colback=10,colframe=9,breakable,coltitle=green!25!black,title=2024]
  
证明: (1) 求 $ \sqrt{a}(a>0) $ 的近似值 Newton 迭代公式
$x_{t+1}=\frac{1}{2}\left(x_{i}+\frac{a}{x_{i}}\right),  x_{0}>0$
对一切 $ i=1,2, \cdots, x_{i} \geq \sqrt{a} $, 且序列 $\left\{x_{i}\right\} $ 递减.

(2) 对任意初值 $ x_{0}>0 $, 此 Newton 迭代公式收敛到 $ \sqrt{a} $ .
\tcblower
(1) 证明数列单调递减且有下界 \(\sqrt{a}\):

由于 \( x_i > 0 \) 且 \( a > 0 \),根据几何-算术均值不等式,我们有:
\[ x_i + \frac{a}{x_i} \geq 2\sqrt{x_i \cdot \frac{a}{x_i}} = 2\sqrt{a} \]
从而得到:
\[ x_{i+1} = \frac{1}{2}\left(x_i +\frac{a}{x_i}\right) \geq \sqrt{a} \]

这表明每一项 \( x_{i+1} \) 都不小于 \( \sqrt{a} \),也即数列 \(\left\{x_i\right\}\) 有下界 \(\sqrt{a}\).

进一步,我们分析数列的单调性.考虑相邻两项之间的差:
\[ x_{i+1} - x_i = \frac{1}{2}\left(\frac{a}{x_i} - x_i\right) = \frac{1}{2}\left(\frac{a - x_i^2}{x_i}\right) \]
注意到如果 \( x_i > \sqrt{a} \),则 \( x_i^2 > a \) 从而 \( a - x_i^2 < 0 \).这表明 \( x_{i+1} - x_i \leq 0 \),即数列 \(\left\{x_i\right\}\) 是单调递减的.

(2)证明数列 \(\left\{x_i\right\}\) 收敛到 \(\sqrt{a}\):

首先定义函数:
\[ \varphi(x) = \frac{1}{2}\left(x + \frac{a}{x}\right) \]
我们需要考察函数 \(\varphi(x)\) 在 \( x = \sqrt{a} \) 附近的性质.计算 \(\varphi(x)\) 的导数:
\[ \varphi'(x) = \frac{1}{2}\left(1 - \frac{a}{x^2}\right) \]

注意到当 \( x > \sqrt{a} \) 时,\( x^2 > a \),从而 \( \frac{a}{x^2} < 1 \) 且 \(\varphi'(x) > 0\).另外,由于 \( \left| \varphi'(x) \right| = \frac{1}{2}\left|1 - \frac{a}{x^2}\right| < \frac{1}{2} \) ,我们有 \(\varphi\) 在 \( \sqrt{a} \) 处是一个压缩映射.因此,根据压缩映射原理和函数的连续性,数列 \(\left\{x_i\right\}\) 必定收敛到 \(\varphi\) 的不动点.计算不动点,解方程 \(\varphi(x) = x\):
\[ \frac{1}{2}\left(x + \frac{a}{x}\right) = x \Rightarrow x = \sqrt{a} \text{ (考虑到 \(x > 0\) 和 \(a > 0\))} \]

从而,数列 \(\left\{x_i\right\}\) 收敛到 \(\sqrt{a}\).

\end{tcolorbox}



\begin{tcolorbox}[enhanced,colback=10,colframe=9,breakable,coltitle=green!25!black,title=2024]
  
证明: 当 $ x_{0}=1.5 $ 时, 迭代法 $ {x}_{{k}+1}=\sqrt{\frac{8}{3+{x}_{{k}}}} $ 收敛于方程 $ f(x)=x^{3}+3 x^{2}-8=0 $ 在区间 $ [1,2] $ 内唯一实根 $ x^{*} $
\tcblower
(1)函数等价性:原方程 \( f(x) = x^3 + 3x^2 - 8 = 0 \) 可以变形为:
 $x^2(x + 3) = 8 \Rightarrow x = \sqrt{\frac{8}{x + 3}}$,
这说明 \( x = \varphi(x) \) 与 \( f(x) = 0 \) 是等价的.

 (2)$\varphi(x)$是单调递减的,于是有 $1<\sqrt{\frac{8}{3+2}}=\varphi(2) \leqslant \varphi(x) \leqslant \varphi(1)=\sqrt{\frac{8}{3+1}}<2 $, 因此 $\varphi(x) \in[1,2]$.


 (3)计算 \( \varphi(x) \) 的导数,以验证迭代函数的局部收敛性:
\[ \varphi'(x) = -\frac{1}{2} \cdot \frac{8}{(3+x)^2} \cdot \frac{1}{\sqrt{\frac{8}{3+x}}}= -\frac{4}{(3+x)^{\frac{3}{2}}} \cdot \frac{\sqrt{3+x}}{\sqrt{8}} = -\frac{4 \sqrt{3+x}}{(3+x)^{\frac{3}{2}} \sqrt{8}} = -\sqrt{2} \cdot (3+x)^{-\frac{3}{2}} \]

对 \( x \) 在 \([1, 2]\) 上的范围进行最大值估计:
\[ \max_{1 \leqslant x \leqslant 2} |\varphi'(x)| = \max_{1 \leqslant x \leqslant 2} \left|\sqrt{2}(3+x)^{-\frac{3}{2}}\right| = \sqrt{2} \cdot 4^{-\frac{3}{2}} = \frac{\sqrt{2}}{8}<1 \]
这表明 \( \varphi(x) \) 是区间 \([1, 2]\) 上的压缩映射.

由于 \( \varphi(x) \) 在 \([1, 2]\) 上是单调递减的,映射到自身,并且其导数的绝对值小于 1,根据压缩映射定理和迭代函数的性质,迭代序列 \( \left\{x_k\right\} \) 收敛于方程 \( f(x) = x^3 + 3x^2 - 8 = 0 \) 在 \([1, 2]\) 区间内的唯一实根 \( x^* \).
\end{tcolorbox}

\begin{tcolorbox}[enhanced,colback=10,colframe=9,breakable,coltitle=green!25!black,title=2024]
  
 若 $ f(x) $ 在零点 $ \xi $ 的某个领域中有二阶连续导数, 且 $ f^{\prime}(\xi) \neq 0 $ .试证: 对由 Newton 迭代法产生的 $ x_{k}(k=0,1,2, \cdots) $, 成立
$$
\lim _{k \rightarrow \infty} \frac{x_{k}-x_{k-1}}{\left(x_{k-1}-x_{k-2}\right)^{2}}=-\frac{f^{\prime \prime}(\xi)}{2 f^{\prime}(\xi)}
$$
\tcblower


为了证明所给定理,我们首先需要了解牛顿迭代法(Newton's Method)的基本工作原理以及如何利用泰勒展开(Taylor Expansion)来逼近函数值.

牛顿迭代法用于寻找函数 $f(x)$ 的根,迭代公式为:
$$
x_{k+1} = x_k - \frac{f(x_k)}{f'(x_k)}
$$
其中,$x_k$ 是在第 $k$ 步的近似根.

1. 泰勒展开:
   由于 $f(x)$ 在 $\xi$ 点有二阶连续导数,我们可以对 $f(x)$ 在 $x_{k-2}$ 点做泰勒展开到二阶:
   $$
   f(x_{k-1}) = f(x_{k-2}) + f'(x_{k-2})(x_{k-1} - x_{k-2}) + \frac{f''(\xi_{k-1})}{2}(x_{k-1} - x_{k-2})^2
   $$
   其中 $\xi_{k-1}$ 是 $x_{k-2}$ 与 $x_{k-1}$ 之间的某个点.

2. 牛顿迭代公式带入:
   根据牛顿法的迭代公式,我们有:
   $$
   f(x_{k-2}) + f'(x_{k-2})(x_{k-1} - x_{k-2}) = 0
   $$
   因此:
   $$
   f(x_{k-1}) = \frac{f''(\xi_{k-1})}{2}(x_{k-1} - x_{k-2})^2
   $$
   同理,对 $x_{k-1}$ 点展开 $f(x)$ 并利用牛顿迭代法,可以得到:
   $$
   f(x_{k-1}) + f'(x_{k-1})(x_k - x_{k-1}) = 0
   $$
   解得:
   $$
   f'(x_{k-1})(x_k - x_{k-1}) = -f(x_{k-1})
   $$
   代入前面的 $f(x_{k-1})$ 的表达式:
   $$
   f'(x_{k-1})(x_k - x_{k-1}) = -\frac{f''(\xi_{k-1})}{2}(x_{k-1} - x_{k-2})^2
   $$

3. 求极限:
   因此,有:
   $$
   \frac{x_k - x_{k-1}}{(x_{k-1} - x_{k-2})^2} = -\frac{f''(\xi_{k-1})}{2f'(x_{k-1})}
   $$
   当 $k \to \infty$,$x_k \to \xi$,$\xi_{k-1} \to \xi$,因此:
   $$
   \lim_{k \to \infty} \frac{x_k - x_{k-1}}{(x_{k-1} - x_{k-2})^2} = -\frac{f''(\xi)}{2f'(\xi)}
   $$



\end{tcolorbox}


\begin{tcolorbox}[enhanced,colback=10,colframe=9,breakable,coltitle=green!25!black,title=2024]
  
设序列 $ \left\{x_{i}\right\} $ 收敛于 $ x $ 且 $ x_{i} \neq x $,
$$
x_{i+1}-x=\left(q+\xi_{i}\right)\left(x_{i}-x\right), \quad i=0,1, \cdots,
$$

其中 $ |q|<1, \quad \xi_{i} \rightarrow 0 $, 证明
$ y_{i}:=x_{i}-\dfrac{\left(x_{i+1}-x_{i}\right)^{2}}{x_{i+2}-2 x_{i+1}+x_{i}} $ 是合理定义的(当 $ i $ 足够大时), 且 $\displaystyle \lim _{i \rightarrow \infty} \frac{y_{i}-x}{x_{i}-x}=0 $, 即 $ \left\{y_{i}\right\} $ 收敛于 $ x $ 的速度要快于 $ \left\{x_{i}\right\} $ .

\tcblower

为了证明这个问题,我们首先需要理解给定的序列以及相关表达式的行为,并确保表达式在分母不为零的情况下是有意义的.我们从序列的定义开始:
$$
x_{i+1}-x=\left(q+\xi_{i}\right)\left(x_{i}-x\right),
$$
这里的 $|q|<1$ 表示 $q$ 是一个小于 $1$ 的常数,而 $\xi_i \rightarrow 0$ 意味着随着 $i$ 增大,$\xi_i$ 会趋近于 $0$.

 根据迭代式,$x_{i+1}-x_{i} = x_{i+1} - x - (x_{i} - x) = (q + \xi_i)(x_i - x) - (x_i - x) = (q - 1 + \xi_i)(x_i - x)$.

 进一步展开 $x_{i+2} - 2x_{i+1} + x_i$:
$$
\begin{aligned}
   x_{i+2} - 2x_{i+1} + x_i &= [x + (q + \xi_{i+1})(x_{i+1} - x)] - 2[x + (q + \xi_i)(x_i - x)] + x_i \\
   &= (q + \xi_{i+1})(x_{i+1} - x) - 2(q + \xi_i)(x_i - x) + (x_i - x) \\
   &= (q + \xi_{i+1})(q + \xi_i)(x_i - x) - 2(q + \xi_i)(x_i - x) + (x_i - x).
\end{aligned}
$$
因为 $\xi_i, \xi_{i+1} \rightarrow 0$, 所以分母 $x_{i+2} - 2x_{i+1} + x_i$ 随着 $i$ 增大而趋近于 $(q^2 - 2q + 1)(x_i - x) = (q-1)^2(x_i - x)$.当 $q \neq 1$ 时,分母不会为零,因此表达式是合理定义的.

由于:
$$
\lim_{i\to \infty}\frac{y_i - x}{x_i - x} = 1 - \lim_{i\to \infty}\frac{(x_{i+1} - x_i)^2}{(x_{i+2} - 2x_{i+1} + x_i)(x_i - x)}= 1 - \frac{[(q-1)(x_i - x)]^2}{(q-1)^2(x_i - x)^2} = 1 - 1 = 0,
$$
可以看到 $y_i$ 收敛到 $x$ 的速度比 $x_i$ 收敛到 $x$ 的速度快,这是因为 $\frac{y_i - x}{x_i - x} \to 0$.

因此,$y_i$ 收敛到 $x$ 的速度确实比原始序列 $\{x_i\}$ 快,从而证明了所给定的命题.
\end{tcolorbox}


\begin{tcolorbox}[enhanced,colback=10,colframe=9,breakable,coltitle=green!25!black,title=2024]


  
设参数 $ a>0 $, 证明迭代公式 $ x_{k+1}=\dfrac{x_{k}\left(x_{k}^{2}+3 a\right)}{3 x_{k}^{2}+a} $ 产生的序列 $ \left\{x_{k}\right\} $ 三阶收敛到 $ \sqrt{a} $.并求极限 $\displaystyle \lim _{k \rightarrow \infty} \frac{x_{k+1}-\sqrt{a}}{\left(x_{k}-\sqrt{a}\right)^{3}} $ .
\tcblower
 由题意可知, 当 $ a>0 $ 时, 迭代函数为
$$
\varphi(x)=\frac{x\left(x^{2}+3 a\right)}{3 x^{2}+a}
$$

满足 $ \varphi(\sqrt{a})=\sqrt{a} $. 所以 $ \sqrt{a} $ 是 $ \varphi $ 的不动点, 为了求导数方便些, 写成
$$
\left(3 x^{2}+a\right) \varphi(x)=x^{3}+3 a x
$$

两边求导数得
$$
\left(3 x^{2}+a\right) \varphi^{\prime}(x)+6 x \varphi(x)=3 x^{2}+3 a
$$

从而得到 $ \varphi^{\prime}(\sqrt{a})=0 $, 这样迭代法就在不动点 $ x^{*}=\sqrt{a} $ 附近局部收敛, 两边再求导数并整理得到
$$
\left(3 x^{2}+a\right) \varphi^{\prime \prime}(x)+12 x \varphi^{\prime}(x)+6 \varphi(x)=6 x
$$

因此以 $ x=\sqrt{a} $ 代入验证 $ \varphi^{\prime \prime}(\sqrt{a})=0 $, 而再求导整理得
$$
\left(3 x^{2}+a\right) \varphi^{\prime \prime \prime}(x)+18 x \varphi^{\prime \prime}(x)+18 \varphi^{\prime}(x)+6 \varphi(x)=6
$$

得到 $ \varphi^{\prime \prime \prime}(\sqrt{a})=\frac{3}{2a} \neq 0 $, 因此证得原迭代格式三阶收敛到 $ \sqrt{a} $, 另一方面, 由收敛阶的定义
$$
x_{k+1}-\sqrt{a}=\frac{x_{k}^{2}+3 a x_{k}-3 \sqrt{a} x_{k}^{2}-a \sqrt{a}}{3 x_{k}^{2}+a}=\frac{\left(x_{k}-\sqrt{a}\right)^{3}}{3 x_{k}^{2}+a}
$$

因此
$$
\lim _{k \rightarrow \infty} \frac{x_{k+1}-\sqrt{a}}{\left(x_{k}-\sqrt{a}\right)^{3}}=\lim _{k \rightarrow \infty} \frac{1}{3 x_{k}^{2}+a}=\frac{1}{4 a}
$$

这也说明原迭代格式三阶收敛到 $ \sqrt{a} $.
\end{tcolorbox}


\begin{tcolorbox}[enhanced,colback=10,colframe=9,breakable,coltitle=green!25!black,title=2024]

设 $ \rho(A) $ 为 $ {n} $ 阶方阵 $ A $ 的谱半径, 证明
$ \rho(A)<1 $ 的充分必要条件是 $ \lim\limits _{k \rightarrow \infty} A^{k}=\boldsymbol{O} $
\tcblower


$\Rightarrow$ 假设: $\rho(A) < 1$.

首先,回顾谱半径的定义:
$$
\rho(A) = \max_{\lambda \in \sigma(A)} |\lambda|,
$$
其中 $\sigma(A)$ 表示矩阵 $A$ 的所有特征值的集合.

根据谱半径和矩阵范数之间的关系,存在一个与矩阵 $A$ 相关的范数 $\|\cdot\|$,使得 $\|A\| \geq \rho(A)$,并且我们可以选择适当的范数使得 $\|A\| < 1$.

因此,根据矩阵的范数性质,对于任意正整数 $v$,我们有:
$$
\|A^v\| \leq \|A\|^v.
$$
由于 $\|A\| < 1$,$\|A\|^v$ 随 $v$ 增加而指数级减小,趋向于 $0$.因此,
$$
\lim_{v \rightarrow \infty} \|A^v\| = 0.
$$
这意味着 $A^v$ 在范数意义下趋向于零矩阵,即对于任意的向量 $x$,$A^v x \to 0$,所以我们得到:
$$
\lim_{v \rightarrow \infty} A^v = O.
$$


$\Leftarrow$ 使用反证法,假设 $\rho(A) \geq 1$.那么存在至少一个特征值 $\lambda$ 使得 $|\lambda| \geq 1$.设 $x$ 是对应 $\lambda$ 的一个非零特征向量,那么对于任何正整数 $v$,有:
$$
A^v x = \lambda^v x.
$$
如果 $|\lambda| \geq 1$,则随着 $v \to \infty$,$\lambda^v$ 不会趋于0,因此 $A^v x$ 也不会趋于0,这与 $\lim\limits_{v \rightarrow \infty} A^v = 0$ 矛盾.因此,我们的假设不成立,即所有的 $\lambda$ 都必须满足 $|\lambda| < 1$,从而:
$$
\rho(A) < 1.
$$

从以上两部分的分析可见,$\rho(A) < 1$ 当且仅当 $\lim\limits_{v \rightarrow \infty} A^v = 0$.
\end{tcolorbox}

\begin{tcolorbox}[enhanced,colback=10,colframe=9,breakable,coltitle=green!25!black,title=2024]

假设$Ax=b$是一个线性方程组,如果$A$是严格列对角占优的,证明雅可比迭代格式是收敛的.


严格列对角占优:矩阵 $A$ 的每一列的对角元素的绝对值大于该列中其他所有元素的绝对值之和.数学上表示为对于所有的 $j$,有
   $$
   |a_{jj}| > \sum_{i \neq j} |a_{ij}|,
   $$
   其中 $a_{ij}$ 表示矩阵 $A$ 的第 $i$ 行第 $j$ 列的元素.
\tcblower

在雅可比迭代中,迭代矩阵 $B_J$ 定义为$B_J = D^{-1}(L+U).$
首先计算 $B_J$ 的 $1$-范数(列范数),这是列元素绝对值的最大和:
   $$
   \|B_J\|_1 = \max_j \sum_i |b_{ij}|,
   $$
其中 $b_{ij}$ 是矩阵 $B_J$ 的元素.

由于 $A$ 严格列对角占优,我们有
   $$
   |a_{jj}| > \sum_{i \neq j} |a_{ij}|,
   $$
   这意味着在 $D^{-1}(L+U)$ 中每一列的对角线元素(经过缩放)大于其他元素的和.因此,

$$
\left\|\boldsymbol{B_J}\right\|_{1}=\max _{1 \leqslant j \leqslant n} \sum_{\substack{i=1 \\ i \neq j}} \frac{\left|a_{i j}\right|}{\left|a_{jj}\right|}=\max _{1 \leqslant j \leqslant n} \frac{\sum\limits_{\substack{i=1 \\ i \neq j}}^{n}\left|a_{i j}\right|}{\left|a_{jj}\right|}<1
$$
   
因此雅可比迭代方法在这种情况下是收敛的.这意味着对于任意初始向量 $x^{(0)}$,迭代序列 $x^{(k)}$ 将收敛于 $Ax = b$ 的解.
\end{tcolorbox}


\begin{tcolorbox}[enhanced,colback=10,colframe=9,breakable,coltitle=green!25!black,title=2024]

假设$Ax=b$是一个线性方程组,如果$A$是严格行对角占优的,证明雅可比迭代格式是收敛的.

\tcblower
迭代矩阵
$$
\boldsymbol{B_J}=\left(\begin{array}{cccc}
0 & -\frac{a_{12}}{a_{11}} & \cdots & -\frac{a_{1 n}}{a_{11}} \\
-\frac{a_{21}}{a_{22}} & 0 & \cdots & -\frac{a_{2 n}}{a_{22}} \\
\vdots & \vdots & & \vdots \\
-\frac{a_{n 1}}{a_{n n}} & -\frac{a_{n 2}}{a_{n n}} & \cdots & 0
\end{array}\right)
$$

由$\boldsymbol{A}$是严格行对角占优知
$$
\left\|\boldsymbol{B_J}\right\|_{\infty}=\max _{1 \leqslant i \leqslant n} \sum_{\substack{j=1 \\ j \neq i}} \frac{\left|a_{i j}\right|}{\left|a_{i i}\right|}=\max _{1 \leqslant i \leqslant n} \frac{\sum\limits_{\substack{j=1 \\ j \neq i}}^{n}\left|a_{i j}\right|}{\left|a_{i i}\right|}<1
$$
得证
\end{tcolorbox}

\begin{tcolorbox}[enhanced,colback=10,colframe=9,breakable,coltitle=green!25!black,title=2024]

根据给定的线性方程组写出高斯-赛德尔(Gauss-Seidel, G-S)迭代格式,并讨论其收敛性.
$$
\begin{cases}
4x_1 - 2x_3 = 4 \\
x_1 + 4x_2 - 2x_3 = 1 \\
3x_1 - 5x_2 + x_3 = 2
\end{cases}
$$

\tcblower

高斯-赛德尔方法的迭代公式基于将矩阵 $A$ 分解为 $D$ (对角部分),$L$ (严格下三角部分)和 $U$ (严格上三角部分):$A = D - L - U$

对于给定的方程组,$A$, $D$, $L$, 和 $U$ 定义如下:

$$
A = \begin{bmatrix}
4 & 0 & -2 \\
1 & 4 & -2 \\
3 & -5 & 1
\end{bmatrix}, \quad
D = \begin{bmatrix}
4 & 0 & 0 \\
0 & 4 & 0 \\
0 & 0 & 1
\end{bmatrix}, \quad
L = \begin{bmatrix}
0 & 0 & 0 \\
-1 & 0 & 0 \\
-3 & 5 & 0
\end{bmatrix}, \quad
U = \begin{bmatrix}
0 & 0 & 2 \\
0 & 0 & 2 \\
0 & 0 & 0
\end{bmatrix}
$$
高斯-赛德尔迭代方法可以表达为:

$$
x^{(k+1)} =   (D - L)^{-1} U x^{(k)}+(D - L)^{-1} b
$$

这里 $(D-L)^{-1} U$ 是迭代矩阵 $B_{G}$,我们需要计算它.首先,我们对矩阵 $(D-L)$ 求逆,然后用此逆矩阵与 $U$ 相乘.

 矩阵 $(D-L)$ 和 $U$

已知:
$$
D = \begin{bmatrix}
4 & 0 & 0 \\
0 & 4 & 0 \\
0 & 0 & 1
\end{bmatrix}, \quad
L = \begin{bmatrix}
0 & 0 & 0 \\
-1 & 0 & 0 \\
-3 & 5 & 0
\end{bmatrix}, \quad
U = \begin{bmatrix}
0 & 0 & 2 \\
0 & 0 & 2 \\
0 & 0 & 0
\end{bmatrix}
$$

因此:
$$
D-L = \begin{bmatrix}
4 & 0 & 0 \\
1 & 4 & 0 \\
3 & -5 & 1
\end{bmatrix}
$$

 求逆 $(D-L)^{-1}$

要计算 $(D-L)^{-1}$,我们使用初等行变换:

$$
\begin{aligned}
\left[\begin{array}{ccc|ccc}
4 & 0 & 0 & 1 & 0 & 0 \\
1 & 4 & 0 & 0 & 1 & 0 \\
3 & -5 & 1 & 0 & 0 & 1
\end{array}\right] & \rightarrow \text{ 第一行除以4} \\
\left[\begin{array}{ccc|ccc}
1 & 0 & 0 & 0.25 & 0 & 0 \\
1 & 4 & 0 & 0 & 1 & 0 \\
3 & -5 & 1 & 0 & 0 & 1
\end{array}\right] & \rightarrow \text{ 使用第一行消去第二、三行的第一列} \\
\left[\begin{array}{ccc|ccc}
1 & 0 & 0 & 0.25 & 0 & 0 \\
0 & 4 & 0 & -0.25 & 1 & 0 \\
0 & -5 & 1 & -0.75 & 0 & 1
\end{array}\right] & \rightarrow \text{ 第二行除以4} \\
\left[\begin{array}{ccc|ccc}
1 & 0 & 0 & 0.25 & 0 & 0 \\
0 & 1 & 0 & -0.0625 & 0.25 & 0 \\
0 & -5 & 1 & -0.75 & 0 & 1
\end{array}\right] & \rightarrow \text{ 使用第二行消去第三行的第二列} \\
\left[\begin{array}{ccc|ccc}
1 & 0 & 0 & 0.25 & 0 & 0 \\
0 & 1 & 0 & -0.0625 & 0.25 & 0 \\
0 & 0 & 1 & 0.0625 & 1.25 & 1
\end{array}\right] & 
\end{aligned}
$$

得到 $(D-L)^{-1}$:
$$
\begin{bmatrix}
0.25 & 0 & 0 \\
-0.0625 & 0.25 & 0 \\
0.0625 & 1.25 & 1
\end{bmatrix}
$$

 计算 $(D-L)^{-1} U$

$$
\begin{aligned}
(D-L)^{-1} U &= \begin{bmatrix}
0.25 & 0 & 0 \\
-0.0625 & 0.25 & 0 \\
0.0625 & 1.25 & 1
\end{bmatrix} \begin{bmatrix}
0 & 0 & 2 \\
0 & 0 & 2 \\
0 & 0 & 0
\end{bmatrix} \\
&= \begin{bmatrix}
0 & 0 & 0.5 \\
0 & 0 & -0.125 \\
0 & 0 & 0.125
\end{bmatrix}
\end{aligned}
$$



 讨论收敛性:

对于迭代矩阵 $B_{G} = \begin{bmatrix} 0 & 0 & 0.5 \\ 0 & 0 & -0.125 \\ 0 & 0 & 0.125 \end{bmatrix}$,其无穷范数 $\|B_{G}\|_{\infty}$ 是矩阵每一行元素绝对值之和的最大值:

$$
\|B_{G}\|_{\infty} = \max(0.5, -0.125, 0.125) = 0.5
$$

因为 $\|B_{G}\|_{\infty} < 1$,所以高斯-赛德尔迭代方法是收敛的.
\end{tcolorbox}


\begin{tcolorbox}[enhanced,colback=10,colframe=9,breakable,coltitle=green!25!black,title=2024]
  
设 $ I \in R^{n \times n} $ 为单位矩阵, $A \in R^{n \times n} $,且 $ \|A\|<1 $, 证明: $ I+A $ 是非奇异的.
\tcblower

要证明矩阵 $I + A$ 是非奇异的,等价于证明 $I + A$ 没有零特征值,即证明方程 $(I + A)x = 0$ 仅有平凡解 $x = 0$.

方法(1):首先,观察到矩阵 $A$ 满足 $\|A\| < 1$.该条件意味着对于所有的向量 $x$,有 $\|Ax\| \leq \|A\|\cdot \|x\| < \|x\|$.

考虑方程 $(I + A)x = 0$.这意味着 $x + Ax = 0$,因此 $Ax = -x$.因此,我们有
$$ \|x\| = \|Ax\| \leq \|A\| \|x\| < \|x\|, $$
上式表明唯一使其成立的是 $\|x\| = 0$,即 $x = 0$.这说明方程 $(I + A)x = 0$ 仅有平凡解.

方法(2):使用特征值的性质来证明.设 $\lambda$ 是 $A$ 的一个特征值,对应的特征向量为 $v$,则 $Av = \lambda v$.对于 $I + A$,我们有:
$$(I + A)v = v + Av = v + \lambda v = (1 + \lambda) v.$$
因此,$1 + \lambda$ 是 $I + A$ 的特征值.由于 $\|A\| < 1$,所有特征值 $\lambda$ 的绝对值都小于 $1$,所以 $1 + \lambda$ 不可能为零(因为 $|\lambda| < 1$ 意味着 $1 + \lambda \neq 0$).由于 $I + A$ 的所有特征值都非零,$I + A$ 是非奇异的.得证.

\begin{tcolorbox}
\textcolor{red}{也可以采用反证法:}

假设 $\boldsymbol{I}+\boldsymbol{A}$ 是奇异矩阵, 则存在非零向量 $ \boldsymbol{x} $, 使得$(\boldsymbol{I}+\boldsymbol{A}) \boldsymbol{x}=\boldsymbol{0},$即得$\boldsymbol{x}=-\boldsymbol{A x},$
两边取范数得$\|\boldsymbol{x}\|=\|-\boldsymbol{A} \boldsymbol{x}\| =\|\boldsymbol{A} \boldsymbol{x}\|\leqslant\|\boldsymbol{A}\|\|\boldsymbol{x}\| .$ 由于 $ \|\boldsymbol{x}\| \neq 0 $, 所以
$\|\boldsymbol{A}\| \geqslant 1,$ 与条件 $ \|\boldsymbol{A}\|<1 $ 矛盾, 因而矩阵 $ \boldsymbol{I}+\boldsymbol{A} $ 是非奇异的.
\end{tcolorbox}
\end{tcolorbox}

\begin{tcolorbox}[enhanced,colback=10,colframe=9,breakable,coltitle=green!25!black,title=2024]
  
设 $ I \in R^{n \times n} $ 为单位矩阵, ${A} \in R^{n \times n} $,证明: 当 $ \|A\|<1 $ 时, $ I+A $ 可逆, 且
$$
\left\|(I+A)^{-1}\right\| \leq \frac{1}{1-\|A\|} .
$$
\tcblower

($\boldsymbol{I}+\boldsymbol{A} $ 可逆上一题已经证得)

由于 $\boldsymbol{I}+\boldsymbol{A} $ 可逆,所以$(\boldsymbol{I}+\boldsymbol{A})^{-1}(\boldsymbol{I}+\boldsymbol{A})=\boldsymbol{I}$. 展开后有$(\boldsymbol{I}+\boldsymbol{A})^{-1}+(\boldsymbol{I}+\boldsymbol{A})^{-1} \boldsymbol{A}=\boldsymbol{I}$.

于是$(\boldsymbol{I}+\boldsymbol{A})^{-1}=\boldsymbol{I}-(\boldsymbol{I}+\boldsymbol{A})^{-1} \boldsymbol{A}$,两边同时取范数得

$$
\left\|(\boldsymbol{I}+\boldsymbol{A})^{-1}\right\|=\left\|\boldsymbol{I}-(\boldsymbol{I}+\boldsymbol{A})^{-1} \boldsymbol{A}\right\| \leqslant\|\boldsymbol{I}\|+\left\|(\boldsymbol{I}+\boldsymbol{A})^{-1} \boldsymbol{A}\right\|
\leqslant 1+\left\|(\boldsymbol{I}+\boldsymbol{A})^{-1}\right\| \cdot\|\boldsymbol{A}\|,
$$

所以
$$
(1-\|\boldsymbol{A}\|)\left\|(\boldsymbol{I}+\boldsymbol{A})^{-1}\right\| \leqslant 1 \Longrightarrow\left\|(\boldsymbol{I}+\boldsymbol{A})^{-1}\right\| \leqslant \frac{1}{1-\|\boldsymbol{A}\|} .
$$

方法二:利用 $\frac{1}{1+x}=\sum\limits_{n=0}^{\infty}(-1)^nx^n,x\in(-1,1)$推广到矩阵中

使用级数展开: 当 $\|A\| < 1$ 时,可以考虑使用Neumann级数(即幂级数)展开 $(I + A)^{-1}$:
   $$
   (I + A)^{-1} = I - A + A^2 - A^3 + \cdots
   $$
   这个级数在 $\|A\| < 1$ 时收敛.

计算级数的范数:对上述级数每一项的范数求和,可以得到:
   $$
   \|(I + A)^{-1}\| \leq \|I\| + \|A\| + \|A^2\| + \|A^3\| + \cdots
   $$
   利用矩阵的范数性质,有 $\|A^k\| \leq \|A\|^k$,于是:
   $$
   \|(I + A)^{-1}\| \leq 1 + \|A\| + \|A\|^2 + \|A\|^3 + \cdots = \frac{1}{1 - \|A\|}
   $$
其中使用了几何级数求和公式 $\sum\limits_{k=0}^\infty x^k = \frac{1}{1-x}$,其中 $|x| < 1$.

因此,当 $\|A\| < 1$ 时,$I + A$ 是可逆的,并且有 $\|(I+A)^{-1}\| \leq \frac{1}{1-\|A\|}$.得证.
\end{tcolorbox}

\begin{tcolorbox}[enhanced,colback=10,colframe=9,breakable,coltitle=green!25!black,title=2024]
  
设 $\boldsymbol{A} \in R^{n \times n}, \boldsymbol{A}=\boldsymbol{A}^{T} $ ,
试证明: $ \|\boldsymbol{A}\|_{2} \leq\|\boldsymbol{A}\|_{\infty} $
\tcblower

 设 $ \lambda_0 $ 是 $\boldsymbol{A}^{\mathrm{T}} \boldsymbol{A} $ 的最大特征值, 对应的特征向量为 $\boldsymbol{x} \neq 0 $, 于是有$\boldsymbol{A}^{\mathrm{T}} \boldsymbol{A x}=\lambda_0 \boldsymbol{x},$两边同时取范数得$\|\boldsymbol{A}^{\mathrm{T}} \boldsymbol{A x}\|=\|\lambda_0 \boldsymbol{x}\|=\lambda_0\|\boldsymbol{x}\|$.由于 $\boldsymbol{A}=\boldsymbol{A}^{T}$,则

 $$\lambda_0\|\boldsymbol{x}\|=\|\boldsymbol{A^2 x}\|\leqslant \|\boldsymbol{A^2}\|\cdot \|\boldsymbol{ x}\|\leqslant \|\boldsymbol{A}\|\cdot \|\boldsymbol{A}\|\cdot \|\boldsymbol{x}\|=\|\boldsymbol{A}\|^2 \cdot \|\boldsymbol{x}\|$$

 由于$\boldsymbol{x} \neq 0 $,所以$\lambda_0\leqslant \|\boldsymbol{A}\|^2$.当算子范数取无穷范数时,即有$\|\boldsymbol{A}\|_{\infty}\geqslant \sqrt{\lambda_0}$.而当算子范数取的是二范数时,
$$\|\boldsymbol{A}\|_{2}=\sqrt{\lambda_{\max}(\boldsymbol{A}^{\mathrm{T}} \boldsymbol{A })} =\sqrt{\lambda_0}$$
 
因此
$$
\|\boldsymbol{A}\|_{2}=\sqrt{\lambda_0} \leqslant\|\boldsymbol{A}\|_{\infty} .
$$
\end{tcolorbox}


\begin{tcolorbox}[enhanced,colback=10,colframe=9,breakable,coltitle=green!25!black,title=2024]
  
设 $ I \mathrm{~A} \in R^{n \times n},\|A\|<1 $, 记
$$
S_{k}=I+A+A^{2}+\cdots+A^{k},
$$

其中 $ I $ 为单位矩阵, 证明:
(1) $ I-A $ 可逆;
(2) $ \lim\limits _{k \rightarrow \infty} S_{k}=(I-A)^{-1} $.
\tcblower

(1) 首先注意到对于任何矩阵 $ A $ 和其范数 $ \|A\| $ ,有 $ \|A\|<1 $ .我们利用范数的性质来证明 $ I-A $ 可逆.
由 $ \|A\|<1 $ 可知,矩阵 $ A $ 的谱半径 $ \rho(A) $ 也小于 1 .这是因为谱半径 (即 $ A $ 的最大绝对特征值) 不超过矩阵的任意范数.于是,矩阵 $ I-A $ 的特征值为 $ 1-\lambda $ ,其中 $ \lambda $ 是 $ A $ 的一个特征值.即有:
 $$ \left|\lambda\right| \leqslant \rho(\boldsymbol{A}) \leqslant\|\boldsymbol{A}\|<1 $$

因为 $ |\lambda|<1 $ ,所以 $ 1-\lambda \neq 0 $ ,即 $ I-A $ 的所有特征值都不为 0 .这表明 $ I-A $ 是非奇异的,即 $ I-A $ 可逆.

(2) 我们先计算 $ (I-A) S_{k} $ :
$$
\begin{aligned}
(I-A) S_{k} & =(I-A)\left(I+A+A^{2}+\cdots+A^{k}\right) \\
& =I+A+A^{2}+\cdots+A^{k}-\left(A+A^{2}+A^{3}+\cdots+A^{k}+A^{k+1}\right) \\
& =I-A^{k+1} .
\end{aligned}
$$
由于 $ \left\|A^{k+1}\right\| \leq\|A\|^{k+1} $ 且 $ \|A\|<1 $ ,所以$\lim\limits_{k\to\infty}\|A\|^{k+1}=0 $, 也即$ A^{k+1} $ 当 $ k \rightarrow \infty $ 时趋向于零矩阵$(\lim\limits_{k\to\infty}A^{k+1}=\boldsymbol{O} )$.因此,有:
$$
\lim _{k \rightarrow \infty}(I-A) S_{k}=\lim _{k \rightarrow \infty}(I-A^{k+1})=I .
$$
由于 $ I-A $ 可逆,那么左乘其逆矩阵得到:
$$
S_{k}=(I-A)^{-1}(I-A) S_{k}=(I-A)^{-1}\left(I-A^{k+1}\right) .
$$
随着 $ k \rightarrow \infty, A^{k+1} \rightarrow O$ :
$$
\lim _{k \rightarrow \infty} S_{k}=(I-A)^{-1}(I-O)=(I-A)^{-1} .
$$
这样,我们便证明了 $\lim\limits _{k \rightarrow \infty} S_{k}=(I-A)^{-1} $.
\end{tcolorbox}






\begin{tcolorbox}[enhanced,colback=10,colframe=9,breakable,coltitle=green!25!black,title=2024]
  
 用 Jacobi 迭代方法求下列方程组的解
$
\left\{\begin{array}{c}
5 x_{1}-x_{2}+x_{3}=5 \\
x_{1}-10 x_{2}-2 x_{3}=-11 \\
-x_{1}+2 x_{2}+10 x_{3}=11
\end{array}\right.
$

(1) 求Jacobi 迭代矩阵 $B_{J}$

(2) 证明迭代方法收敛:

(3) 假设初值 $ x^{0}=(0,0,0)^{T} $, 试计算 $ x^{1} $ 和 $ x^{2} $, 保留 4 位有效数字

\begin{center}
\begin{tabular}{|c|c|c|c|}
\hline & $ x_{1} $ & $ x_{2} $ & $ x_{3} $ \\
\hline$ x^{0} $ & 0 & 0 & 0 \\
\hline$ x^{\prime} $ & & & \\
\hline$ x^{2} $ & & & \\
\hline
\end{tabular}
\end{center}
\tcblower

首先,我们将原方程组写为矩阵形式 $A x = b$,其中

$$
A = \begin{bmatrix}
5 & -1 & 1 \\
1 & -10 & -2 \\
-1 & 2 & 10
\end{bmatrix}, \quad \boldsymbol{b} = \begin{bmatrix} 5 \\ -11 \\ 11 \end{bmatrix}.
$$

 (1) 求Jacobi 迭代矩阵 $B_J$

在Jacobi迭代中,迭代矩阵 $B_J$ 定义为 $B_J =D^{-1}(L+U)=I-D^{-1}A$,其中 $D$ 是 $A$ 的对角部分,$L$ 是下三角部分(不含对角线),$U$ 是上三角部分(不含对角线).

对于矩阵 $A$,我们有:

$$
D = \begin{bmatrix} 5 & 0 & 0 \\ 0 & -10 & 0 \\ 0 & 0 & 10 \end{bmatrix}, \quad
L = \begin{bmatrix} 0 & 0 & 0 \\ 1 & 0 & 0 \\ -1 & 2 & 0 \end{bmatrix}, \quad
U = \begin{bmatrix} 0 & -1 & 1 \\ 0 & 0 & -2 \\ 0 & 0 & 0 \end{bmatrix}.
$$

因此,

$$
B_J = D^{-1}(L+U) = \begin{bmatrix} 1/5 & 0 & 0 \\ 0 & -1/10 & 0 \\ 0 & 0 & 1/10 \end{bmatrix}\begin{bmatrix} 0 & -1 & 1 \\ 1 & 0 & -2 \\ -1 & 2 & 0 \end{bmatrix}= \begin{bmatrix} 0 & 1/5 & -1/5 \\ 1/10 & 0 & -1/5 \\ 1/10 & -1/5 & 0 \end{bmatrix}.
$$

因此,Jacobij迭代矩阵 $ B_{J} $ 为:
$
B_{J}=\left[\begin{array}{ccc}
0 & 0.2 & -0.2 \\
0.1 & 0 & 0.2 \\
0.1 & -0.2 & 0
\end{array}\right]
$

(2) 证明迭代方法收敛

Jacobi方法收敛的充分条件之一是矩阵 $B_J$ 的谱半径 $\rho(B_J) < 1$.我们需要计算 $B_J$ 的谱半径.

$$|\lambda I-B_J|= \begin{array}{|ccc|} \lambda & -\frac15 & \frac15 \\ -\frac{1}{10} & \lambda & \frac{1}{5} \\ -\frac{1}{10} &\frac{1}{5} & \lambda \end{array}=\lambda^3-\frac{1}{25}\lambda=\lambda(\lambda+\frac15)(\lambda-\frac15)$$

求得特征值分别为$-\frac 15,0,\frac15$, 由于$\rho(B_J)=\frac 15 < 1$,则迭代方法收敛.

 (3) 假设初值 $ x^{0}=(0,0,0)^{T}$,计算 $ x^{1} $ 和 $ x^{2} $

迭代公式为 $\boldsymbol{x}^{(k+1)} = B_J \boldsymbol{x}^{(k)} + \boldsymbol{f}$,其中 $\boldsymbol{f} = D^{-1} \boldsymbol{b}$.首先计算 $\boldsymbol{f}$:

$$
\boldsymbol{f} = \begin{bmatrix} 1/5 & 0 & 0 \\ 0 & -1/10 & 0 \\ 0 & 0 & 1/10 \end{bmatrix} \begin{bmatrix} 5 \\ -11 \\ 11 \end{bmatrix} = \begin{bmatrix} 1 \\ 1.1 \\ 1.1 \end{bmatrix}.
$$

先计算 $x^{1}$ : $\boldsymbol{x}^{1} = B_J \boldsymbol{x}^{0} + \boldsymbol{f} = \boldsymbol{f}$, 然后用 $x^{1}$ 来计算 $x^{2}$:
即$\boldsymbol{x}^{2} = B_J \boldsymbol{x}^{1} + \boldsymbol{f}$

根据计算,我们得到: $x^{1} = (1.0000, 1.1000, 1.1000)^T$, $x^{2} = (1.0000, 0.9800, 0.9800)^T$

填入表格中:

$$
\begin{array}{|c|c|c|c|}
\hline & x_1 & x_2 & x_3 \\
\hline x^{0} & 0 & 0 & 0 \\
\hline x^{1} & 1.0000 & 1.1000 & 1.1000 \\
\hline x^{2} & 1.0000 & 0.9800 & 0.9800 \\
\hline
\end{array}
$$




也可以分别从上式三个方程中分离出 $ x_{1}, x_{2} $ 和 $ x_{3} $
$
\left\{\begin{array}{l}
x_{1}=0.2 x_{2}-0.2 x_{3}+1 \\
x_{2}=0.1 x_{1}-0.2 x_{3}+1.1 \\
x_{3}=0.1 x_{1}-0.2 x_{2}+1.1
\end{array}\right.
$
据此建立迭代公式
$$
\left\{\begin{array}{l}
x_{1}^{(k+1)}=0.2 x_{2}^{(k)}-0.2 x_{3}^{(k)}+1\\
x_{2}^{(k+1)}=0.1 x_{1}^{(k)}-0.2 x_{3}^{(k)}+1.1 \\
x_{3}^{(k+1)}=0.1 x_{1}^{(k)}-0.2 x_{2}^{(k)}+1.1
\end{array}\right.
$$
取迭代初值 $ x_{1}^{(0)}=x_{2}^{(0)}=x_{3}^{(0)}=0 $, 然后计算.
\end{tcolorbox}

\begin{tcolorbox}[enhanced,colback=10,colframe=9,breakable,coltitle=green!25!black,title=2024]
 设 $ \boldsymbol{A} \in \mathbf{R}^{n \times n}, \boldsymbol{B} \in \mathbf{R}^{n \times n} $ 均为非奇异矩阵, 证明:
$$
\left\|\boldsymbol{A}^{-1}-\boldsymbol{B}^{-1}\right\| \leqslant\left\|\boldsymbol{A}^{-1}\right\| \cdot\|\boldsymbol{A}-\boldsymbol{B}\| \cdot\left\|\boldsymbol{B}^{-1}\right\| .
$$
 \tcblower
由条件得
$$
\boldsymbol{A}^{-1}-\boldsymbol{B}^{-1}=\boldsymbol{A}^{-1}\left(I-\boldsymbol{A B}^{-1}\right)=\boldsymbol{A}^{-1}(\boldsymbol{B}-\boldsymbol{A}) \boldsymbol{B}^{-1},
$$
两边取范数得
$$
\left\|\boldsymbol{A}^{-1}-\boldsymbol{B}^{-1}\right\|=\left\|\boldsymbol{A}^{-1}(\boldsymbol{B}-\boldsymbol{A}) \boldsymbol{B}^{-1}\right\| \leqslant\left\|\boldsymbol{A}^{-1}\right\| \cdot\|\boldsymbol{B}-\boldsymbol{A}\| \cdot\left\|\boldsymbol{B}^{-1}\right\| .
$$
\end{tcolorbox}


\begin{tcolorbox}[enhanced,colback=10,colframe=9,breakable,coltitle=green!25!black,title=2024]
 设 $ \boldsymbol{x}^{(k)} \in \mathbf{R}^{n}, k=0,1,2, \cdots, \boldsymbol{x}^{*} \in \mathbf{R}^{n}, \boldsymbol{B} \in \mathbf{R}^{n \times n} $.

(1) 给出向量序列 $ \boldsymbol{x}^{(k)}(k=0,1,2, \cdots) $ 收敛于向量 $ \boldsymbol{x}^{*} $ 的定义;

(2) 设 $ \lim\limits _{k \rightarrow \infty} \boldsymbol{x}^{(k)}=\boldsymbol{x}^{*} $, 证明: $ \lim\limits _{k \rightarrow \infty} B \boldsymbol{x}^{(k)}=\boldsymbol{B} \boldsymbol{x}^{*} $.

 \tcblower

设 $ \|\cdot\| $ 为 $ \mathbf{R}^{n} $ 中的一种范数.

(1) 如果
$$
\lim _{k \rightarrow \infty}\left\|\boldsymbol{x}^{(k)}-x^{*}\right\|=0,
$$
则称向量序列 $ \left\{\boldsymbol{x}^{(k)}\right\}_{k=0}^{\infty} $ 收敛于向量 $ \boldsymbol{x}^{*} $.

(2) 因 $ \lim\limits _{k \rightarrow \infty} x^{(k)}=x^{*} $, 则 $ \lim\limits _{k \rightarrow \infty}\left\|x^{(k)}-x^{*}\right\|=0 $. 又
$$
\left\|\boldsymbol{B} \boldsymbol{x}^{(k)}-\boldsymbol{B} \boldsymbol{x}^{*}\right\|=\left\|\boldsymbol{B}\left(\boldsymbol{x}^{(k)}-\boldsymbol{x}^{*}\right)\right\| \leqslant\|\boldsymbol{B}\| \cdot\left\|\boldsymbol{x}^{(k)}-\boldsymbol{x}^{*}\right\|,
$$
所以
$$
\begin{aligned}
\lim _{k \rightarrow \infty}\left\|\boldsymbol{B} \boldsymbol{x}^{(k)}-\boldsymbol{B} \boldsymbol{x}^{*}\right\| & \leqslant \lim _{k \rightarrow \infty}\|\boldsymbol{B}\| \cdot\left\|\boldsymbol{x}^{(k)}-\boldsymbol{x}^{*}\right\| \\
& =\|\boldsymbol{B}\| \lim _{k \rightarrow \infty}\left\|\boldsymbol{x}^{(k)}-\boldsymbol{x}^{*}\right\|=0,
\end{aligned}
$$
所以 $ \lim\limits _{k \rightarrow \infty} B \boldsymbol{x}^{(k)}=\boldsymbol{B} \boldsymbol{x}^{*} $.

 \end{tcolorbox}


 \begin{tcolorbox}[enhanced,colback=10,colframe=9,breakable,coltitle=green!25!black,title=2024]
设 $ \boldsymbol{A} \in \mathbf{R}^{n \times n} $, 证明:
$$
\|\boldsymbol{A}\|_{2} \leqslant \sqrt{\|\boldsymbol{A}\|_{1}\|\boldsymbol{A}\|_{\infty}} ;
$$
 \tcblower
设 $ \lambda $ 是矩阵 $ \boldsymbol{A}^{\mathrm{T}} \boldsymbol{A} $ 的最大特征值, 对应的特征向量为 $ \boldsymbol{y} \neq 0 $, 则有
$$
\|\boldsymbol{A}\|_{2}=\sqrt{\lambda}, \boldsymbol{A}^{\mathrm{T}} \boldsymbol{A} \boldsymbol{y}=\lambda \boldsymbol{y} .
$$
对 $ \boldsymbol{A}^{\mathrm{T}} \boldsymbol{A} \boldsymbol{y}=\lambda \boldsymbol{y} $ 两边取无穷范数, 得
$$
\lambda\|\boldsymbol{y}\|_{\infty}=\left\|\boldsymbol{A}^{\mathrm{T}} \boldsymbol{A} \boldsymbol{y}\right\|_{\infty} \leqslant\left\|\boldsymbol{A}^{\mathrm{T}}\right\|_{\infty}\|\boldsymbol{A}\|_{\infty}\|\boldsymbol{y}\|_{\infty}=\|\boldsymbol{A}\|_{1}\|\boldsymbol{A}\|_{\infty}\|\boldsymbol{y}\|_{\infty},
$$
即
$$
\lambda \leqslant\|\boldsymbol{A}\|_{1}\|\boldsymbol{A}\|_{\infty},
$$
因而
$$
\|\boldsymbol{A}\|_{2} \leqslant \sqrt{\|\boldsymbol{A}\|_{1}\|\boldsymbol{A}\|_{\infty}} .
$$

 \end{tcolorbox}


  \begin{tcolorbox}[enhanced,colback=10,colframe=9,breakable,coltitle=green!25!black,title=2024]
 给定线性方程组 $ \boldsymbol{A x}=\boldsymbol{b} $, 这里 $ \boldsymbol{A} \in \mathbf{R}^{n \times n} $ 为非奇异矩阵, $ \boldsymbol{b} \in \mathbf{R}^{n}, \boldsymbol{x} \in \mathbf{R}^{n} $.设有下面的迭代格式:
$$
\boldsymbol{x}^{(k+1)}=\boldsymbol{x}^{(k)}+\omega\left(\boldsymbol{b}-\boldsymbol{A} \boldsymbol{x}^{(k)}\right), \quad k=0,1,2, \cdots,\quad(*)
$$
其中 $ \omega \neq 0 $ 为常数.

(1) 证明: 如果迭代格式 $(*)$ 收敛, 则迭代序列 $ \left\{\boldsymbol{x}^{(k)}\right\}_{k=0}^{\infty} $ 收敛于方程 $ \boldsymbol{A x}=\boldsymbol{b} $的解;

(2) 设 $ n=2, \boldsymbol{A}=\left[\begin{array}{ll}4 & 1 \\ 1 & 4\end{array}\right] $, 问 $ \omega $ 取何值时迭代格式 $(*)$ 收敛?
 \tcblower
 (1) 设迭代格式 $(*)$ 收敛, 不妨设 $ \lim\limits _{k \rightarrow \infty} x^{(k)}=x^{*} $. 在 $(*)$ 式两边取极限得
$$
\boldsymbol{x}^{*}=\boldsymbol{x}^{*}+\omega\left(\boldsymbol{b}-\boldsymbol{A} \boldsymbol{x}^{*}\right),
$$

由于 $ \omega \neq 0 $, 所以 $ \boldsymbol{b}-\boldsymbol{A} \boldsymbol{x}^{*}=\mathbf{0} $, 即 $ \boldsymbol{x}^{*} $ 是方程 $ \boldsymbol{A x}=\boldsymbol{b} $ 的解.

(2) 将 $(*)$ 改写为 $ \boldsymbol{x}^{(k+1)}=(\boldsymbol{I}-\omega \boldsymbol{A}) \boldsymbol{x}^{(k)}+\omega \boldsymbol{b} $. 根据迭代法收敛定理可知该迭代格式收敛的充要条件是 $ \rho(\boldsymbol{I}-\omega \boldsymbol{A})<1 $. 迭代矩阵 $ \boldsymbol{I}-\omega \boldsymbol{A} $ 的特征方程是
$$
|\lambda \boldsymbol{I}-(\boldsymbol{I}-\omega \boldsymbol{A})|=\left|\begin{array}{cc}
\lambda-(1-4 \omega) & \omega \\
\omega & \lambda-(1-4 \omega)
\end{array}\right|=0,
$$

展开得
$$
[\lambda-(1-4 \omega)]^{2}-\omega^{2}=0,
$$

求得上述方程的根为 $ \lambda_{1}=1-3 \omega, \lambda_{2}=1-5 \omega $. 
所以 $\rho(\boldsymbol{I}-\omega \boldsymbol{A})=\max \{|1-3 \omega|,|1-5\omega|\} $.

令 $ \left|\lambda_{1}\right|<1 $, 则 $ 0<\omega<\frac{2}{3} $; 令 $ \left|\lambda_{2}\right|<1 $,则 $ 0<\omega<\frac{2}{5} $. 故当 $ 0<\omega<\frac{2}{5} $ 时迭代格式 $(*)$ 收敛.

 \end{tcolorbox}

\begin{tcolorbox}[enhanced,colback=8,colframe=7,breakable,coltitle=green!25!black,title=2024]
 (1) 试用简单迭代法的理论证明: 对于任意 $ x_{0} \in[0,4] $, 由迭代格式
$
x_{k+1}=\sqrt{2+x_{k}}, \quad k=0,1,2, \cdots
$
得到的序列 $ \left\{x_{k}\right\}_{k=0}^{\infty} $ 均收敛于同一个数 $ x^{*} $;

(2) 你能否判定对于任意 $ x_{0} \in[0, \infty) $, 由上述迭代得到的序列 $ \left\{x_{k}\right\}_{k=0}^{\infty} $ 也收敛于数 $ x^{*} $ ?

\tcblower
 (1) 迭代函数 $ \varphi(x)=\sqrt{2+x} $ 满足:
 
(a) 当 $ x \in[0,4] $ 时, $ 0<\sqrt{2} \leqslant \varphi(x) \leqslant \sqrt{6}<4 $;

(b) $ \left|\varphi^{\prime}(x)\right|=\frac{1}{2 \sqrt{2+x}} \leqslant \frac{1}{2 \sqrt{2}}<1, \forall x \in[0,4] $.

所以对任意 $ x_{0} \in[0,4] $, 迭代格式均收敛于同一个数.

(2) 对任意 $ x_{0} \in[0, \infty) $, 不妨设 $ x_{0}>2 $.

(a) 当 $ x \in\left[0, x_{0}\right] $ 时, $ 0<\sqrt{2} \leqslant \varphi(x) \leqslant \sqrt{2+x_{0}}<x_{0} $;

(b) $ \left|\varphi^{\prime}(x)\right|=\frac{1}{2 \sqrt{2+x}} \leqslant \frac{1}{2 \sqrt{2}}<1, \forall x \in\left[0, x_{0}\right] $.

所以对任意 $ x_{0} \in[0, \infty) $, 迭代格式均收敛于同一个数.
\end{tcolorbox}

\begin{tcolorbox}[enhanced,colback=8,colframe=7,breakable,coltitle=green!25!black,title=2024]
1. 解方程 $ 12-3 x+2 \cos x=0 $ 的迭代格式为 $ x_{n+1}=4+\frac{2}{3} \cos x_{n} $.

(1) 证明: 对任意 $ x_{0} \in \mathbf{R} $, 均有 $ \lim\limits _{n \rightarrow \infty} x_{n}=x^{*} $ ( $ x^{*} $ 为方程的根);

(2) 此迭代法的收敛阶是多少?
 \tcblower
 (1) 迭代函数 $ \varphi(x)=4+\frac{2}{3} \cos x $, 对任意 $ x \in \mathbf{R} $

$$
4-\frac{2}{3} \leqslant 4+\frac{2}{3} \cos x \leqslant 4+\frac{2}{3},\quad x \in(+\infty,-\infty)
$$
$$
\varphi(x) \in\left[4-\frac{2}{3}, 4+\frac{2}{3}\right] \subset(-\infty,+\infty) \\
$$
又因为:
$$
\varphi^{\prime}(x)=-\frac{2}{3} \sin x, \quad L=\max _{-\infty<x<\infty}\left|\varphi^{\prime}(x)\right|=\frac{2}{3}<1
$$

故迭代公式在 $ (-\infty, \infty) $ 满足收敛性定理, 即 $ \left\{x_{k}\right\} $ 收敛于方程的根 $ x^{*} $.

(2) 由
$$
\lim _{k \rightarrow \infty} \frac{x^{*}-x_{k+1}}{x^{*}-x_{k}}=\lim _{k \rightarrow \infty} \frac{\varphi\left(x^{*}\right)-\varphi\left(x_{k}\right)}{x^{*}-x_{k}}=\varphi^{\prime}\left(x^{*}\right)=-\frac{2}{3} \sin x^{*} \neq 0
$$
可知迭代线性收敛.
 \end{tcolorbox}


\begin{tcolorbox}[enhanced,colback=8,colframe=7,breakable,coltitle=green!25!black,title=2024]

2. 设参数 $ a>0 $, 写出用 Newton 法解方程 $ x^{2}-a=0 $ 和方程 $ 1-\frac{a}{x^{2}}=0 $的迭代公式, 分别记为 $ x_{k+1}=\varphi_{1}\left(x_{k}\right) $ 和 $ x_{k+1}=\varphi_{2}\left(x_{k}\right) $, 确定常数 $ c_{1} $ 和 $ c_{2} $, 使迭代法
$$
x_{k+1}=c_{1} \varphi_{1}\left(x_{k}\right)+c_{2} \varphi_{2}\left(x_{k}\right), \quad k=0,1,2, \cdots
$$
产生的序列 $ \left\{x_{k}\right\} $ 三阶收敛到 $ \sqrt{a} $.

 \tcblower
 由题意可知, 对于方程 $ x^{2}-a=0 $, 记 $ f_{1}(x)=x^{2}-a $, 则 $ f_{1}^{\prime}(x)=2 x $, 因此该方程的 Newton 法的迭代函数为
$$
\varphi_{1}(x)=x-\frac{f_{1}(x)}{f_{1}^{\prime}(x)}=\frac{x}{2}+\frac{a}{2 x}
$$

同理对于方程 $ 1-\frac{a}{x^{2}}=0 $, 记 $ f_{2}(x)=1-\frac{a}{x^{2}} $, 则 $ f_{2}^{\prime}(x)=\frac{2 a}{x^{3}} $, 因此该方程的 Newton 法的迭代函数为
$$
\varphi_{2}(x)=x-\frac{f_{2}(x)}{f_{2}^{\prime}(x)}=\frac{3 x}{2}-\frac{x^{3}}{2 a}
$$
这两个函数 Newton 法迭代公式分别是 $ x_{k+1}=\varphi_{1}\left(x_{k}\right) $ 和 $ x_{k+1}=\varphi_{2}\left(x_{k}\right) $, 所以迭代函数
$$
\varphi(x)=c_{1} \varphi_{1}(x)+c_{2} \varphi_{2}(x)=\frac{c_{1}}{2}\left(x+\frac{a}{x}\right)+\frac{c_{2}}{2}\left(3 x-\frac{x^{3}}{a}\right)
$$
且求得 $ \varphi^{\prime}(x)=\frac{c_{1}}{2}\left(1-\frac{a}{x^{2}}\right)+\frac{c_{2}}{2}\left(3-\frac{3 x^{2}}{a}\right), \varphi^{\prime \prime}(x)=\frac{a c_{1}}{x^{3}}-\frac{3 c_{2} x}{a} $, 要使该迭代格式产生的序列 $ \left\{x_{k}\right\} $ 三阶收敛到 $ \sqrt{a} $, 需要满足以下条件
$$
\varphi(\sqrt{a})=\sqrt{a}, \quad \varphi^{\prime}(\sqrt{a})=0, \quad \varphi^{\prime \prime}(\sqrt{a})=0
$$
因此代入可得
$$
\left\{\begin{array}{l}
\varphi(\sqrt{a})=\left(c_{1}+c_{2}\right) \sqrt{a}=\sqrt{a} \\
\varphi^{\prime}(x)=\frac{c_{1}}{2}\left(1-\frac{a}{a}\right)+\frac{c_{2}}{2}\left(3-\frac{3 a}{a}\right)=0 \\
\varphi^{\prime \prime}(\sqrt{a})=\left(c_{1}-3 c_{2}\right) \frac{1}{\sqrt{a}}=0
\end{array}\right.
$$
联立解得 $ c_{1}=\frac{3}{4}, c_{2}=\frac{1}{4} $, 所以迭代函数为
$$
\varphi(x)=\frac{3}{4} \varphi_{1}(x)+\frac{1}{4} \varphi_{2}(x)
$$
此时该迭代函数满足 $ \varphi(\sqrt{a})=\sqrt{a}, \varphi^{\prime}(\sqrt{a})=0, \varphi^{\prime \prime}(\sqrt{a})=0 $, 进一步还可验证 $ \varphi^{\prime \prime \prime}(\sqrt{a})=-\frac{3}{a} \neq 0 $, 所以得到了求 $ \sqrt{a} $ 的三阶收敛的迭代公式
$$
x_{k+1}=\frac{3}{8}\left(x_{k}+\frac{a}{x_{k}}\right)+\frac{1}{8}\left(3 x_{k}-\frac{x_{k}^{3}}{a}\right), \quad k=0,1,2, \cdots
$$
 \end{tcolorbox}


\begin{tcolorbox}[enhanced,colback=8,colframe=7,breakable,coltitle=green!25!black,title=2024]

应用牛顿法于方程 $ f(x)=x^{n}-a=0 $ 和 $ f(x)=1-\frac{a}{x^{n}}=0 $, 分别导出求 $ \sqrt[n]{a} $ 的迭代公式,并求
$$
\lim _{k \rightarrow \infty}\left(\sqrt[n]{a}-x_{k+1}\right) /\left(\sqrt[n]{a}-x_{k}\right)^{2} .
$$
\tcblower
若 $ f(x)=x^{n}-a $, 则
$$
f^{\prime}(x)=n x^{n-1}, \quad f^{\prime \prime}(x)=n(n-1) x^{n-2} .
$$
因为 $ x^{*}=\sqrt[n]{a} $ 为方程 $ f(x)=0 $ 的根, 所以牛顿迭代公式为
$$
x_{k+1}=x_{k}-\frac{f\left(x_{k}\right)}{f^{\prime}\left(x_{k}\right)}=x_{k}-\frac{x_{k}^{n}-a}{n x_{k}^{n-1}}=\frac{(n-1) x_{k}^{n}+a}{n x_{k}^{n-1}} \text {, }
$$
故
$$
\lim _{k \rightarrow \infty} \frac{\sqrt[n]{a}-x_{k+1}}{\left(\sqrt[n]{a}-x_{k}\right)^{2}}=-\frac{f^{\prime \prime}(\sqrt[n]{a})}{2 f^{\prime}(\sqrt[n]{a})}=\frac{n(n-1)(\sqrt[n]{a})^{n-2}}{2 n(\sqrt[n]{a})^{n-1}}=\frac{n-1}{2 \sqrt[n]{a}}
$$
若 $ f(x)=1-\frac{a}{x^{n}} $, 则
$$
f^{\prime}(x)=\frac{a n}{x^{n+1}}, \quad f^{\prime \prime}(x)=-\frac{a n(n+1)}{x^{n+2}} .
$$
因为 $ x^{*}=\sqrt[n]{a} $ 为方程 $ f(x)=0 $ 的根, 所以牛顿迭代公式为
$$
x_{k+1}=x_{k}-\frac{f\left(x_{k}\right)}{f^{\prime}\left(x_{k}\right)}=x_{k}-\frac{1-\frac{a}{x_{k}^{n}}}{\frac{a \cdot n}{x_{k}^{n+1}}}=x_{k}-\frac{x_{k}^{n+1}-a x_{k}}{a n}=\frac{(a n+a) x_{k}-x_{k}^{n+1}}{a n} \text {, }
$$
故
$$
\lim _{k \rightarrow \infty} \frac{\sqrt[n]{a}-x_{k+1}}{\left(\sqrt[n]{a}-x_{k}\right)^{2}}=-\frac{f^{\prime \prime}(\sqrt[n]{a})}{2 f^{\prime}(\sqrt[n]{a})}=-\frac{-\frac{a n(n+1)}{(\sqrt[n]{a})^{n+2}}}{2 \cdot \frac{a n}{(\sqrt[n]{a})^{n+1}}}=\frac{n+1}{2 \sqrt[n]{a}}
$$

 \end{tcolorbox}

\begin{tcolorbox}[enhanced,colback=8,colframe=7,breakable,coltitle=green!25!black,title=2024]

设 $ a>0, x_{0}>0 $, 证明迭代公式
$
x_{k+1}=\dfrac{x_{k}\left(x_{k}^{2}+3 a\right)}{\left(3 x_{k}^{2}+a\right)}
$
是计算 $ \sqrt{a} $ 的三阶方法.
\tcblower
分析: 本题应说明 $ \left\{x_{k}\right\} $ 的极限为 $ \sqrt{a} $, 并且 $ \lim _{k \rightarrow \infty} \frac{x_{k+1}-\sqrt{a}}{\left(x_{k}-\sqrt{a}\right)^{3}}= $ $ c(\neq 0) $ 才行.第一个结论可按数列极限理论来证, 第二个结论可按收敛阶的定义计算或按收敛阶的定理来证.

证法 1\;  (1)先证明 $ \lim _{k \rightarrow \infty} x_{k}=\sqrt{a} $.
显然,由迭代公式知, $ a>0, x_{0}>0 $ 时, $ x_{k}>0(k=1,2, \cdots) $.一方面,由
$$
\begin{aligned}
x_{1}-\sqrt{a}= & \frac{x_{0}\left(x_{0}^{2}+3 a\right)}{\left(3 x_{0}^{2}+a\right)}-\sqrt{a}= \\
& \frac{x_{0}^{3}+3(\sqrt{a})^{2} x_{0}-3 \sqrt{a} x_{0}^{2}-(\sqrt{a})^{3}}{3 x_{0}^{2}+a}= \\
& \frac{\left(x_{0}-\sqrt{a}\right)^{3}}{3 x_{0}^{2}+a}
\end{aligned}
$$
可以看出,当 $ x_{0} \geqslant \sqrt{a} $ 时, $ x_{1}-\sqrt{a} \geqslant 0, x_{1} \geqslant \sqrt{a} $; 当 $ 0<x_{0} \leqslant \sqrt{a} $时, $ x_{1}-\sqrt{a} \leqslant 0, x_{1} \leqslant \sqrt{a} $. 利用归纳法易证: 当 $ x_{0} \geqslant \sqrt{a} $ 时, $ x_{k} \geqslant $ $ \sqrt{a}(k=1,2, \cdots) $; 当 $ 0<x_{0} \leqslant \sqrt{a} $ 时, $ x_{k} \leqslant \sqrt{a}(k=1,2, \cdots) $.另一方面, 由
$$
\begin{aligned}
\frac{x_{1}}{x_{0}}= & \frac{x_{0}^{2}+3 a}{3 x_{0}^{2}+a}= \\
& \frac{3 x_{0}^{2}+a+2\left(a-x_{0}^{2}\right)}{3 x_{0}^{2}+a}=1+\frac{2\left(a-x_{0}^{2}\right)}{3 x_{0}^{2}+a}
\end{aligned}
$$
知, 当 $ x_{0} \geqslant \sqrt{a} $ 时, $ \frac{x_{1}}{x_{0}} \leqslant 1, x_{1} \leqslant x_{0} $; 当 $ 0<x_{0} \leqslant \sqrt{a} $ 时, $ \frac{x_{1}}{x_{0}} \geqslant 1 $, 即 $ x_{1} \geqslant x_{0} $. 再次使用归纳法可得: 当 $ x_{0} \geqslant \sqrt{a} $ 时, $ \left\{x_{k}\right\} $ 单调减小; 当 0 $ <x_{0} \leqslant \sqrt{a} $ 时, $ \left\{x_{k}\right\} $ 单调增加. 利用单调有界原理知, 不论 $ x_{0} \geqslant \sqrt{a} $还是 $ 0<x_{0} \leqslant \sqrt{a} $, 数列 $ \left\{x_{k}\right\} $ 均有极限. 令该极限为 $ l $, 对原迭代式两边取极限, 得
$$
l=\frac{l\left(l^{2}+3 a\right)}{3 l^{2}+a}
$$
解得 $ l_{1}=0, l_{2}=\sqrt{a}, l_{3}=-\sqrt{a} $. 由题意知 $ l=\sqrt{a} $, 即 $ \left\{x_{k}\right\} $ 收玫于 $ \sqrt{a} $.

(2)证明 $ \left\{x_{k}\right\} $ 三阶收敛.由于
$$
\begin{aligned}
\lim _{k \rightarrow \infty} \frac{\sqrt{a}-x_{k+1}}{\left(\sqrt{a}-x_{k}\right)^{3}}= & \lim _{k \rightarrow \infty} \frac{\sqrt{a}-\frac{x_{k}^{3}+3 a x_{k}}{3 x_{k}^{2}+a}}{\left(\sqrt{a}-x_{k}\right)^{3}}= \\
& \lim _{k \rightarrow \infty} \frac{\left(\sqrt{a}-x_{k}\right)^{3}}{\left(\sqrt{a}-x_{k}\right)^{3}\left(3 x_{k}^{2}+a\right)}= \\
& \lim _{k \rightarrow \infty} \frac{1}{3 x_{k}^{2}+a}=\frac{1}{4 a} \neq 0
\end{aligned}
$$
所以原迭代公式是求 $ \sqrt{a} $ 的三阶方法.

证法 2 \; 利用关于收敛阶的定理来证明.可证明, 对于
$$
\begin{array}{c}
\varphi(x)=\frac{x\left(x^{2}+3 a\right)}{3 x^{2}+a}, \quad x>0 \\
\varphi(\sqrt{a})=\sqrt{a}, \quad \varphi^{\prime}(\sqrt{a})=\varphi^{\prime \prime}(\sqrt{a})=0, \quad \varphi^{\prime \prime \prime}(\sqrt{a})=\frac{3}{2 a} \neq 0
\end{array}
$$
由此可说明原方法是三阶的.
 \end{tcolorbox}



\begin{tcolorbox}[enhanced,colback=8,colframe=7,breakable,coltitle=green!25!black,title=2024]
 给定方程 $ x+\mathrm{e}^{-x}-4=0 $.
 
(1) 分析该方程存在几个根;

(2) 用迭代法求出这些根 (精确到 4 位有效数字), 并说明所用迭代格式为什么是收敛的.
 \tcblower
(1) 记 $ f(x)=x+\mathrm{e}^{-x}-4 $, 则 $ f^{\prime}(x)=1-\mathrm{e}^{-x} $, 令 $ f^{\prime}(x)=0 $ 得 $ x=0 $.当 $ x>0 $ 时 $ f^{\prime}(x)>0 $, 当 $ x<0 $ 时 $ f^{\prime}(x)<0 $, 因此 $ x=0 $ 为 $ f(x) $ 的极小值点. 又 $ f(-2)=\mathrm{e}^{2}-6>0, f(-1)=\mathrm{e}-5<0, f(3)=\mathrm{e}^{-3}-1<0, f(4)=\mathrm{e}^{-4}>0 $, 所以方程 $ f(x)=0 $ 有两个实根 $ x_{1}^{*} \in(3,4), x_{2}^{*} \in(-2,-1) $.
(2) 构造迭代格式:
\begin{equation*}
    x_{k+1}=4-\mathrm{e}^{-x_{k}}, \quad k=0,1, \cdots,\tag{1}
\end{equation*}


取初值 $ x_{0}=3.5 $, 计算得 $ x_{1}=3.96980, x_{2}=3.98112, x_{3}=3.98134 $, 所以 $ x_{1}^{*} \approx 3.981 $.
记 $ \varphi_{1}(x)=4-\mathrm{e}^{-x} $, 则 $ \varphi_{1}^{\prime}(x)=\mathrm{e}^{-x}>0 $.
当 $ x \in[3,4] $ 时, 有
$$
\varphi_{1}(x) \in\left[\varphi_{1}(3), \varphi_{1}(4)\right]=\left[4-\mathrm{e}^{-3}, 4-\mathrm{e}^{-4}\right] \subset[3,4],
$$
且 $ \left|\varphi_{1}^{\prime}(x)\right| \leqslant \mathrm{e}^{-3}<1 $, 所以迭代格式(1)对任意初值 $ x_{0} \in[3,4] $ 均收敛于 $ x_{1}^{*} $.
构造迭代格式:
\begin{equation*}
    x_{k+1}=-\ln \left(4-x_{k}\right), \quad k=0,1, \cdots,\tag{2}
\end{equation*}
取 $ x_{0}=-1.5 $, 计算得 $ x_{1}=-1.70475, x_{2}=-1.74130, x_{3}=-1.74769, x_{4}=-1.74880 $, $ x_{5}=-1.74899 $, 所以 $ x_{2}^{*} \approx-1.749 $.
记 $ \varphi_{2}(x)=-\ln (4-x) $, 则
$$
\varphi_{2}^{\prime}(x)=\frac{1}{4-x}>0, \quad x \in[-2,-1] .
$$
当 $ x \in[-2,-1] $ 时, 有
$$
\varphi_{2}(x) \in\left[\varphi_{2}(-2), \varphi_{2}(-1)\right]=[-\ln 6,-\ln 5] \subset[-2,-1],
$$
且 $ \left|\varphi_{2}^{\prime}(x)\right| \leqslant \frac{1}{5}<1 $, 所以迭代格式(2)对任意 $ x \in[-2,-1] $ 均收敛于 $ x_{2}^{*} $.
 \end{tcolorbox}



\begin{tcolorbox}[enhanced,colback=8,colframe=7,breakable,coltitle=green!25!black,title=2024]

已知方程 $ x^{3}+2 x-1=0 $ 在区间 $ [0,1] $ 上有唯一实根 $ x^{*} $, 证明对任意初值 $ x_{0} \in[0,1] $, 迭代格式
$$
x_{k+1}=\frac{2 x_{k}^{3}+1}{3 x_{k}^{2}+2}, \quad k=0,1, \cdots
$$
均收敛于 $ x^{*} $, 并分析该迭代格式的收敛阶数.

 \tcblower
 方法 1: 方程的 Newton 迭代格式为
$$
x_{k+1}=x_{k}-\frac{x_{k}^{3}+2 x_{k}-1}{3 x_{k}^{2}+2}=\frac{2 x_{k}^{3}+1}{3 x_{k}^{2}+2} .
$$
记 $ f(x)=x^{3}+2 x-1 $, 则 $ f(0) \cdot f(1)=-2<0 $; 当 $ x \in[0,1] $ 时, $ f^{\prime}(x)=3 x^{2}+2>0 $;当 $ x \in(0,1) $ 时, $ f^{\prime \prime}(x)=6 x>0 ; 0-\frac{f(0)}{f^{\prime}(0)}=\frac{1}{2}<1,1-\frac{f(1)}{f^{\prime}(1)}=\frac{3}{5}>0 $. 所以对任意初值 $ x_{0} \in[0,1] $, Newton 迭代收敛于方程在 $ [0,1] $ 中的根. 该迭代格式是 2 阶收敛的.

方法 2: 记 $ \varphi(x)=\frac{2 x^{3}+1}{3 x^{2}+2} $, 则原方程可以改写为 $ x=\varphi(x) $. 对 $ \varphi(x) $ 求导得
$$
\varphi^{\prime}(x)=\frac{6 x\left(x^{3}+2 x-1\right)}{\left(3 x^{2}+2\right)^{2}} .
$$

当 $ x \in[0,1] $ 时, 容易验证
$$
\left|\varphi^{\prime}(x)\right|<1, \quad 0 \leqslant \varphi(x) \leqslant \frac{2 x^{2}+1}{3 x^{2}+2}<1,
$$

故对任意 $ x_{0} \in[0,1] $, 迭代格式收敛于方程在 $ [0,1] $ 中的根. 又 $ \varphi^{\prime}\left(x^{*}\right)=0, \varphi^{\prime \prime}\left(x^{*}\right) \neq 0 $,所以迭代格式为 2 阶收敛.
 \end{tcolorbox}

\begin{tcolorbox}[enhanced,colback=8,colframe=7,breakable,coltitle=green!25!black,title=2024]
 证明: 若 $ f(x) $ 在其零点 $ \xi $ 的某邻域中有 2 阶连续导数, 且 $ f^{\prime}(\xi) \neq 0, f^{\prime \prime}(\xi) \neq $ 0 , 则 Newton 法是 2 阶局部收敛的.
\tcblower

Newton 迭代格式为
$$
x_{k+1}=x_{k}-\frac{f\left(x_{k}\right)}{f^{\prime}\left(x_{k}\right)}, \quad k=0,1,2, \cdots,
$$
迭代函数为
$$
\varphi(x)=x-\frac{f(x)}{f^{\prime}(x)}
$$
对其求导, 得
$$
\varphi^{\prime}(x)=1-\frac{\left[f^{\prime}(x)\right]^{2}-f(x) f^{\prime \prime}(x)}{\left[f^{\prime}(x)\right]^{2}}=\frac{f(x) f^{\prime \prime}(x)}{\left[f^{\prime}(x)\right]^{2}},
$$
易知 $ \varphi^{\prime}(\xi)=0 $. 又在解的邻域内 $ \varphi(x) $ 有 2 阶连续导数, 所以 Newton 迭代格式至少是 2 阶局部收敛的.
对 $ \varphi(x) $ 再求一次导数得
$$
\varphi^{\prime \prime}(x)=f^{\prime}(x) \cdot \frac{f^{\prime \prime}(x)}{\left[f^{\prime}(x)\right]^{2}}+f(x)\left(\frac{f^{\prime \prime}(x)}{\left[f^{\prime}(x)\right]^{2}}\right)^{\prime}
$$
易知
$$
\varphi^{\prime \prime}(\xi)=\frac{f^{\prime \prime}(\xi)}{f^{\prime}(\xi)}
$$
当 $ f^{\prime \prime}(\xi) \neq 0 $ 时, $ \varphi^{\prime \prime}(\xi) \neq 0 $, 即 Newton 迭代格式是 2 阶收敛的.
\end{tcolorbox}


\begin{tcolorbox}[enhanced,colback=8,colframe=7,breakable,coltitle=green!25!black,title=2024]
已知 $ x=\varphi(x) $ 在区间 $ [a, b] $ 上只有一个实根, 且当 $ x \in[a, b] $ 时, $ \left|\varphi^{\prime}(x)\right| \geqslant L>1(L $为常数), 问如何将 $ x=\varphi(x) $ 化为适合于迭代的形式.
\tcblower
由已知 $ x=\varphi(x) $ 在区间 $ [a, b] $ 上只有一个实根, 故其反函数 $ x=\psi(x) $ 存在, 又
$$
\varphi^{\prime}(x)=\frac{1}{\psi^{\prime}(x)}
$$
当 $ x \in[a, b] $ 时, $ \left|\varphi^{\prime}(x)\right| \geqslant L>1 $ ( $ L $ 为常数), 所以有
$$
\left|\psi^{\prime}(x)\right|=\frac{1}{\left|\varphi^{\prime}(x)\right|}<1
$$
因此 $ x=\psi(x) $ 收敛.
\end{tcolorbox}

  \begin{tcolorbox}[enhanced,colback=10,colframe=9,breakable,coltitle=green!25!black,title=2024]

 设 $ A $ 为 $ n $ 阶非奇异矩阵且有分解式 $ A=L U $, 其中 $ L $ 是单位下三角阵, $ U $ 为上三角阵, 求证 $ A $ 的所有顺序主子式均不为零.
 \tcblower
证明: 由题意可知, 若将 $ A=L U $ 分解式中的 $ L $ 与 $ U $ 分块
$$
L=\left[\begin{array}{cc}
L_{k \times k} & 0_{k \times(n-k)} \\
L_{(n-k) \times k} & L_{(n-k) \times(n-k)}
\end{array}\right], \quad U=\left[\begin{array}{cc}
U_{k \times k} & U_{k \times(n-k)} \\
0_{(n-k) \times k} & U_{(n-k) \times(n-k)}
\end{array}\right]
$$

其中, $ L_{k \times k} $ 为 $ k $ 阶单位下三角阵, $ U_{k \times k} $ 为 $ k $ 阶上三角阵, 则 $ A $ 的 $ k $ 阶顺序主子式为 $ A_{k}=L_{k \times k} U_{k \times k} $, 又由 $ A $ 为 $ n $ 阶非奇异矩阵, 因此 $ |A|=a_{11}^{(1)} a_{22}^{(2)} \cdots a_{n n}^{(n)} \neq 0 $,则
$$
\left|A_{k}\right|=\left|L_{k \times k}\right| \cdot\left|U_{k \times k}\right|=a_{11}^{(1)} a_{22}^{(2)} \cdots a_{k k}^{(k)} \neq 0
$$
证得 $ A $ 的所有顺序主子式均不为零.

 \end{tcolorbox}

  \begin{tcolorbox}[enhanced,colback=10,colframe=9,breakable,coltitle=green!25!black,title=2024]

设 $ A \in \mathbf{R}^{n \times n} $ 是对称矩阵, $ \lambda_{1} $ 和 $ \lambda_{n} $ 分别是 $ A $ 的按模最大和按模最小的特征值 $ \left(\lambda_{n} \neq 0\right) $, 则 $ \operatorname{cond}_{2}(A)=\left|\frac{\lambda_{1}}{\lambda_{n}}\right| $.
 \tcblower
证明: 由题意可知
$$
\operatorname{cond}(A)_{2}=\|A\|_{2} \cdot\left\|A^{-1}\right\|_{2}=\sqrt{\lambda_{\max }\left(A^{\mathrm{T}} A\right)} \cdot \sqrt{\lambda_{\max }\left(\left(A^{-1}\right)^{\mathrm{T}} A^{-1}\right)}
$$
由于 $ A \in \mathbf{R}^{n \times n} $ 是对称矩阵, 则
$$
\operatorname{cond}(A)_{2}=\sqrt{\lambda_{\max }\left(A^{2}\right)} \cdot \sqrt{\lambda_{\max }\left(A^{-1}\right)^{2}}=\frac{\sqrt{\lambda_{\max }\left(A^{2}\right)}}{\sqrt{\lambda_{\min }\left(A^{2}\right)}}=\left|\frac{\lambda_{1}}{\lambda_{n}}\right|
$$
其中, $ \lambda_{1} $ 和 $ \lambda_{n} $ 分别是 $ A $ 的按模最大和按模最小的特征值 $ \left(\lambda_{n} \neq 0\right) $.

 \end{tcolorbox}


  \begin{tcolorbox}[enhanced,colback=10,colframe=9,breakable,coltitle=green!25!black,title=2024]

已知矩阵 $ A=\left[\begin{array}{ccc}1 & a & a \\ a & 1 & a \\ a & a & 1\end{array}\right] $, 求 $ a $ 为何值时, $ A $ 为正定阵; $ a $ 为何值时对于线性方程组 $ A x=b $, 采用 Jacobi 迭代法和 Gauss-Seidel 迭代法收敛.
 \tcblower
 
 由题意可知, 为了使 $ A $ 为正定矩阵, 则只需满足顺序主子式
$$
\begin{array}{l}
\Delta_{1}=1>0, \quad \Delta_{2}=\left|\begin{array}{cc}
1 & a \\
a & 1
\end{array}\right|=1-a^{2}>0 ,\quad
\Delta_{3}=\left|\begin{array}{lll}
1 & a & a \\
a & 1 & a \\
a & a & 1
\end{array}\right|=(2 a+1)(a-1)^{2}>0
\end{array}
$$
因此解得 $ -\frac{1}{2}<a<1 $.
若对于 Jacobi 迭代法的迭代矩阵 $ G_{J}=\left[\begin{array}{ccc}0 & -a & -a \\ -a & 0 & a \\ -a & -a & 0\end{array}\right] $, 其特征多项式

$$
\left|\begin{array}{ccc}
\lambda & a & a \\
a & \lambda & a \\
a & a & \lambda
\end{array}\right|=(\lambda-a)^{2}(\lambda+2 a)=0
$$

因此特征值为 $ \lambda_{1}=a, \lambda_{2}=a, \lambda_{3}=-2 a $, 由迭代法收敛的充分必要条件, Jacobi 迭代法的收敛充分必要条件是 $ \rho(G)<1 $, 即满足 $ |2 a|<1 $, 解得 $ -\frac{1}{2}<a<\frac{1}{2} $.
同样对于 Gauss-Seidel 迭代法的迭代矩阵
$$
G=\left[\begin{array}{lll}
1 & 0 & 0 \\
a & 1 & 0 \\
a & a & 1
\end{array}\right]^{-1}\left[\begin{array}{ccc}
0 & -a & -a \\
0 & 0 & -a \\
0 & 0 & 0
\end{array}\right]=\left[\begin{array}{ccc}
0 & -a & -a \\
0 & a^{2} & a^{2}-a \\
0 & a^{2}-a^{3} & 2 a^{2}-a^{3}
\end{array}\right]
$$

其特征多项式
$$
\left|\begin{array}{ccc}
\lambda & a & a \\
0 & \lambda-a^{2} & a-a^{2} \\
0 & a^{3}-a^{2} & \lambda-2 a^{2}+a^{3}
\end{array}\right|=\lambda\left[\lambda^{2}+\left(a^{3}-3 a^{2}\right) \lambda+a^{3}\right]=0
$$

若要使 Gauss-Seidel 迭代法收敛, 则对其三个特征值都需满足 $ \left|\lambda_{1}\right|<1,\left|\lambda_{2}\right|< $ $ 1,\left|\lambda_{3}\right|<1 $, 这里 $ \lambda_{1}=0, \lambda_{2}, \lambda_{3} $ 是方程 $ \lambda^{2}+\left(a^{3}-3 a^{2}\right) \lambda+a^{3}=0 $ 的两个根, 则由
条件只需满足
$$
\left|a^{3}-3 a^{2}\right|<1+a^{3}<2
$$

解得 $ -\frac{1}{2}<a<1 $, 此时系数矩阵 $ A $ 是对称正定矩阵.
 \end{tcolorbox}


  \begin{tcolorbox}[enhanced,colback=10,colframe=9,breakable,coltitle=green!25!black,title=2024]


 设 $ n $ 阶方阵 $ A=\left[a_{i j}\right]_{n \times n} $ 的对角线元素 $ a_{k k} \neq 0, k=1,2, \cdots, n $, 考虑利用迭代法求解线性方程组 $ A x=b $, 求证:
(1) Jacobi 迭代法收敛当且仅当 $ \left|\begin{array}{cccc}\lambda a_{11} & a_{12} & \cdots & a_{1 n} \\ a_{21} & \lambda a_{22} & \cdots & a_{2 n} \\ \vdots & \vdots & & \vdots \\ a_{n 1} & a_{n 2} & \cdots & \lambda a_{n n}\end{array}\right|=0 $ 的根 $ \lambda $ 均满足 $ |\lambda|<1 ; $的根 $ \lambda $ 均满足 $ |\lambda|<1 $.

(2) Gauss-Seidel 迭代法收敛当且仅当方程 $ \left|\begin{array}{cccc}\lambda a_{11} & a_{12} & \cdots & a_{1 n} \\ \lambda a_{21} & \lambda a_{22} & \cdots & a_{2 n} \\ \vdots & \vdots & & \vdots \\ \lambda a_{n 1} & \lambda a_{n 2} & \cdots & \lambda a_{n n}\end{array}\right|=0 $的根 $ \lambda $ 均满足 $ |\lambda|<1 $.

 \tcblower

证明 (1) 由题意可知, Jacobi 迭代的迭代矩阵为 $ G=D^{-1}(L+U) $, 则其特征方程
$$
|\lambda I-G|=\left|\lambda I-D^{-1}(L+U)\right|=\left|D^{-1}\right| \cdot|\lambda D-(L+U)|=0
$$
因此特征值满足 $ |\lambda D-(L+U)|=0 $, 而
$$
|\lambda D-(L+U)|=\left|\begin{array}{cccc}
\lambda a_{11} & a_{12} & \cdots & a_{1 n} \\
a_{21} & \lambda a_{22} & \cdots & a_{2 n} \\
\vdots & \vdots & & \vdots \\
a_{n 1} & a_{n 2} & \cdots & \lambda a_{n n}
\end{array}\right|=0
$$

由 Jacobi 迭代法收敛当且仅当 $ \rho(G)=\max\limits _{\lambda \in \sigma(A)}|\lambda|<1 $, 因此命题成立.
(2) Gauss-Seidel 迭代法的迭代矩阵为 $ G=(D-L)^{-1} U $, 则其特征方程
$$
|\lambda I-G|=\left|\lambda I-(D-L)^{-1} U\right|=\left|(D-L)^{-1}\right| \cdot|\lambda(D-L)-U|=0
$$
因此特征值满足 $ |\lambda(D-L)-U|=0 $, 而
$$
|\lambda(D-L)-U|=\left|\begin{array}{cccc}
\lambda a_{11} & a_{12} & \cdots & a_{1 n} \\
\lambda a_{21} & \lambda a_{22} & \cdots & a_{2 n} \\
\vdots & \vdots & & \vdots \\
\lambda a_{n 1} & \lambda a_{n 2} & \cdots & \lambda a_{n n}
\end{array}\right|=0
$$
由 Gauss-Seidel 迭代法收敛当且仅当 $ \rho(G)=\max\limits _{\lambda \in \sigma(A)}|\lambda|<1 $, 因此命题成立.
 \end{tcolorbox}


  \begin{tcolorbox}[enhanced,colback=10,colframe=9,breakable,coltitle=green!25!black,title=2024]
 解线性方程组 $ A x=b $ 的 Jacobi 迭代法的一种改进称为 JOR 方法, 其迭代公式为
$$
x^{(k+1)}=\omega B_{J} x^{(k)}+(1-\omega) x^{(k)}, \quad k=0,1,2, \cdots
$$
其中, $ B_{J} $ 是 Jacobi 迭代法的迭代矩阵, 试证明若 Jacobi 迭代法收敛, 则 JOR 方法对 $ 0<\omega \leqslant 1 $ 收敛.
 \tcblower

证明: 由题意可知, 设 $ B_{J} $ 的特征值为 $ \lambda\left(B_{J}\right) $, 则迭代矩阵 $ B=\omega B_{J}+(1-\omega) I $对应的特征值 $ \lambda(B)=\omega \lambda\left(B_{J}\right)+1-\omega $, 因此有
$$
|\lambda(B)|=\left|\omega \lambda\left(B_{J}\right)+1-\omega\right| \leqslant\left|\omega \lambda\left(B_{J}\right)\right|+|1-\omega|
$$
由于 Jacobi 迭代法收敛, 则 $ \lambda\left(B_{J}\right)<1 $, 而 $ 0<\omega \leqslant 1 $, 所以
$$
|\lambda(B)| \leqslant \omega\left|\lambda\left(B_{J}\right)\right|+1-\omega<1
$$
由迭代法收敛条件证得 JOR 方法收敛.
 \end{tcolorbox}


  \begin{tcolorbox}[enhanced,colback=10,colframe=9,breakable,coltitle=green!25!black,title=2024]
 设求解方程组 $ A x=b $ 的简单迭代法 $ x^{(k+1)}=G x^{(k)}+g(k=0,1,2, \cdots) $收敛, 证明对 $ 0<\omega<1 $, 迭代法 $ x^{(k+1)}=[(1-\omega) I+\omega G] x^{(k)}+\omega g(k=0,1 $, $ 2, \cdots) $ 收敛.
 \tcblower
证明 由题意可知, 设 $ B=(1-\omega) I+\omega G, \lambda(B), \lambda(G) $ 分别为 $ B $ 和 $ G $ 的特征值, 则显然
$$
\lambda(B)=(1-\omega)+\omega \lambda(G)
$$

由于简单迭代法收敛知 $ |\lambda(G)|<1 $, 又由 $ 0<\omega<1 $, 则 $ \lambda(B) $ 是 1 和 $ \lambda(G) $ 的加权平均, 因此
$$
|\lambda(B)|<|\lambda(G)|<1
$$

由迭代法的收敛条件, 迭代法 $ x^{(k+1)}=[(1-\omega) I+\omega G] x^{(k)}+\omega g, k=0,1,2, \cdots $收敛.

 \end{tcolorbox}



 \begin{tcolorbox}[enhanced,colback=10,colframe=9,breakable,coltitle=green!25!black,title=2024]
 给定线性方程组
$
\left[\begin{array}{lll}
a_{11} & a_{12} & a_{13} \\
a_{21} & a_{22} & a_{23} \\
a_{31} & a_{32} & a_{33}
\end{array}\right]\left[\begin{array}{l}
x_{1} \\
x_{2} \\
x_{3}
\end{array}\right]=\left[\begin{array}{l}
b_{1} \\
b_{2} \\
b_{3}
\end{array}\right],
$
其中 $ a_{i i} \neq 0, i=1,2,3 $.

(1) 写出用 Jacobi 迭代格式求解方程组的矩阵形式 $ \boldsymbol{x}^{(k+1)}=\boldsymbol{J} \boldsymbol{x}^{(k)}+\boldsymbol{f}_{J} $;

(2) 证明: 若 Jacobi 迭代矩阵 $ \boldsymbol{J} $ 满足 $ \|\boldsymbol{J}\|_{\infty}<1 $, 则求解该方程组的 Gauss-Seidel 迭代格式收敛.

\tcblower
 (1) 根据题意, 有
$$
\begin{aligned}
\boldsymbol{J} & =-\boldsymbol{D}^{-1}(\boldsymbol{L}+\boldsymbol{U})=-\left[\begin{array}{ccc}
\frac{1}{a_{11}} & 0 & 0 \\
0 & \frac{1}{a_{22}} & 0 \\
0 & 0 & \frac{1}{a_{33}}
\end{array}\right]\left[\begin{array}{ccc}
0 & a_{12} & a_{13} \\
a_{21} & 0 & a_{23} \\
a_{31} & a_{32} & 0
\end{array}\right]  =\left[\begin{array}{ccc}
0 & -\frac{a_{12}}{a_{11}} & -\frac{a_{13}}{a_{11}} \\
-\frac{a_{21}}{a_{22}} & 0 & -\frac{a_{23}}{a_{22}} \\
-\frac{a_{31}}{a_{33}} & -\frac{a_{32}}{a_{33}} & 0
\end{array}\right]
\end{aligned}
$$
$$
\boldsymbol{f}_{J}=\boldsymbol{D}^{-1} \boldsymbol{b}=\left[\begin{array}{c}
\frac{b_{1}}{a_{11}} \\
\frac{b_{2}}{a_{22}} \\
\frac{b_{3}}{a_{33}}
\end{array}\right] .
$$
(2) 若 $ \|\boldsymbol{J}\|_{\infty}<1 $, 则 $\displaystyle \max _{1 \leqslant i \leqslant 3}\left(\sum_{\substack{j=1 \\ j \neq i}}^{3}\left|-\frac{a_{i j}}{a_{i i}}\right|\right)<1 $, 即对 $ i=1,2,3 $ 有
$$
\sum_{\substack{j=1 \\ j \neq i}}^{3}\left|\frac{a_{i j}}{a_{i i}}\right|<1 \text { 或 } \sum_{\substack{j=1 \\ j \neq i}}^{3}\left|a_{i j}\right|<\left|a_{i i}\right|
$$
故原方程组系数矩阵为严格对角占优阵, 从而 Gauss-Seidel 格式收玫.

 \end{tcolorbox}
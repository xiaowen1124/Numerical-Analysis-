\newpage
\section{期中测试题5道}

1.用直接三角分解法(LU分解)求解以下线性方程组 $ \left\{\begin{array}{l}x-2 y+3 z=2, \\ 2 x-3 y+4 z=3, \\ 3 x-4 y+6 z=5 .\end{array}\right. $

2. 已知函数表如下, 应用二次多项式 $ P_{2}(x) $ 拟合函数 $ f(x) $, 求出二次多项式 $ P_{2}(x) $ 的表达式。
\begin{center}
\begin{tabular}{|c|c|c|c|c|c|}
\hline$ x $ & $ -{2} $ & $ -{1} $ & $ {0} $ & 1 & 2 \\
\hline $ {f}({x}) $ & 0 & $-1$ & $-2$ & $-1$ & 0 \\
\hline
\end{tabular}
\end{center}

3. 证明:当 $ x_{0}=1.5 $ 时,迭代法 $ x_{k+1}=\sqrt{\frac{10}{x_{k}+4}} $ 收敛于方程 $ f(x)=x^{3}+4 x^{2}-10=0 $ 在区间[0,2] 内唯一实根。

4. 试确定 $ a(a \neq 0) $, 使得求解方程组 $ \left[\begin{array}{ccc}a & 1 & 3 \\ 1 & a & 2 \\ -3 & 2 & a\end{array}\right]\left[\begin{array}{l}x \\ y \\ z\end{array}\right]=\left[\begin{array}{l}b_{1} \\ b_{2} \\ b_{3}\end{array}\right] $ 的Jacobi迭代格式收敛。

5. 求一个次数不超过4次的插值多项式, 使它满足:
$$
\begin{array}{l}
p(0)=f(0)=0, p(1)=f(1)=1, p^{\prime}(0)=f^{\prime}(0)=0, \\
p^{\prime}(1)=f^{\prime}(1)=2, p^{\prime \prime}(1)=f^{\prime \prime}(1)=0
\end{array}
$$
并写出其余项表达式(设存在5阶导数)

\section{数值积分单元测试题 (5 道题)}

1. 确定求积公式中的待定参数, 使其代数精度尽量高
$$
\int_{0}^{h} f(x) d x \approx \frac{h}{2}[f(0)+f(h)]+\alpha h^{2}\left[f^{\prime}(0)-f^{\prime}(h)\right]
$$

2. Newton-Cotes求积公式 $\displaystyle \int_{a}^{b} f(x) d x \approx \sum\limits_{k=0}^{n} A_{k} f\left(x_{k}\right) $, 则 $ \sum\limits_{k=0}^{n} A_{k}= $ $ \qquad $

3. 证明 Simpsion 数值积分公式的误差:
$$
\int_{a}^{b} f(x) d x-\frac{b-a}{6}\left[f(a)+4 f\left(\frac{a+b}{2}\right)+f(b)\right]=-\frac{\left(\frac{b-a}{2}\right)^{5}}{90} f^{(4)}(\xi)
$$

4. 给定求积公式 $ \int_{a}^{b} f(x) d x \approx(b-a) f\left(\frac{a+b}{2}\right) $.

(1)求该求积公式的代数精度;

(2)证明:存在 $ \eta \in(a, b) $, 使得
$$
\int_{a}^{b} f(x) d x-(b-a) f\left(\frac{a+b}{2}\right)=\frac{(b-a)^{3}}{24} f^{\prime \prime}(\eta)
$$

5. 考虑积分 $ I(f)=\int_{0}^{3} f(x) d x $ 及对应的求积公式 $ Q(f)=\frac{3}{4} f(0)+\frac{9}{4} f(2) $

(1)证明求积公式 $ Q(f) $ 是以 $ x_{0}=0, x_{1}=1, x_{2}=2 $ 为求积节点的插值

(2)型求积公式. $ (f) \approx Q(f) $ 的 代数精度

(3)设 $ f(x) \in C^{3}[0,3] $, 求截断误差 $ I(f)-Q(f) $ 形如 $ \alpha f^{(\beta)}(\xi) $ 的表达式,其中 $ \xi \in(0,3) , \alpha, \beta $ 为常数
\section{插值与拟合习题}
\subsection{课后习题}
\begin{tcolorbox}[breakable,enhanced,arc=0mm,outer arc=0mm,
		boxrule=0pt,toprule=1pt,leftrule=0pt,bottomrule=1pt, rightrule=0pt,left=0.2cm,right=0.2cm,
		titlerule=0.5em,toptitle=0.1cm,bottomtitle=-0.1cm,top=0.2cm,
		colframe=white!10!biru,colback=white!90!biru,coltitle=white,
            coltext=black,title =2024-03-10, title style={white!10!biru}, before skip=8pt, after skip=8pt,before upper=\hspace{2em},
		fonttitle=\bfseries,fontupper=\normalsize]
  
1. 已知等距插值节点,
$$
x_{0}<x_{1}<x_{2}<x_{3}, \quad x_{i+1}-x_{i}=h \quad({i}=0,1,2)
$$
且 $ f(x) $ 在 $ \left[x_{0}, x_{3}\right] $ 上有四阶连续导数, 证明 $ f(x) $ 的 Lagrange 插值多项式余项的误差界为

(1) 二次插值的误差界
$$
R_{2}=\max _{x_{0} \leq x \leq x_{2}}\left|f(x)-L_{2}(x)\right| \leq \frac{\sqrt{3}}{27} h^{3} \max _{x_{0} \leq x \leq x_{2}}\left|f^{\prime \prime \prime}(x)\right|
$$
(2) 三次插值的误差界
$$
R_{3}=\max _{x_{0} \leq x \leq x_{3}}\left|f(x)-L_{3}(x)\right| \leq \frac{1}{24} h^{4} \max _{x_{0} \leq x \leq x_{3}}\left|f^{(4)}(x)\right|
$$
 \tcblower

(1) 二次插值的误差界:

首先,我们知道二次 Lagrange 插值多项式为
$$
L_{2}(x)=f\left(x_{0}\right) \frac{(x-x_{1})(x-x_{2})}{(x_{0}-x_{1})(x_{0}-x_{2})}+f\left(x_{1}\right) \frac{(x-x_{0})(x-x_{2})}{(x_{1}-x_{0})(x_{1}-x_{2})}+f\left(x_{2}\right) \frac{(x-x_{0})(x-x_{1})}{(x_{2}-x_{0})(x_{2}-x_{1})}
$$

定义余项为 $ R_2(x)=f(x)-L_{2}(x) $,则存在 $ \xi \in(x_{0}, x_{2}) $,使得
$$
R_2(x)=\frac{f^{\prime \prime \prime}(\xi)}{3 !}(x-x_{0})(x-x_{1})(x-x_{2})
$$
令 $ x=x_{0}+t h $ ,因此,我们有


$$
\begin{aligned}
\max _{x_{0} \leqslant x \leqslant x_{2}} |R_2(x)| &=\max _{x_{0} \leqslant x \leqslant x_{2}}\left|\frac{f^{\prime \prime \prime}(x)}{6}(x-x_{0})(x-x_{1})(x-x_{2})\right| \\
&\leqslant \frac{1}{6} \max _{x_{0} \leqslant x \leqslant x_{2}}\left|f^{\prime \prime \prime}(x)\right| \cdot \max _{x_{0} \leqslant x \leqslant x_{2}}\left|\left(x-x_{0}\right)\left(x-x_{1}\right)\left(x-x_{2}\right)\right| \\
&=\frac{1}{6} h^{3} \max _{0 \leqslant t \leqslant 2}|t(t-1)(t-2)| \cdot \max _{x_{0} \leqslant x \leqslant x_{2}}\left|f^{\prime \prime \prime}(x)\right| \\
\end{aligned}
$$

令$g(t)=t(t-1)(t-2)=t^3-3t^2+2t, \quad 0 \leqslant t \leqslant 2$.
令$ g^{\prime}(t)=0 $, 即 $ 3 t^{2}-6 t+2=0 $, 解得
$$
t_{1}=1-\frac{1}{\sqrt{3}}, \quad t_{2}=1+\frac{1}{\sqrt{3}}
$$
由 $ g(0)=0, g\left(t_{1}\right)=\frac{2 \sqrt{3}}{9}, g\left(t_{2}\right)=-\frac{2 \sqrt{3}}{9}, g(2)=0 $知$\max\limits _{0 \leqslant t \leqslant 2} |g(t)|=\dfrac{2 \sqrt{3}}{9}$.于是
$$
R_{2}=\max _{x_{0} \leqslant x \leqslant x_{2}}|R_2(x)|=\max _{x_{0} \leqslant x \leqslant x_{2}}\left|f(x)-L_{2}(x)\right| \leqslant\frac{1}{6} h^{3}\cdot\frac{2 \sqrt{3}}{9} \max _{x_{0} \leqslant x \leqslant x_{2}}\left|f^{\prime \prime \prime}(x)\right|=\frac{\sqrt{3}}{27} h^{3}  \max _{x_{0} \leqslant x \leqslant x_{2}}\left|f^{\prime \prime \prime}(x)\right|
$$

(2) 三次插值的误差界:

三次 Lagrange 插值多项式为
$$
L_{3}(x)=f\left(x_{0}\right) \frac{(x-x_{1})(x-x_{2})(x-x_{3})}{(x_{0}-x_{1})(x_{0}-x_{2})(x_{0}-x_{3})}+\cdots
$$

定义余项为 $ R_3(x)=f(x)-L_{3}(x) $,同理可以得到
$$
R_{3}= \max _{x_{0} \leqslant x \leqslant x_{3}}|R_3(x)|=\max _{x_{0} \leqslant x \leqslant x_{3}}\left|f(x)-L_{3}(x)\right| =\max _{x_{0} \leqslant x \leqslant x_{3}}\left|\frac{f^{(4)}(x)}{4 !}\left(x-x_{0}\right)\left(x-x_{1}\right)\left(x-x_{2}\right)\left(x-x_{3}\right)\right|
$$


令 $ x=x_{0}+t h $ ,则

$$
\begin{aligned}
R_3 & =\max _{x_{0} \leqslant x \leqslant x_{3}}\left|\frac{f^{(4)}(x)}{4 !}\left(x-x_{0}\right)\left(x-x_{1}\right)\left(x-x_{2}\right)\left(x-x_{3}\right)\right| \\
& \leqslant \frac{1}{24} \max _{x_{0} \leqslant x \leqslant x_{3}}\left|f^{(4)}(x)\right| \cdot \max _{x_{0} \leqslant x \leqslant x_{3}}\left|\left(x-x_{0}\right)\left(x-x_{1}\right)\left(x-x_{2}\right)\left(x-x_{3}\right)\right| \\
& =\frac{h^{4}}{24} \max _{0 \leqslant t \leqslant 3}|t(t-1)(t-2)(t-3)| \cdot \max _{x_{0} \leqslant x \leqslant x_{3}}\left|f^{(4)}(x)\right|
\end{aligned}
$$

令$h(t)=t(t-1)(t-2)(t-3),t\in [0,3]$.为求解$|h(t)|$的最大值,不妨设$m=(t-1)(t-2),n=t(t-3)$,则易知$m=n+2$.由于$n=(t-\frac 32)^2-\frac 94 \in [-\frac 94,0]$,且$|h(t)|=|mn|=|n(n+2)|=|(n+1)^2-1|$,所以当$n=-1$时$|h(t)|$取最大值$1$.即$\max\limits _{0 \leqslant t \leqslant 3}|t(t-1)(t-2)(t-3)|=1$.因此我们得到

$$
R_3=\max _{x_{0} \leqslant x \leqslant x_{3}}\left|f(x)-L_{3}(x)\right| \leqslant \frac{1}{24} h^{4}\cdot \max _{x_{0} \leqslant x \leqslant x_{3}}\left|f^{(4)}(x)\right|
$$


\end{tcolorbox}


\begin{tcolorbox}[breakable,enhanced,arc=0mm,outer arc=0mm,
		boxrule=0pt,toprule=1pt,leftrule=0pt,bottomrule=1pt, rightrule=0pt,left=0.2cm,right=0.2cm,
		titlerule=0.5em,toptitle=0.1cm,bottomtitle=-0.1cm,top=0.2cm,
		colframe=white!10!biru,colback=white!90!biru,coltitle=white,
            coltext=black,title =2024-03-10, title style={white!10!biru}, before skip=8pt, after skip=8pt,before upper=\hspace{2em},
		fonttitle=\bfseries,fontupper=\normalsize]
  
2. 若 $ f(x)=a_{0}+a_{1} x+\cdots+a_{n-1} x^{n-1}+a_{n} x^{n} $ 有 $ n $ 个不同实根 $ x_{1}, {x}_{2}, {x}_{3}, \cdots $, $ x_{n} $. 证明:

$$
\sum_{j=1}^{n} \frac{x_{j}^{k}}{f^{\prime}\left(x_{j}\right)}=\left\{\begin{array}{ll}
0 & 0 \leq k \leq n-2 \\
a_{n}^{-1} & k=n-1
\end{array}\right.
$$

 \tcblower

 由已知条件知 $ f(x)=a_{n}\left(x-x_{1}\right)\left(x-x_{2}\right) \cdots\left(x-x_{n}\right) $.
记 $ g(x)=x^{k}, \omega_{n}(x)=\prod\limits_{j=1}^{n}\left(x-x_{j}\right) $, 则
$$
f(x)=a_n \prod\limits_{j=1}^{n}\left(x-x_{j}\right)=a_{n} \omega_{n}(x), \quad f^{\prime}\left(x_{j}\right)=a_{n} \omega_{n}^{\prime}\left(x_{j}\right),
$$
$$\begin{aligned}
\text { 其中 }\; \omega_{n}^{\prime}(x)&=  \left(x-x_{2}\right)\left(x-x_{3}\right) \cdots\left(x-x_{n}\right)+\left(x-x_{1}\right)\left(x-x_{3}\right) \cdots\left(x-x_{n}\right) \\
& +\cdots+\left(x-x_{1}\right)\left(x-x_{2}\right) \cdots\left(x-x_{n-1}\right)\\
&=\left(x_{j}-x_{1}\right)\left(x_{j}-x_{2}\right) \cdots\left(x_{j}-x_{j-1}\right)\left(x_{j}-x_{j+1}\right) \cdots\left(x_{j}-x_{n}\right) 
\end{aligned} $$
由差商的性质可得 $$ g\left[x_{1}, \cdots, x_{n}\right]=\sum\limits_{j=1}^{n}\frac{g(x_j)}{\left(x_{j}-x_{1}\right)\left(x_{j}-x_{2}\right) \cdots\left(x_{j}-x_{j-1}\right)\left(x_{j}-x_{j+1}\right) \cdots\left(x_{j}-x_{k}\right) }=\sum\limits_{j=1}^{k} \frac{g\left(x_{j}\right)}{\omega_{n}^{\prime}\left(x_{j}\right)}  $$
再根据差商与导数之间的关系有$ g\left[x_{1}, \cdots, x_{n}\right]=\frac{f^{(n-1)}(\xi)}{(n-1) !}$,其中$ \xi $ 介于 $ x_{1}, \cdots, x_{n} $ 之间,因此
$$
\begin{aligned}
\sum_{j=1}^{n} \frac{x_{j}^{k}}{f^{\prime}\left(x_{j}\right)} & =\sum_{j=1}^{n} \frac{x_{j}^{k}}{a_{n} \omega_{n}^{\prime}\left(x_{j}\right)}=\frac{1}{a_{n}} \sum_{j=1}^{n} \frac{x_{j}^{k}}{\omega_{n}^{\prime}\left(x_{j}\right)} \\
& =\frac{1}{a_{n}} g\left[x_{1}, x_{2}, \cdots, x_{n}\right] \\
& =\frac{1}{a_{n}} \frac{g^{(n-1)}(\xi)}{(n-1) !},
\end{aligned}
$$
其中 $ \xi $ 介于 $ x_{1}, x_{2}, \cdots, x_{n} $ 之间.
当 $ 0 \leqslant k \leqslant n-2 $ 时, $ g^{(n-1)}(x)=\dfrac{\mathrm{d}^{n-1}}{\mathrm{~d} x^{n-1}} x^{k}=0 $, 故 $ g^{(n-1)}(\xi)=0 $;
当 $ k=n-1 $ 时, $ g^{(n-1)}(x)=\dfrac{\mathrm{d}^{n-1}}{\mathrm{~d} x^{n-1}} x^{n-1}=(n-1) ! $, 故 $ g^{(n-1)}(\xi)=(n-1) ! $, 故有
$$
\sum_{j=1}^{n} \frac{x_{j}^{k}}{f^{\prime}\left(x_{j}\right)}=\frac{1}{a_{n}} \frac{g^{(n-1)}(\xi)}{(n-1) !}=\left\{\begin{array}{ll}
0, & 0 \leqslant k \leqslant n-2 ; \\
a_{n}^{-1}, & k=n-1 .
\end{array}\right.
$$

\end{tcolorbox}

\begin{tcolorbox}[breakable,enhanced,arc=0mm,outer arc=0mm,
		boxrule=0pt,toprule=1pt,leftrule=0pt,bottomrule=1pt, rightrule=0pt,left=0.2cm,right=0.2cm,
		titlerule=0.5em,toptitle=0.1cm,bottomtitle=-0.1cm,top=0.2cm,
		colframe=white!10!biru,colback=white!90!biru,coltitle=white,
            coltext=black,title =2024-03-10, title style={white!10!biru}, before skip=8pt, after skip=8pt,before upper=\hspace{2em},
		fonttitle=\bfseries,fontupper=\normalsize]
  
3. 设 $ x_{0}, x_{1}, \cdots x_{n} $ 为 $ n+1 $ 个互异插值节点, $ {l}_{{0}}({x}), {l}_{{1}}({x}), \cdots, {l}_{{n}}({x}) $ 为 Lagrange 插值基函数, 试证明:

(1) $ \sum\limits_{j=0}^{n} l_{j}(x)=1 $;

(2) $ \sum\limits_{j=0}^{n} x_{j}^{k} l_{j}(x) \equiv x^{k}, k=1,2, \cdots, n $;

(3) $ \sum\limits_{j=0}^{n}\left(x_{j}-x\right)^{k} l_{j}(x)=0, k=1,2, \cdots, n $;

(4) $ \sum\limits_{j=0}^{n} l_{j}(0) x_{j}^{k}=\left\{\begin{array}{ll}1 & k=0 \\ 0 & k=1,2, \cdots, n \\ (-1)^{n} x_{0} x_{1} \cdots x_{n} & k=n+1\end{array}\right. $;
 \tcblower
(1) 令 $ f(x)\equiv1,$ 则 $ y_{j}=f\left(x_{j}\right)=1, j=0,1, \cdots, n $;且函数 $ f(x) $ 的 $ n $ 次Lagrange插值多项式为
$$
L_{n}(x)=\sum_{j=0}^{n}  y_j \cdot l_{j}(x)=\sum_{j=0}^{n}   l_{j}(x)
$$
插值余项为
$$
R_{n}(x)=f(x)-L_{n}(x)=\frac{f^{(n+1)}(\xi)}{(n+1) !}  \omega_{n+1}(x)
$$
因为 $f(x)\equiv 1$,故$f^{(n+1)}(\xi)=0,$于是$R_{n}(x)=0 .$即$f(x)-L_{n}(x)=0$.亦即
$$
\sum_{j=0}^{n}  l_{j}(x)=1
$$

(2) 令 $ f(x)=x^{k}, k=0,1, \cdots, n $.则 $ y_{j}=f\left(x_{j}\right)=x_j^k$,  于是函数 $ f(x) $ 的 $ n $ 次插值多项式为
$$
L_{n}(x)=\sum_{j=0}^{n}  y_j \cdot l_{j}(x)=\sum_{j=0}^{n} x_{j}^{k} l_{j}(x)
$$
插值余项为
$$
R_{n}(x)=f(x)-L_{n}(x)=\frac{f^{(n+1)}(\xi)}{(n+1) !}  \omega_{n+1}(x)
$$
因为 $ k \leqslant n $, 则 $ f^{(n+1)}(x)=\dfrac{\mathrm{d}^{n+1}}{\mathrm{~d} x^{n+1}} x^{k}=0 $,
故$f^{(n+1)}(\xi)=0,$于是$R_{n}(x)=0 .$即$f(x)-L_{n}(x)=0$.亦即
$$
\sum_{j=0}^{n} x_{j}^{k} l_{j}(x)=x^{k}, \quad k=0,1, \cdots, n
$$

(3) 对 $ k=1,2, \cdots, n $, 由二项式定理得
$$
\begin{aligned}
\sum_{j=0}^{n}\left(x_{j}-x\right)^{k} l_{j}(x) & =\sum_{j=0}^{n}\left[l_{j}(x) \sum_{i=0}^{k}\left(\begin{array}{l}
k \\
i
\end{array}\right) x_{j}^{i}(-x)^{k-i}\right] \\
& =\sum_{j=0}^{n} \sum_{i=0}^{k}\left[\left(\begin{array}{l}
k \\
i
\end{array}\right) x_{j}^{i}(-x)^{k-i} l_{j}(x)\right] \\
& =\sum_{i=0}^{k} \sum_{j=0}^{n}\left[\left(\begin{array}{l}
k \\
i
\end{array}\right) x_{j}^{i}(-x)^{k-i} l_{j}(x)\right] \\
& =\sum_{i=0}^{k}\left[\left(\begin{array}{l}
k \\
i
\end{array}\right)(-x)^{k-i} \sum_{j=0}^{n} x_{j}^{i} l_{j}(x)\right] \\
& =\sum_{i=0}^{k}\left(\begin{array}{l}
k \\
i
\end{array}\right)(-x)^{k-i} x^{i} \\
& =(x-x)^{k}=0
\end{aligned}
$$


(4) 若函数 $ f(x) $ 在 $ [a, b] $ 上具有 $ n+1 $ 阶导数, 则有
$$
f(x)-L_n(x)=f(x)-\sum_{j=0}^{n} l_{j}(x) f\left(x_{j}\right)=\frac{f^{(n+1)}(\xi)}{(n+1) !} \omega_{n+1}(x)
$$
其中 $ \omega_{n+1}(x)=\left(x-x_{0}\right)\left(x-x_{1}\right) \cdots\left(x-x_{n}\right) $, $\xi \in (a,b)$


当 $ f(x)=1 $ 时,则$1-\sum\limits_{j=0}^{n} l_{j}(x) f\left(x_{j}\right)=0$.
进而有
$$
\sum_{j=0}^{n} l_{j}(0)=1
$$
当 $ f(x)=x^{k}(k=1,2, \cdots, n) $ 时, 有$x^{k}=\sum\limits_{j=0}^{n} l_{j}(x) x_{j}^{k}$.
将 $ x=0 $ 代入得
$$
\sum_{j=0}^{n} l_{i}(0) x_{j}^{k}=0
$$
当 $ f(x)=x^{n+1} $ 时, 有
$$
x^{n+1}=\sum_{j=0}^{n} l_{j}(x) x_{j}^{n+1}+\omega_{n+1}(x)
$$
将 $ x=0 $ 代入得
$$
\sum_{j=0}^{n} l_{j}(0) x_{j}^{n+1}=-\omega_{n+1}(0)=(-1)^{n} x_{0} x_{1} \cdots x_{n}
$$
综上,
$$ \sum\limits_{j=0}^{n} l_{j}(0) x_{j}^{k}=\left\{\begin{array}{ll}1 & k=0 \\ 0 & k=1,2, \cdots, n \\ (-1)^{n} x_{0} x_{1} \cdots x_{n} & k=n+1\end{array}\right. $$

\begin{tcolorbox}[title=补充一个推论(可直接用于上面的证明中)]
   若 $ f(x) $ 是次数不超过 $ n $ 的多项式,则它的 $ n $ 次 Lagrange 插值多项式就是它本身.
\tcblower
证明: 设 $ p_{n}(x) $ 是 $ f(x) $ 的满足插值条件 $ p_{n}\left(x_{i}\right)=f\left(x_{i}\right)(i=0,1,2, \cdots, n) $的 $ n $ 次 Lagrange 插值多项式. 因为 $ f(x) $ 是次数不超过 $ n $ 的多项式, 所以 $ f^{(n+1)}(x) \equiv 0 $.
则 $ p_{n}(x) $ 关于 $ f(x) $ 的余项为
$$
r_{n}(x)=f(x)-p_{n}(x)=\frac{\omega_{n+1}(x)}{(n+1) !} f^{(n+1)}(\xi) \equiv 0,
$$
于是 $ p_{n}(x) \equiv f(x) $. 证毕.
\end{tcolorbox}

\end{tcolorbox}




\begin{tcolorbox}[breakable,enhanced,arc=0mm,outer arc=0mm,
		boxrule=0pt,toprule=1pt,leftrule=0pt,bottomrule=1pt, rightrule=0pt,left=0.2cm,right=0.2cm,
		titlerule=0.5em,toptitle=0.1cm,bottomtitle=-0.1cm,top=0.2cm,
		colframe=white!10!biru,colback=white!90!biru,coltitle=white,
            coltext=black,title =2024-03-10, title style={white!10!biru}, before skip=8pt, after skip=8pt,before upper=\hspace{2em},
		fonttitle=\bfseries,fontupper=\normalsize]
  
  4. 设 $ p_{n}(x) $ 是 $ e^{x} $ 在区间 $ [0,1] $ 上的 Lagrange 型插值多项式,插值节点 $ x_{k}=\frac{k}{n}, k=0,1, \cdots, n $. 证明:
$$
\lim _{n \rightarrow \infty} \max _{0 \leqslant x \leqslant 1}\left|p_{n}(x)-e^{x}\right|=0 .
$$

 \tcblower
【证明】 设 $ f(x)=e^x, x \in[0,1],$于是插值余项$R_n(x)=f(x)-p_{n}(x)$,即

$$ f(x)-p_{n}(x)=\frac{f^{(n+1)}(\xi)}{(n+1) !} \prod_{i=0}^{n}\left(x-x_{i}\right) \quad \xi \in(\min \{x, 0\}, \max \{x, 1\}) $$
$$
f^{\prime}(x)=e^x, \quad f^{\prime \prime}(x)=e^x ,\quad f^{\prime \prime \prime}(x)=e^x, \cdots, f^{(n+1)}(x)=e^x
$$

当 $ x \in[0,1] $ 时, $ \left|f^{(n+1)}(x)\right| \leqslant e $ ,
且 $ \left|x-x_{i}\right| \leqslant 1, i=0,1,2, \cdots, n $.
于是当 $ x \in[0,1] $ 时, 有
$$
\left|f(x)-p_{n}(x)\right| \leqslant \frac{e}{(n+1) !}
$$
所以
$$
\max _{0 \leqslant x \leqslant 1}\left|p_n(x)-e^x\right| \leqslant \frac{e}{(n+1) !}
$$
$$
\lim _{n \rightarrow \infty} \max _{0 \leqslant x \leqslant 1}\left|p_n(x)-e^x\right| \leqslant \lim _{n \rightarrow \infty} \frac{e}{(n+1) !}=0 .
$$
由迫敛性即得证.
\end{tcolorbox}


\begin{tcolorbox}[breakable,enhanced,arc=0mm,outer arc=0mm,
		boxrule=0pt,toprule=1pt,leftrule=0pt,bottomrule=1pt, rightrule=0pt,left=0.2cm,right=0.2cm,
		titlerule=0.5em,toptitle=0.1cm,bottomtitle=-0.1cm,top=0.2cm,
		colframe=white!10!biru,colback=white!90!biru,coltitle=white,
            coltext=black,title =2024-03-10, title style={white!10!biru}, before skip=8pt, after skip=8pt,before upper=\hspace{2em},
		fonttitle=\bfseries,fontupper=\normalsize]
  
5. 设 $ {f}({x}) $ 在 $ [{a}, {b}] $ 上二阶导数连续, 且 $ {f}({a})=0, {f}({b})=0 $,证明:
$$
\max _{a \leqslant x \leqslant b}|f(x)| \leqslant \frac{(b-a)^{2}}{8} \max _{a \leqslant x \leqslant b}\left|f^{\prime \prime}(x)\right|
$$
 \tcblower
\textbf{方法一:}
由于 $ f(x) $ 在 $ [a, b] $ 上二阶导数连续, 所以 $ f(x) $ 在 $ [a, b] $ 连续, 根据最值定理知 $f$ 在 $ [a, b] $ 取得最大值和最小值.于是存在点 $ x_0 \in[a, b] $, 使得
$$
|f(x_0)|=\max _{x \in[a, b]}|f(x)| .
$$

  若 $ x_{0}=a $ 或 $ b $, 则结论显然成立. 
 
 若 $ a<x_{0}<b $,分析如下:当 $ f(x_0)>0 $ 时, 根据 $ f(x) \leq|f(x)| \leq f(x_0) $ 可知 $ f(x_0) $ 为 $ f(x) $ 在 $ [a, b] $ 上的最大值; 当 $ f(x_0)<0 $ 时, 根据 $ -f(x) \leq|f(x)| \leq-f(x_0) $ 可知 $ f(x) \geq f(x_0) $, 即 $ f(x_0) $ 为 $ f(x) $ 在 $ [a, b] $ 上的最小值. 总而言之, $ f(x_0) $ 必定为 $ f(x) $ 的最值, 再结合 $ x_0 \in(a, b) $, 就有 $ f^{\prime}(x_0)=0 $.
 
 利用带 Lagrange 余项的 Taylor 公式将 $ f(x) $ 在点 $ x_{0} $ 展开:
$$
\begin{array}{l}
0=f(a)=f\left(x_{0}\right)+f^{\prime}\left(x_{0}\right)\left(a-x_{0}\right)+\frac{1}{2} f^{\prime \prime}(\xi)\left(a-x_{0}\right)^{2}, \xi \in\left(a, x_{0}\right), \\
0=f(b)=f\left(x_{0}\right)+f^{\prime}\left(x_{0}\right)\left(b-x_{0}\right)+\frac{1}{2} f^{\prime \prime}(\eta)\left(b-x_{0}\right)^{2}, \eta \in\left(x_{0}, b\right),
\end{array}
$$

将 $f^{\prime}\left(x_{0}\right)=0 $ 代入上面两式, 并进行如下讨论

若 $ a<x_0 \leq \frac{a+b}{2} $, 可知 $$ |f(x_0)|=\left|-\frac{f^{\prime \prime}(\xi)}{2}(a-x_0)^{2}\right| \leq \frac{(b-a)^{2}}{8} \max _{x \in[a, b]}\left|f^{\prime \prime}(x)\right| $$
若 $ \frac{a+b}{2} \leq x_0<b $, 可知 $$ |f(x_0)|=\left|-\frac{f^{\prime \prime}(\eta)}{2}(b-x_0)^{2}\right| \leq \frac{(b-a)^{2}}{8} \max _{x \in[a, b]}\left|f^{\prime \prime}(x)\right| $$

综上可知 $$ \max _{a \leq x \leq b}|f(x)|=|f(x_0)| \leq \frac{(b-a)^{2}}{8} \max _{a \leq x \leq b}\left|f^{\prime \prime}(x)\right| $$

\textbf{方法二:}
以$ x=a $ 和 $ x=b $ 为插值节点,作函数 $ f(x) $ 的一次插值多项式:
$$
L_{1}(x)=f(a) \frac{x-b}{a-b}+f(b) \frac{x-b}{b-a},
$$

因为 $ f(a)=f(b)=0 $ ,则有 $ L_{1}(x)=0 $ ,且插值多项式$L_{1}(x)$的余项
$$
R_1(x)=f(x)-L_{1}(x)=\frac{f^{\prime \prime}(\xi)}{2}(x-a)(x-b),
$$

其中 $ \xi \in(\min \{x, a\}, \max \{x, b\}) $. 因而,
$$
f(x)=\frac{f^{\prime \prime}(\xi)}{2}(x-a)(x-b), x \in[a, b], \xi \in(a, b),
$$

当 $ x \in[a, b] $ 时,
$$
\begin{aligned}
|f(x)| &\leq \max_{x\in[a,b]} \left|\frac{f^{\prime \prime}(x)}{2}(x-a)(x-b)\right| \\
&\leq \frac{1}{2} \max _{x \in[a, b]}\left|f^{\prime \prime}(x)\right| \cdot \max _{x \in[a, b]}|(x-a)(x-b)| \\
&=\frac{(b-a)^{2}}{8} \max _{x \in[a, b]}\left|f^{\prime \prime}(x)\right|
\end{aligned}
$$

于是,有:
$$
\max _{a \leqslant x \leqslant b}|f(x)| \leqslant \frac{(b-a)^{2}}{8} \max _{a \leqslant x \leqslant b}\left|f^{\prime \prime}(x)\right| .
$$
\end{tcolorbox}



\begin{tcolorbox}[breakable,enhanced,arc=0mm,outer arc=0mm,
		boxrule=0pt,toprule=1pt,leftrule=0pt,bottomrule=1pt, rightrule=0pt,left=0.2cm,right=0.2cm,
		titlerule=0.5em,toptitle=0.1cm,bottomtitle=-0.1cm,top=0.2cm,
		colframe=white!10!biru,colback=white!90!biru,coltitle=white,
            coltext=black,title =2024-03-10, title style={white!10!biru}, before skip=8pt, after skip=8pt,before upper=\hspace{2em},
		fonttitle=\bfseries,fontupper=\normalsize]
  
6. 设 $ S(x) $ 是函数 $ f(x) $ 在区间 $ [0,2] $ 上满足第一类条件的三次样条, 并且
$$
S(x)=\left\{\begin{array}{ll}
2 x^{3}-3 x+4, & 0 \leq x<1, \\
(x-1)^{3}+b(x-1)^{2}+c(x-1)+3, & 1 \leq x \leq 2,
\end{array}\right.
$$
求 $ S^{\prime}(0) $ 和 $ S^{\prime}(2) $ 的值.
 \tcblower

取 $ x_{0}=0, x_{1}=1, x_{2}=2 $, 根据三次样条函数的定义, 有 $ S(x) \in C^{2}[0,2] $, 由 $ S(x) $ 及其导数的连续性, 即有: 
$$ S\left(x_{1}-0\right)=S\left(x_{1}+0\right), S^{\prime}\left(x_{1}-0\right)=S^{\prime}\left(x_{1}+0\right), S^{\prime \prime}\left(x_{1}-0\right)=S^{\prime \prime}\left(x_{1}+0\right) $$ 
或者写成如下形式:
$$
\left\{\begin{array}{l}
\lim\limits _{x \rightarrow 1^{+}} S(x)=\lim\limits _{x \rightarrow 1^{-}} S(x) \\
\lim \limits_{x \rightarrow 1^{+}} S^{\prime}(x)=\lim\limits _{x \rightarrow 1^{-}} S^{\prime}(x)\\
\lim \limits_{x \rightarrow 1^{+}} S^{\prime\prime}(x)=\lim\limits _{x \rightarrow 1^{-}} S^{\prime\prime}(x)
\end{array}\right.
$$

而$ \begin{array}{l}S^{\prime}(x)=\left\{\begin{array}{ll}6 x^{2}-3 , &0 \leqslant x<1 \\ 3(x-1)^{2}+2 b(x-1)+c,  &1 \leqslant x \leqslant 2\end{array}\right. \quad S^{\prime \prime}(x)=\left\{\begin{array}{ll}12 x & 0 \leqslant x<1 \\ 6(x-1)+2 b & 1 \leqslant x \leqslant 2\end{array}\right. \end{array} $

由此可得
$$
\left\{\begin{array}{l}
3=3 \\
3=c \\
12=2b
\end{array}\right.
$$

解得$b=6, c=3$
于是有 $ S^{\prime}(0)=-3 $ 和 $ S^{\prime}(2)=18 $.
\end{tcolorbox}


\begin{tcolorbox}[breakable,enhanced,arc=0mm,outer arc=0mm,
		boxrule=0pt,toprule=1pt,leftrule=0pt,bottomrule=1pt, rightrule=0pt,left=0.2cm,right=0.2cm,
		titlerule=0.5em,toptitle=0.1cm,bottomtitle=-0.1cm,top=0.2cm,
		colframe=white!10!biru,colback=white!90!biru,coltitle=white,
            coltext=black,title =2024-03-10, title style={white!10!biru}, before skip=8pt, after skip=8pt,before upper=\hspace{2em},
		fonttitle=\bfseries,fontupper=\normalsize]
  
7. 已知函数表如下:
\begin{tabular}{|c|c|c|c|c|c|}
\hline$ x $ & $-2$ & $-1$ & 0 & 1 & 2 \\
\hline$ f(x) $ & 0 & 1 & 2 & 1 & 0 \\
\hline
\end{tabular}
用二次曲线拟合表中数据.

 \tcblower
\textbf{方法一:}对于给定的一组数据 $ \left(x_{i}, y_{i}\right), i=0,1, \cdots, m $, 求作 $ n $ 次多项式
$$
y=\sum_{k=0}^{n} a_{k} x^{k}
$$
使残差平方和为最小, 于是有法(正则)方程组
$$
\left[\begin{array}{ccccc}
m & \sum x_{i} & \sum x_{i}^{2} & \cdots & \sum x_{i}^{n} \\
\sum x_{i} & \sum x_{i}^{2} & \sum x_{i}^{3} & \cdots & \sum x_{i}^{n+1} \\
\vdots & \vdots & \vdots & & \vdots \\
\sum x_{i}^{n} & \sum x_{i}^{n+1} & \sum x_{i}^{n+2} & \cdots & \sum x_{i}^{2 n}
\end{array}\right]\left[\begin{array}{c}
a_{0} \\
a_{i} \\
\vdots \\
a_{n}
\end{array}\right]=\left[\begin{array}{c}
\sum y_{i} \\
\sum x_{i} y_{i} \\
\vdots \\
\sum x_{i}^{n} y_{i}
\end{array}\right]
$$
其中 $ \sum $ 是 $ \sum\limits_{i=1}^{n} $ 的简写.求出 $ a_{0}, a_{1}, \cdots, a_{n} $, 就得到了拟合多项式的系数.

设二次拟合函数 $ y=a_{0}+a_{1} x+a_{2} x^{2} $.于是根据上面我们有

$$\left[\begin{array}{ccl}5 & \sum\limits_{i=1}^{5} x_{i} & \sum\limits_{i=1}^{5} x_{i}^{2} \\ \sum\limits_{i=1}^{5} x_{i} & \sum\limits_{i=1}^{5} x_{i}^{2} & \sum\limits_{i=1}^{5} x_{i}^{3} \\ \sum\limits_{i=1}^{5} x_{i}^{2} & \sum\limits_{i=1}^{5} x_{i}^{3} & \sum\limits_{i=1}^{5} x_{i}^{4}\end{array}\right]\left[\begin{array}{l}a_{0} \\ a_{1} \\ a_{2}\end{array}\right]=\left[\begin{array}{c}\sum\limits_{i=1}^{5} y_{i} \\ \sum\limits_{i=1}^{5} x_{i} y_{i} \\ \sum\limits_{i=1}^{5} x_{i}^{2} y_{i}\end{array}\right]$$
$$  \text { 即 }\left[\begin{array}{ccc}5 & 0 & 10 \\ 0 & 10 & 0 \\ 10 & 0 & 34\end{array}\right]\left[\begin{array}{l}a_{0} \\ a_{1} \\ a_{2}\end{array}\right]=\left[\begin{array}{l}4 \\ 0 \\ 2\end{array}\right]  $$
计算得$a_{0}=1.6571 \quad a_{1}=0 \quad a_{2}=-0.4286$.
因此拟合多项式为
$$
y=1.6571-0.4286 x^{2}
$$

\textbf{方法二 :}
设二次拟合函数 $ y=a_{0}+a_{1} x+a_{2} x^{2} $.
正则方程组 $ \boldsymbol{A}^{\top} \boldsymbol{A} \boldsymbol{\alpha}=\boldsymbol{A}^{\top} \boldsymbol{Y} $
$$
\left[\begin{array}{rrrrr}
1 & 1 & 1 & 1 & 1 \\
-2 & -1 & 0 & 1 & 2 \\
4 & 1 & 0 & 1 & 4
\end{array}\right]\left[\begin{array}{rrr}
1 & -2 & 4 \\
1 & -1 & 1 \\
1 & 0 & 0 \\
1 & 1 & 1 \\
1 & 2 & 4
\end{array}\right]\left[\begin{array}{l}
a_{0} \\
a_{1} \\
a_{2}
\end{array}\right]=\left[\begin{array}{rrrrr}
1 & 1 & 1 & 1 & 1 \\
-2 & -1 & 0 & 1 & 2 \\
4 & 1 & 0 & 1 & 4
\end{array}\right]\left[\begin{array}{l}
0 \\
1 \\
2 \\
1 \\
0
\end{array}\right]
$$
化简为
$$
\left[\begin{array}{rrr}
5 & 0 & 10 \\
0 & 10 & 0 \\
10 & 0 & 34
\end{array}\right]\left[\begin{array}{l}
a_{0} \\
a_{1} \\
a_{2}
\end{array}\right]=\left[\begin{array}{l}
4 \\
0 \\
2
\end{array}\right]
$$
计算得$a_{0}=1.6571 \quad a_{1}=0 \quad a_{2}=-0.4286$.因此拟合多项式为
$$
y=1.6571-0.4286 x^{2}
$$

\textbf{方法三:}
等价于求解超定方程组
$$
\left[\begin{array}{rrr}
1 & -2 & (-2)^{2} \\
1 & -1 & (-1)^{2} \\
1 & 0 & 0^{2} \\
1 & 1 & 1^{2} \\
1 & 2 & 2^{2}
\end{array}\right]\left[\begin{array}{l}
a_{0} \\
a_{1} \\
a_{2}
\end{array}\right]=\left[\begin{array}{l}
0 \\
1 \\
2 \\
1 \\
0
\end{array}\right]
$$
的最小二乘解, 即正则方程组 $ \boldsymbol{A}^{\mathrm{T}} \boldsymbol{A} \boldsymbol{\alpha}=\boldsymbol{A}^{\mathrm{T}} \boldsymbol{Y} $ 的解.
计算得
$$
\boldsymbol{A}^{\top} \boldsymbol{A}=\left[\begin{array}{rrr}
5 & 0 & 10 \\
0 & 10 & 0 \\
10 & 0 & 34
\end{array}\right], \boldsymbol{A}^{\top} \boldsymbol{Y}=\left[\begin{array}{l}
4 \\
0 \\
2
\end{array}\right]
$$
解得 $ \alpha=\left(\begin{array}{lll}1.6571 & 0 & -0.4286\end{array}\right)^{\mathrm{\top}} $, 拟合多项式
$$
y=1.6571-0.4286 x^{2}
$$
均方误差
$$
\delta=\sum_{i=1}^{5}\left[y\left(x_{i}\right)-y_{i}\right]^{2}=0.13912813
$$
\end{tcolorbox}

\begin{tcolorbox}[breakable,enhanced,arc=0mm,outer arc=0mm,
		boxrule=0pt,toprule=1pt,leftrule=0pt,bottomrule=1pt, rightrule=0pt,left=0.2cm,right=0.2cm,
		titlerule=0.5em,toptitle=0.1cm,bottomtitle=-0.1cm,top=0.2cm,
		colframe=white!10!biru,colback=white!90!biru,coltitle=white,
            coltext=black,title =2024-03-10, title style={white!10!biru}, before skip=8pt, after skip=8pt,before upper=\hspace{2em},
		fonttitle=\bfseries,fontupper=\normalsize]
  
8. 求一个次数不超过 4 次的插值多项式 $ p(x) $, 使它满足:
$$
\begin{array}{l}
p(0)=f(0)=0, p(1)=f(1)=1, p^{\prime}(0)=f^{\prime}(0)=0, \\
p^{\prime}(1)=f^{\prime}(1)=1, p^{\prime \prime}(1)=f^{\prime \prime}(1)=0
\end{array}
$$
并求其余项表达式(设 $ f(x) $ 存在 5 阶导数).
 \tcblower
为了找到一个次数不超过 4 次的插值多项式 $p(x)$,我们可以利用这些插值条件来构建插值多项式.

设 $p(x) = ax^4 + bx^3 + cx^2 + dx + e$,我们需要找到参数 $a, b, c, d, e$ 来满足给定的插值条件.根据插值条件,我们有:
$$
\begin{cases}
p(0) = e = 0 \\
p(1) = a + b + c + d + e = 1 \\
p'(0) = d = 0 \\
p'(1) = 4a + 3b + 2c + d = 1 \\
p''(1) = 12a + 6b + 2c = 0
\end{cases}
$$
解这个方程组,可以得到 $a = 1$,$b = -3$,$c = 3$,$d = 0$,$e = 0$.
因此,插值多项式为 $p(x) = x^4 -3x^3 +3x^2$.

由于$ f(x) $ 在区间$[0,1]$存在 5 阶导数,且注意到$x=0$为二重节点,$x=1$为三重节点.故插值余项为
$$R(x) = f(x) - p(x)=\frac{1}{5 !} f^{(5)}(\xi) x^{2}(x-1)^{3}, \xi \in[0,1]$$


\colorbox{yellow}{方法二}

根据插值条件,我们设插值多项式为 $ p(x)=x^{2}(ax^{2}+bx+c) $. 解得 $ a=1, b=-3, c=3 $.

设$x_0=0,x_1=1$,为求出余项 $ R(x)=f(x)-p(x) $, 根据 $ R\left(x_{i}\right)=0 $ , $R^{\prime}\left(x_{i}\right)=0(i=0,1) $ 和$R^{\prime\prime}\left(x_{1}\right)=0 $, 设
$$
R(x)=K(x)\left(x-x_{0}\right)^2\left(x-x_{1}\right)^{3}
$$
为确定 $ K(x) $, 构造
$$
\varphi(t)=f(t)-p(t)-K(x)\left(x-x_{0}\right)^2\left(x-x_{1}\right)^{3}
$$
显然 $\varphi(x)=0, \varphi\left(x_{i}\right)=0, i=0,1 $, 且 $\varphi^{\prime}\left(x_{1}\right)=0,\varphi^{\prime\prime}\left(x_{1}\right)=0 $, 

反复应用罗尔定理得 $ \varphi^{(5)}(t) $ 在区间 $ [0, 1] $ 上至少有一个零点 $ \xi $, 故有
$$
\varphi^{(5)}(\xi)=f^{(5)}(\xi)-5 ! K(x)=0
$$
于是
$$
K(x)=\frac{1}{5 !} f^{(5)}(\xi)
$$
故余项表达式
$$
R(x)=\frac{1}{5 !} f^{(5)}(\xi)\left(x-x_{0}\right)^2\left(x-x_{1}\right)^{3}=\frac{f^{(5)}(\xi)}{5!}x^{2}(x-1)^{3}
$$


\end{tcolorbox}



\begin{tcolorbox}[breakable,title=定理]
 设 $ f^{(n)}(x) $ 在 $ [a, b] $ 上连续, $ f^{(n+1)}(x) $ 在 $ (a, b) $ 内存在, 节点 $ a \leqslant x_{0}<x_{1}<\cdots< $ $ x_{n} \leqslant b, L_{n}(x) $ 是满足条件 $L_{n}\left(x_{j}\right)=y_{j}, \quad j=0,1, \cdots, n$ 的插值多项式, 则对任何 $ x \in[a, b] $, 插值余项
$$
R_{n}(x)=f(x)-L_{n}(x)=\frac{f^{(n+1)}(\xi)}{(n+1) !} \omega_{n+1}(x),
$$
这里 $ \xi \in(a, b) $ 且依赖于 $ x$, $\omega_{n+1}(x)=\left(x-x_{0}\right)\left(x-x_{1}\right) \cdots\left(x-x_{n}\right)$.

\tcblower
证明: 由给定条件知 $ R_{n}(x) $ 在节点 $ x_{k}(k=0,1, \cdots, n) $ 上为零, 即 $ R_{n}\left(x_{k}\right)=0(k=0 $, $ 1, \cdots, n) $,于是

$$
R_{n}(x)=K(x)\left(x-x_{0}\right)\left(x-x_{1}\right) \cdots\left(x-x_{n}\right)=K(x) \omega_{n+1}(x),
$$
其中 $ K(x) $ 是与 $ x $ 有关的待定函数.
现把 $ x $ 看成 $ [a, b] $ 上的一个固定点, 作函数
$$
\varphi(t)=f(t)-L_{n}(t)-K(x)\left(t-x_{0}\right)\left(t-x_{1}\right) \cdots\left(t-x_{n}\right),
$$
根据 $ f $ 的假设可知 $ \varphi^{(n)}(t) $ 在 $ [a, b] $ 上连续, $ \varphi^{(n+1)}(t) $ 在 $ (a, b) $ 内存在. 根据插值条件及余项定义, 可知 $ \varphi(t) $ 在点 $ x_{0}, x_{1}, \cdots, x_{n} $ 及 $ x $ 处均为零, 故 $ \varphi(t) $ 在 $ [a, b] $ 上有 $ n+2 $ 个零点, 根据罗尔 (Rolle) 定理, $ \varphi^{\prime}(t) $ 在 $ \varphi(t) $ 的两个零点间至少有一个零点, 故 $ \varphi^{\prime}(t) $ 在 $ [a, b] $ 内至少有 $ n+1 $ 个零点. 对 $ \varphi^{\prime}(t) $ 再应用罗尔定理, 可知 $ \varphi^{\prime \prime}(t) $ 在 $ [a, b] $ 内至少有 $ n $ 个零点. 依此类推, $ \varphi^{(n+1)}(t) $在 $ (a, b) $ 内至少有一个零点, 记为 $ \xi \in(a, b) $, 使
$$
\varphi^{(n+1)}(\xi)=f^{(n+1)}(\xi)-(n+1) ! K(x)=0,
$$
于是
$$
K(x)=\frac{f^{(n+1)}(\xi)}{(n+1) !}, \quad \xi \in(a, b), \text { 且依赖于 } x .
$$
将它代入式$R_n(x)$,就得到余项表达式. 证毕.
\end{tcolorbox}



\subsection{补充习题}

\begin{tcolorbox}[enhanced,colback=10,colframe=9,breakable,coltitle=green!25!black,title=2024]
 求经过 $ A(0,1), B(1,2), C(2,3) $ 三个样点的插值多项式.
\tcblower
由题意可知, 三个插值节点及对应的函数值为
$$
\begin{array}{l}
x_{0}=0, \quad x_{1}=1, \quad x_{2}=2 \\
y_{0}=1, \quad y_{1}=2, \quad y_{2}=3 \\
\end{array}
$$
由拉格朗日二次插值公式得
$$
\begin{aligned}
L_{2}(x) & =\frac{\left(x-x_{1}\right)\left(x-x_{2}\right)}{\left(x_{0}-x_{1}\right)\left(x_{0}-x_{2}\right)} y_{0}+\frac{\left(x-x_{0}\right)\left(x-x_{2}\right)}{\left(x_{1}-x_{0}\right)\left(x_{1}-x_{2}\right)} y_{1}+\frac{\left(x-x_{0}\right)\left(x-x_{1}\right)}{\left(x_{2}-x_{0}\right)\left(x_{2}-x_{1}\right)} y_{2} \\
& =\frac{(x-1)(x-2)}{(0-1)(0-2)} \times 1+\frac{(x-0)(x-2)}{(1-0)(1-2)} \times 2+\frac{(x-0)(x-1)}{(2-0)(2-1)} \times 3 \\
& =x+1 .
\end{aligned}
$$
\end{tcolorbox}


\begin{tcolorbox}[enhanced,colback=10,colframe=9,breakable,coltitle=green!25!black,title=2024]

 设 $ f(x)=\ln (1+x), x \in[0,1], p_{n}(x) $ 为 $ f(x) $ 以 $ (n+1) $ 个等距节点 $ x_{i}=\frac{i}{n}, i=0,1 $, $ 2, \cdots, n $ 为插值节点的 $ n $ 次插值多项式, 证明
$$
\lim _{n \rightarrow \infty} \max _{0 \leqslant x \leqslant 1}\left|f(x)-p_{n}(x)\right|=0 .
$$
\tcblower
 $$ f(x)-p_{n}(x)=\frac{f^{(n+1)}(\xi)}{(n+1)!} \prod_{i=0}^{n}\left(x-x_{i}\right) \quad \xi \in(\min \{x, 0\}, \max \{x, 1\}) $$
$$f^{\prime}(x)=\frac{1}{x+1}, \quad f^{\prime \prime}(x)=-\frac{1}{(x+1)^{2}} ,\quad f^{\prime \prime \prime}(x)=\frac{2!}{(x+1)^{3}}, \cdots, f^{(n+1)}(x)=(-1)^{n} \frac{n!}{(x+1)^{n+1}}
$$

当 $ x \in[0,1] $ 时, $ \left|f^{(n+1)}(x)\right| \leqslant n! $

当 $ x \in[0,1] $ 时 $ ,\left|x-x_{i}\right| \leqslant 1, i=0,1,2, \cdots, n $
于是当 $ x \in[0,1] $ 时,有
$$
\left|f(x)-p_{n}(x)\right| \leqslant \frac{n!}{(n+1)!}=\frac{1}{n+1}
$$

所以
$$
\max _{0 \leqslant x \leqslant 1}\left|f(x)-p_{n}(x)\right| \leqslant \frac{1}{n+1} \quad \Rightarrow
\lim _{n \rightarrow \infty} \max _{0 \leqslant x \leqslant 1}\left|f(x)-p_{n}(x)\right| \leqslant \lim _{n \rightarrow \infty} \frac{1}{n+1}=0 .
$$

\end{tcolorbox}


 \begin{tcolorbox}[enhanced,colback=10,colframe=9,breakable,coltitle=green!25!black,title=2024]
 设 $ \left\{x_{i}\right\}_{i=1}^{n} $ 是首项系数为 $ a_{n} $ 的 $ n $ 次多项式 $ f(x) $ 的互异实零点, 证明有如下等式成立
$$
\sum_{j=1}^{n} \frac{x_{j}^{k}}{f^{\prime}\left(x_{j}\right)}=\left\{\begin{array}{ll}
0, & 0 \leqslant k \leqslant n-2, \\
a_{n}^{-1}, & k=n-1
\end{array}\right.
$$
\tcblower
 由题意可知, 设$f(x)=a_{0}+a_{1} x+\cdots+a_{n-1} x^{n-1}+a_{n} x^{n}$.
又由于 $ f(x) $ 有 $ n $ 个不同实根 $ x_{1}, x_{2}, \cdots, x_{n} $, 则函数 $ f(x) $ 可表示为$f(x)=a_{n}\left(x-x_{1}\right)\left(x-x_{2}\right) \cdots\left(x-x_{n}\right)$, 且 $ f^{\prime}\left(x_{j}\right)=a_{n}\left(x_{j}-x_{1}\right)\left(x_{j}-x_{2}\right) \cdots\left(x_{j}-x_{j-1}\right)\left(x_{j}-x_{j+1}\right) \cdots\left(x_{j}-x_{n}\right) $.

若令 $ \omega_{n}(x)=\left(x-x_{1}\right)\left(x-x_{2}\right) \cdots\left(x-x_{n}\right) $, 则
$$
\omega_{n}^{\prime}\left(x_{j}\right)=\left(x_{j}-x_{1}\right)\left(x_{j}-x_{2}\right) \cdots\left(x_{j}-x_{j-1}\right)\left(x_{j}-x_{j+1}\right) \cdots\left(x_{j}-x_{n}\right)
$$
得到 $ a_{n} \omega_{n}^{\prime}\left(x_{j}\right)=f^{\prime}\left(x_{j}\right) $, 因此原表达式
$$
\sum_{j=1}^{n} \frac{x_{j}^{k}}{f^{\prime}\left(x_{j}\right)}=\sum_{j=1}^{n} \frac{x_{j}^{k}}{a_{n} \omega_{n}^{\prime}\left(x_{j}\right)}
$$
令 $ g(x)=x^{k} $, 则由差商与节点函数值关系 $\displaystyle g\left[x_{1}, x_{2}, \cdots, x_{n}\right]=\sum_{j=1}^{n} \frac{x_{j}^{k}}{\omega_{n}^{\prime}\left(x_{j}\right)} $, 因此
$$
\sum_{j=1}^{n} \frac{x_{j}^{k}}{f^{\prime}\left(x_{j}\right)}=\frac{1}{a_{n}} g\left[x_{1}, x_{2}, \cdots, x_{n}\right]
$$
又由差商与导数关系 $ g\left[x_{1}, x_{2}, \cdots, x_{n}\right]=\dfrac{g^{(n-1)}(\xi)}{(n-1)!} $, 因此可得到
$$
\sum_{j=1}^{n} \frac{x_{j}^{k}}{f^{\prime}\left(x_{j}\right)}=\left\{\begin{array}{ll}
0, & 0 \leqslant k \leqslant n-2, \\
a_{n}^{-1}, & k=n-1
\end{array}\right.
$$

\end{tcolorbox}

  \begin{tcolorbox}[enhanced,colback=10,colframe=9,breakable,coltitle=green!25!black,title=2024]
 试证明差商和函数值的关系
$$
f\left[x_{0}, x_{1}, x_{2}, \cdots, x_{n}\right]=\sum_{k=0}^{n} \frac{f\left(x_{k}\right)}{\omega^{\prime}\left(x_{k}\right)}
$$
其中, $ \omega^{\prime}\left(x_{k}\right)=\prod_{i=0, i \neq k}^{n}\left(x_{k}-x_{i}\right) $.
\tcblower
 (方法一) 利用归纳法, 由当 $ n=1 $ 时,
$$
f\left[x_{0}, x_{1}\right]=\frac{f\left(x_{1}\right)-f\left(x_{0}\right)}{x_{1}-x_{0}}=\frac{f\left(x_{0}\right)}{x_{0}-x_{1}}+\frac{f\left(x_{1}\right)}{x_{1}-x_{0}}
$$
假设当阶数为 $ n $ 时成立, 即有
$$
\begin{aligned}
 f\left[x_{0}, x_{1}, x_{2}, \cdots, x_{n}\right] 
= & \sum_{k=0}^{n} \frac{f\left(x_{k}\right)}{\left(x_{k}-x_{0}\right)\left(x_{k}-x_{1}\right) \cdots\left(x_{k}-x_{k-1}\right)\left(x_{k}-x_{k+1}\right) \cdots\left(x_{k}-x_{n}\right)} \\
 f\left[x_{1}, x_{2}, x_{3}, \cdots, x_{n+1}\right] 
= & \sum_{k=1}^{n+1} \frac{f\left(x_{k}\right)}{\left(x_{k}-x_{1}\right)\left(x_{k}-x_{2}\right) \cdots\left(x_{k}-x_{k-1}\right)\left(x_{k}-x_{k+1}\right) \cdots\left(x_{k}-x_{n+1}\right)}
\end{aligned}
$$
因此当阶数为 $ n+1 $ 时, 由定义
$$
\begin{aligned}
f&\left[x_{0}, x_{1}, x_{2}, \cdots, x_{n+1}\right]\\
= & \frac{f\left[x_{1}, x_{2}, x_{3}, \cdots, x_{n+1}\right]-f\left[x_{0}, x_{1}, x_{2}, \cdots, x_{n}\right]}{x_{n+1}-x_{0}} \\
= & \frac{1}{x_{n+1}-x_{0}}\left[\frac{f\left(x_{n+1}\right)}{\left(x_{n+1}-x_{1}\right)\left(x_{n+1}-x_{1}\right) \cdots\left(x_{n+1}-x_{n}\right)}\right.  \left.+\frac{-f\left(x_{0}\right)}{\left(x_{0}-x_{1}\right)\left(x_{0}-x_{2}\right) \cdots\left(x_{0}-x_{n}\right)}\right] \\
& +\sum_{k=1}^{n} \frac{f\left(x_{k}\right)}{x_{n+1}-x_{0}} \cdot \frac{\left(x_{k}-x_{0}\right)-\left(x_{k}-x_{n+1}\right)}{\left(x_{k}-x_{0}\right)\left(x_{k}-x_{1}\right) \cdots\left(x_{k}-x_{k-1}\right)\left(x_{k}-x_{k+1}\right) \cdots\left(x_{k}-x_{n+1}\right)} \\
= & \sum_{k=0}^{n+1} \frac{f\left(x_{k}\right)}{\left(x_{k}-x_{0}\right)\left(x_{k}-x_{1}\right) \cdots\left(x_{k}-x_{k-1}\right)\left(x_{k}-x_{k+1}\right) \cdots\left(x_{k}-x_{n+1}\right)}
\end{aligned}
$$
因此由归纳原理, 证得该命题成立.

(方法二) 由题意可知, 函数 $ f(x) $ 关于节点 $ \left\{x_{k}\right\}_{k=0}^{n} $ 的 Lagrange 插值多项式为
$$
\begin{aligned}
L_{n}(x) & =\sum_{k=0}^{n} f\left(x_{k}\right) l_{k}(x) \\
& =\sum_{k=0}^{n} \frac{\left(x-x_{0}\right) \cdots\left(x-x_{k-1}\right)\left(x-x_{k+1}\right) \cdots\left(x-x_{n}\right)}{\left(x_{k}-x_{0}\right) \cdots\left(x_{k}-x_{k-1}\right)\left(x_{k}-x_{k+1}\right) \cdots\left(x_{k}-x_{n}\right)} f\left(x_{k}\right)
\end{aligned}
$$
Newton 插值多项式为
$$
N_{n}(x)=  f\left(x_{0}\right)+f\left[x_{0}, x_{1}\right] \omega_{1}(x)+\cdots+f\left[x_{0}, x_{1}, \cdots, x_{n-1}\right] \omega_{n-1}(x)  +f\left[x_{0}, x_{1}, \cdots, x_{n}\right] \omega_{n}(x)
$$
其中, $ \omega_{k}(x)=\left(x-x_{0}\right)\left(x-x_{1}\right) \cdots\left(x-x_{k-1}\right)(1 \leqslant k \leqslant n) $ 是首项系数为 1 的 $ k $次多项式.

由插值多项式的唯一性可知 $ N_{k}(x) \equiv L_{k}(x) $, 比较其 $ x^{n} $ 的系数便得到所需证明的结论.
\end{tcolorbox}

  \begin{tcolorbox}[enhanced,colback=10,colframe=9,breakable,coltitle=green!25!black,title=2024]
  若函数 $ F(x)=f(x)+g(x) $, 则证明 $ n $ 阶差商性质
$$
F\left[x_{0}, x_{1}, \cdots, x_{n}\right]=f\left[x_{0}, x_{1}, \cdots, x_{n}\right]+g\left[x_{0}, x_{1}, \cdots, x_{n}\right]
$$
\tcblower
 由差商与函数值的关系可知
$$
F\left[x_{0}, x_{1}, \cdots, x_{n}\right]=\sum_{k=0}^{n} \frac{F\left(x_{k}\right)}{\omega^{\prime}\left(x_{k}\right)}
$$
其中, $\displaystyle \omega^{\prime}\left(x_{k}\right)=\prod_{i=0, i \neq k}^{n}\left(x_{k}-x_{i}\right) $. 由 $ F(x)=f(x)+g(x) $, 得 $ F\left(x_{k}\right)=f\left(x_{k}\right)+g\left(x_{k}\right) $, $ k=0,1, \cdots, n $, 因此
$$
\begin{aligned}
F\left[x_{0}, x_{1}, \cdots, x_{n}\right] & =\sum_{k=0}^{n} \frac{f\left(x_{k}\right)+g\left(x_{k}\right)}{\omega^{\prime}\left(x_{k}\right)}=\sum_{k=0}^{n} \frac{f\left(x_{k}\right)}{\omega^{\prime}\left(x_{k}\right)}+\sum_{k=0}^{n} \frac{g\left(x_{k}\right)}{\omega^{\prime}\left(x_{k}\right)} \\
& =f\left[x_{0}, x_{1}, \cdots, x_{n}\right]+g\left[x_{0}, x_{1}, \cdots, x_{n}\right] .
\end{aligned}
$$
\end{tcolorbox}

  \begin{tcolorbox}[enhanced,colback=10,colframe=9,breakable,coltitle=green!25!black,title=2024]
 设 $ f(x) $ 在区间 $ [a, b] $ 上具有二阶导数连续, 求证
$$
\max _{a \leqslant x \leqslant b}\left|f(x)-\left[f(a)+\frac{f(b)-f(a)}{b-a}(x-a)\right]\right| \leqslant \frac{1}{8}(b-a)^{2} M_{2}
$$
其中, $ M_{2}=\max\limits _{a \leqslant x \leqslant b}\left|f^{\prime \prime}(x)\right| $.
\tcblower
 由题意可知, 通过两点 $ (a, f(a)) $ 及 $ (b, f(b)) $ 的线性插值为
$$
L_{1}(x)=f(a)+\frac{f(b)-f(a)}{b-a}(x-a)
$$
于是
$$
\begin{aligned}
& \max _{a \leqslant x \leqslant b}\left|f(x)-\left[f(a)+\frac{f(b)-f(a)}{b-a}(x-a)\right]\right| \\
= & \max _{a \leqslant x \leqslant b}\left|f(x)-L_{1}(x)\right|=\max _{a \leqslant x \leqslant b}\left|\frac{f^{\prime \prime}(\xi)}{2!}(x-a)(x-b)\right| \\
\leqslant & \frac{M_{2}}{2} \max _{a \leqslant x \leqslant b}|(x-a)(x-b)| \leqslant \frac{1}{8}(b-a)^{2} M_{2}
\end{aligned}
$$
其中, $ M_{2}=\max\limits _{a \leqslant x \leqslant b}\left|f^{\prime \prime}(x)\right| $.
  \end{tcolorbox}


    \begin{tcolorbox}[enhanced,colback=10,colframe=9,breakable,coltitle=green!25!black,title=2024]
 已知 $ \omega(x)=\prod\limits_{i=0}^{n}\left(x-x_{i}\right) $, 证明 $ \omega^{\prime}\left(x_{k}\right)=\prod\limits_{i=0, i \neq k}^{n}\left(x_{k}-x_{i}\right) $.
 \tcblower
 由题意可知, $ \omega(x)=\prod\limits_{i=0}^{n}\left(x-x_{i}\right)=(x-x)\left(x-x_{1}\right) \cdots\left(x-x_{n}\right) $, 因此
$$
\begin{aligned}
\omega^{\prime}\left(x_{k}\right) & =\lim _{x \rightarrow x_{k}} \frac{\omega(x)-\omega\left(x_{k}\right)}{x-x_{k}}=\lim _{x \rightarrow x_{k}} \frac{\omega(x)}{x-x_{k}} \\
& =\left(x_{k}-x_{0}\right)\left(x_{k}-x_{1}\right) \cdots\left(x_{k}-x_{k-1}\right)\left(x_{k}-x_{k+1}\right) \cdots\left(x_{k}-x_{n}\right) \\
& =\prod_{i=0, i \neq k}^{n}\left(x_{k}-x_{i}\right)
\end{aligned}
$$
  \end{tcolorbox}

 \begin{tcolorbox}[enhanced,colback=10,colframe=9,breakable,coltitle=green!25!black,title=2024]
证明表达式 $ \sum\limits_{k=0}^{n}\left(x_{k}-x\right)^{j} l_{k}(x)=0, j=1,2, \cdots, n $.
\tcblower
 由题意可知, 设 $ f(x)=(x-t)^{j}, j=1,2, \cdots, n $, 因此$f\left(x_{k}\right)=\left(x_{k}-t\right)^{j}$.由于 $ j \leqslant n $, 所以其余项为零, 且
$$
f(x)=L(x)=\sum_{k=0}^{n} l_{k}(x) f\left(x_{k}\right)
$$
即 $ (x-t)^{j}=\sum\limits_{k=0}^{n} l_{k}(x)\left(x_{k}-t\right)^{j} $, 若令 $ t=x $, 有 $ \sum\limits_{k=0}^{n}\left(x_{k}-x\right)^{j} l_{k}(x)=0 $.
\end{tcolorbox}
  \begin{tcolorbox}[enhanced,colback=10,colframe=9,breakable,coltitle=green!25!black,title=2024]

若 $ g(x) $ 是 $ f(x) $ 以 $ x_{0}, x_{1}, \cdots, x_{n-1} $ 为插值节点的 $ n-1 $ 次插值多项式, $ h(x) $是 $ f(x) $ 以 $ x_{1}, x_{2}, \cdots, x_{n} $ 为插值节点的 $ n-1 $ 次插值多项式, 证明: 函数
$$
g(x)+\frac{x-x_{0}}{x_{n}-x_{0}}[h(x)-g(x)]
$$
是 $ f(x) $ 以 $ x_{0}, x_{1}, \cdots, x_{n} $ 为插值节点的 $ n $ 次插值多项式.
 \tcblower
 由条件知
$$
\begin{array}{l}
g\left(x_{i}\right)=f\left(x_{i}\right), \quad i=0,1, \cdots, n-1, \\
h\left(x_{i}\right)=f\left(x_{i}\right), \quad i=1,2, \cdots, n .
\end{array}
$$
记
$$
L(x)=g(x)+\frac{x-x_{0}}{x_{n}-x_{0}}[h(x)-g(x)],
$$
因为 $ g(x) $ 和 $ h(x) $ 为 $ n-1 $ 次多项式, 所以 $L(x) $ 为 $ n $ 次多项式, 而且
$$
L\left(x_{0}\right)=g\left(x_{0}\right)=f\left(x_{0}\right),
$$
当 $ j=1,2, \cdots, n-1 $ 时, 有
$$
\begin{aligned}
L\left(x_{j}\right)&=g\left(x_{j}\right)+\frac{x_{j}-x_{0}}{x_{n}-x_{0}}\left[h\left(x_{j}\right)-g\left(x_{j}\right)\right] \\
&=f\left(x_{j}\right)+\frac{x_{j}-x_{0}}{x_{n}-x_{0}}\left[f\left(x_{j}\right)-f\left(x_{j}\right)\right]=f\left(x_{j}\right), \\
L\left(x_{n}\right)&=g\left(x_{n}\right)+\frac{x_{n}-x_{0}}{x_{n}-x_{0}}\left[h\left(x_{n}\right)-g\left(x_{n}\right)\right]=h\left(x_{n}\right)=f\left(x_{n}\right),
\end{aligned}
$$
因此有
$$
L\left(x_{i}\right)=f\left(x_{i}\right), \quad i=0,1, \cdots, n,
$$
即 $L(x) $ 为 $ f(x) $ 以 $ x_{0}, x_{1}, \cdots, x_{n} $ 为插值节点的 $ n $ 次插值多项式.
\end{tcolorbox}

  \begin{tcolorbox}[enhanced,colback=10,colframe=9,breakable,coltitle=green!25!black,title=2024]
 设 $ l_{i}(x)(i=0,1, \cdots, n) $ 是关于互异节点 $ x_{i}(i=0,1, \cdots, n) $ 的 Lagrange 插值基函数, 求证
$$
\sum_{i=0}^{n} l_{i}(0) x_{i}^{k}=\left\{\begin{array}{ll}
1, & k=0, \\
0, & k=1,2, \cdots, n \\
(-1)^{n} x_{0} x_{1} \cdots x_{n}, & k=n+1
\end{array}\right.
$$
\tcblower
 若函数 $ f(x) $ 在 $ [a, b] $ 上具有 $ n+1 $ 阶导数, 则有
$$
f(x)=\sum_{i=0}^{n} l_{i}(x) f\left(x_{i}\right)+\frac{f^{(n+1)}(\xi)}{(n+1)!} \omega(x)
$$
其中, $ \omega(x)=\left(x-x_{0}\right)\left(x-x_{1}\right) \cdots\left(x-x_{n}\right) $.

(1) 当 $ f(x)=1 $ 时, 余项为零, 因此 $ \sum\limits_{i=0}^{n} l_{i}(x) f\left(x_{i}\right)=1 $, 即
$$
\sum_{i=0}^{n} l_{i}(0)=1
$$

(2) 当 $ f(x)=x^{k}(k=1,2, \cdots, n) $ 时, 余项仍为零, 因此 $ \sum\limits_{i=0}^{n} l_{i}(x) x_{i}^{k}=x^{k} $, 将 $ x=0 $ 代入, 即可得到
$$
\sum_{i=0}^{n} l_{i}(x) x_{i}^{k}=0
$$

(3) 当 $ f(x)=x^{n+1} $ 时, 此时余项为 $ \omega(x) $, 因此
$$
x^{n+1}=\sum_{i=0}^{n} l_{i}(x) x_{i}^{n+1}+\omega(x)
$$
将 $ x=0 $ 代入
$$
\sum_{i=0}^{n} l_{i}(0) x_{i}^{n+1}=(-1)^{n} x_{0} x_{1} \cdots x_{n}
$$
综上证得命题成立.
\end{tcolorbox}

\begin{tcolorbox}[enhanced,colback=10,colframe=9,breakable,coltitle=green!25!black,title=2024]
证明 $ n $ 阶差商性质: 若 $ F(x)=c f(x) $, 则 $ F\left[x_{0}, x_{1}, \cdots, x_{n}\right]=c f\left[x_{0}\right. $, $ \left.x_{1}, \cdots, x_{n}\right] $.
\tcblower
 由差商与函数值关系可得
$$
F\left[x_{0}, x_{1}, \cdots, x_{n}\right]=\sum_{k=0}^{n} \frac{F\left(x_{k}\right)}{\omega^{\prime}\left(x_{k}\right)}
$$
其中, $ \omega^{\prime}\left(x_{k}\right)=\prod_{i=0, i \neq k}^{n}\left(x_{k}-x_{i}\right) $.
又由于 $ F(x)=c f(x) $, 因此 $ F\left(x_{k}\right)=c f\left(x_{k}\right) $, 且
$$
F\left[x_{0}, x_{1}, \cdots, x_{n}\right]=\sum_{k=0}^{n} \frac{c f\left(x_{k}\right)}{\omega^{\prime}\left(x_{k}\right)}=c \sum_{k=0}^{n} \frac{f\left(x_{k}\right)}{\omega^{\prime}\left(x_{k}\right)}=c f\left[x_{0}, x_{1}, \cdots, x_{n}\right]
$$
\end{tcolorbox}



\begin{tcolorbox}[enhanced,colback=10,colframe=9,breakable,coltitle=green!25!black,title=2024]
证明两点三次 Hermite 插值余项是
$
R_{3}(x)=\frac{f^{(4)}(\xi)}{4!}\left(x-x_{k}\right)^{2}\left(x-x_{k+1}\right)^{2}, \quad \xi \in\left(x_{k}, x_{k+1}\right)
$
\tcblower
 由题意可知, 若 $ x \in\left[x_{k}, x_{k+1}\right] $, 且插值多项式满足条件
$$
H_{3}\left(x_{k}\right)=f\left(x_{k}\right), \quad H_{3}^{\prime}\left(x_{k}\right)=f^{\prime}\left(x_{k}\right), \quad H_{3}\left(x_{k+1}\right)=f\left(x_{k+1}\right), \quad H_{3}^{\prime}\left(x_{k+1}\right)=f^{\prime}\left(x_{k+1}\right)
$$
若设插值余项为
$$
R(x)=f(x)-H_{3}(x)
$$
由插值条件可知, 显然
$$
R\left(x_{k}\right)=R\left(x_{k+1}\right)=0, \quad R^{\prime}\left(x_{k}\right)=R^{\prime}\left(x_{k+1}\right)=0
$$
因此余项 $ R(x) $ 可写成
$$
R(x)=g(x)\left(x-x_{k}\right)^{2}\left(x-x_{k+1}\right)^{2}
$$
其中, $ g(x) $ 是关于 $ x $ 的待定函数, 若把 $ x $ 看成 $ \left[x_{k}, x_{k+1}\right] $ 上的一个固定点, 作函数
$$
\varphi(t)=f(t)-H_{3}(t)-g(x)\left(t-x_{k}\right)^{2}\left(t-x_{k+1}\right)^{2}
$$
根据余项表达式和插值条件, 有
$$
\varphi\left(x_{k}\right)=0, \quad \varphi\left(x_{k+1}\right)=0, \quad \varphi^{\prime}\left(x_{k}\right)=0, \quad \varphi^{\prime}\left(x_{k+1}\right)=0
$$
又由于
$$
\varphi(x)=f(x)-H_{3}(x)-g(x)\left(x-x_{k}\right)^{2}\left(x-x_{k+1}\right)^{2}=f(x)-H_{3}(x)-R(x)=0
$$
因此 $ \varphi(t) $ 至少存在 5 个零点, 由 Rolle 中值定理可知, 存在 $ \xi_{1} \in\left(x_{k}, x\right) $ 和 $ \xi_{2} \in $ $ \left(x, x_{k+1}\right) $, 使 $ \varphi^{\prime}\left(\xi_{1}\right)=0, \varphi^{\prime}\left(\xi_{2}\right)=0 $, 即 $ \varphi^{\prime}(x) $ 在 $ \left[x_{k}, x_{k+1}\right] $ 上有四个互异零点,又由 Rolle 中值定理, $ \varphi^{\prime \prime}(t) $ 在 $ \varphi^{\prime}(t) $ 的两个零点间至少有一个零点, 则 $ \varphi^{\prime \prime}(t) $ 在 $ \left(x_{k}, x_{k+1}\right) $ 内至少有三个互异零点, 以此类推, $ \varphi^{(4)}(t) $ 在 $ \left(x_{k}, x_{k+1}\right) $ 内至少有一个零点, 记为 $ \xi \in\left(x_{k}, x_{k+1}\right) $, 使得
$$
\varphi^{(4)}(\xi)=f^{(4)}(\xi)-H_{3}^{(4)}(\xi)-4!g(x)=0
$$
由于 $ H_{3}^{(4)}(t)=0 $, 因此 $ g(x)=\frac{f^{(4)}(\xi)}{4!}, \xi \in\left(x_{k}, x_{k+1}\right) $, 其中, $ \xi $ 依赖于 $ x $, 即证得余项
$$
R_{3}(x)=\frac{f^{(4)}(\xi)}{4!}\left(x-x_{k}\right)^{2}\left(x-x_{k+1}\right)^{2}
$$

\end{tcolorbox}


  \begin{tcolorbox}[enhanced,colback=10,colframe=9,breakable,coltitle=green!25!black,title=2024]
  

设函数 $ f(x) $ 在区间 $ \left[x_{0}, x_{2}\right] $ 上有四阶连续导数. 求满足下列条件的次数不超过 3 次的插值多项式 $ H(x) $ :
$$
H\left(x_{i}\right)=f\left(x_{i}\right), \quad i=0,1,2, \quad H^{\prime}\left(x_{1}\right)=f^{\prime}\left(x_{1}\right),
$$
其中 $ x_{0}<x_{1}<x_{2} $. 并且, 求余项 $ f(x)-H(x) $ 的表达式.
 \tcblower

 由给定条件, 多项式 $ H(x) $ 经过点 $ \left(x_{0}, f\left(x_{0}\right)\right),\left(x_{1}, f\left(x_{1}\right)\right),\left(x_{2}, f\left(x_{2}\right)\right) $, 即有 $ H\left(x_{i}\right)=f\left(x_{i}\right), i=0,1,2 $, 因此可令
$$
\begin{aligned}
H(x)= & N_{2}(x)+A\left(x-x_{0}\right)\left(x-x_{1}\right)\left(x-x_{2}\right) \\
= & f\left(x_{0}\right)+f\left[x_{0}, x_{1}\right]\left(x-x_{0}\right)+f\left[x_{0}, x_{1}, x_{2}\right]\left(x-x_{0}\right)\left(x-x_{1}\right)  +A\left(x-x_{0}\right)\left(x-x_{1}\right)\left(x-x_{2}\right),
\end{aligned}
$$
其中 $ A $ 为待定常数. 为了确定 $ A $, 可利用条件 $ H^{\prime}\left(x_{1}\right)=f^{\prime}\left(x_{1}\right) $. 由于
$$
\begin{aligned}
H^{\prime}(x)= & f\left[x_{0}, x_{1}\right]+f\left[x_{0}, x_{1}, x_{2}\right]\left(2 x-x_{0}-x_{1}\right) \\
& +A\left(x-x_{1}\right)\left(x-x_{2}\right)+A\left(x-x_{0}\right)\left(x-x_{2}\right)+A\left(x-x_{0}\right)\left(x-x_{1}\right),
\end{aligned}
$$
因此, 由 $ f^{\prime}\left(x_{1}\right)=H^{\prime}\left(x_{1}\right) $ 可得
$$
A=\frac{f^{\prime}\left(x_{1}\right)-f\left[x_{1}, x_{2}\right]-\left(x_{1}-x_{0}\right) f\left[x_{0}, x_{1}, x_{2}\right]}{\left(x_{1}-x_{0}\right)\left(x_{1}-x_{2}\right)} .
$$
将它代入 便得到所要求的插值多项式 $ H(x) $.

现讨论余项: 由 $ R\left(x_{0}\right)=0, R\left(x_{1}\right)=0, R\left(x_{2}\right)=0, R^{\prime}\left(x_{1}\right)=0 $, 构造
$$
R(x)=f(x)-H(x)=K(x) \cdot\left(x-x_{0}\right)\left(x-x_{1}\right)^{2}\left(x-x_{2}\right),
$$
其中 $ K(x) $ 为待定函数. 作辅助函数
$$
\varphi(t)=f(t)-H(t)-K(x)\left(t-x_{0}\right)\left(t-x_{1}\right)^{2}\left(t-x_{2}\right),
$$

显然
$$
\varphi\left(x_{j}\right)=0(j=0,1,2) \text { 且 } \varphi^{\prime}\left(x_{1}\right)=0, \varphi(x)=0 .
$$

据罗尔定理, 当 $ x $ 不在节点上时, $ \varphi^{\prime}(t) $ 在 $ \left(x_{0}, x_{2}\right) $ 上至少有 4 个互异的零点, 反复应用罗尔定理可推得 $ \varphi^{(4)}(t) $ 在 $ \left(x_{0}, x_{2}\right) $ 上至少有 1 个零点, 记作 $ \xi$ , 其满足:
$$
\varphi^{(4)}(\xi)=f^{(4)}(\xi)-4 ! K(x)=0 .
$$
推出 $ K(x)=\dfrac{f^{(4)}(\xi)}{4 !} $. 代入得
$$
R(x)=\frac{f^{(4)}(\xi)}{4 !}\left(x-x_{0}\right)\left(x-x_{1}\right)^{2}\left(x-x_{2}\right),
$$
其中 $ \xi \in\left(x_{0}, x_{2}\right) $.

\end{tcolorbox}

  \begin{tcolorbox}[enhanced,colback=10,colframe=9,breakable,coltitle=green!25!black,title=2024]
  

设 $ f(x) $ 在 $ \left[x_{0}, x_{3}\right] $ 上有五阶连续导数, 且 $ x_{0}<x_{1}<x_{2}<x_{3} $

(1) 试作一个次数不高于四次的多项式 $ H_{4}(x) $, 满足条件
$$
\left\{\begin{array}{l}
H_{4}\left(x_{j}\right)=f\left(x_{j}\right), \quad j=0,1,2,3 \\
H_{4}^{\prime}\left(x_{1}\right)=f^{\prime}\left(x_{1}\right)
\end{array}\right.
$$
(2) 余项 $ E(x)=f(x)-H_{4}(x) $ 的表达式
 \tcblower

Solution:

(1)可由lagrange 型插值多项式的构造法去寻求 $ H_{4}(x) $ 的构
造方法

设 $ H_{4}(x)=N_{3}(x)+\mathrm{A}\left(x-x_{0}\right)\left(x-x_{1}\right)\left(x-x_{2}\right)\left(x-x_{3}\right) $
代入插值条件 $ H_{4}^{\prime}\left(x_{1}\right)=f^{\prime}\left(x_{1}\right) $ 即可确定 $ A $
(2)构造辅助函数应用罗尔定理证明余项误差表达式
$$
E(x)=f(x)-H_{4}(x)=\frac{f^{(5)}(\xi)}{5 !}\left(x-x_{0}\right)\left(x-x_{1}\right)^{2}\left(x-x_{2}\right)\left(x-x_{3}\right)
$$

\end{tcolorbox}


  \begin{tcolorbox}[enhanced,colback=10,colframe=9,breakable,coltitle=green!25!black,title=2024]
  
设 $ x_{0} \neq x_{2} $, 求作次数 $ \leqslant 2 $ 的多项式 $ p(x) $, 使满足插值条件
$$
p\left(x_{0}\right)=y_{0}, p^{\prime}\left(x_{1}\right)=y_{1}^{\prime}, p\left(x_{2}\right)=y_{2}
$$
 \tcblower
满足条件$f\left(x_{0}\right)=y_{0}, f\left(x_{2}\right)=y_{2}$的插值多项式可表示为:
$$
f(x)=y_{0}+\frac{y_{2}-y_{0}}{x_{2}-x_{0}}\left(x-x_{0}\right)
$$
令所求的插值多项式
$$
p(x)  =f(x)+A\left(x-x_{0}\right)\left(x-x_{2}\right)  =y_{0}+\frac{y_{2}-y_{0}}{x_{2}-x_{0}}\left(x-x_{0}\right)+A\left(x-x_{0}\right)\left(x-x_{2}\right)
$$
由于
$$
p^{\prime}(x)=\frac{y_{2}-y_{0}}{x_{2}-x_{0}}+A\left(2 x-x_{0}-x_{2}\right) .
$$
按插值条件 $ p^{\prime}\left(x_{1}\right)=y_{1}^{\prime} $ 有
$$
\frac{y_{2}-y_{0}}{x_{2}-x_{0}}+A\left(2 x_{1}-x_{0}-x_{2}\right)=y_{1}^{\prime}
$$
由此解出
$$
A=\frac{y_{1}^{\prime}-\frac{y_{2}-y_{0}}{x_{2}-x_{0}}}{2 x_{1}-x_{0}-x_{2}}
$$
然后将$A$ 代入$p(x)$表达式即可. 这里要求 $ 2 x_{1}-x_{0}-x_{2} \neq 0 $, 即
$$
x_{1} \neq \frac{x_{0}+x_{2}}{2}
$$

\end{tcolorbox}



  \begin{tcolorbox}[enhanced,colback=10,colframe=9,breakable,coltitle=green!25!black,title=2024]
  
 求二次多项式 $ P_{2}(x) $,使它满足
$$
P_{2}\left(x_{0}\right)=y_{1}, \quad P_{2}\left(x_{2}\right)=y_{2}, \quad P_{2}^{\prime}\left(x_{1}\right)=y_{1}^{\prime}
$$
其中
$$
x_{0}<x_{1}<x_{2}, \quad x_{1} \neq \frac{x_{1}+x_{2}}{2}
$$

 \tcblower


设
$$ L_{1}(x)=y_{1} \frac{x-x_{2}}{x_{0}-x_{2}}+y_{2} \frac{x-x_{0}}{x_{2}-x_{0}}=\frac{y_{1}\left(x-x_{2}\right)-y_{2}\left(x-x_{0}\right)}{x_{0}-x_{2}}=y_{1}+\frac{y_{2}-y_{1}}{x_{2}-x_{0}}\left(x-x_{0}\right) $$
并令 $ P_{2}(x)=L_{1}(x)+r(x) $, 则易知 $ q(x)=P_{2}(x)-L_{1}(x) $ 为不超过 2 次的多项式, 且.
$$
\begin{array}{l}
r\left(x_{0}\right)=P_{2}\left(x_{0}\right)-L_{1}\left(x_{0}\right)=y_{1}-y_{1}=0 \\
r\left(x_{2}\right)=P_{2}\left(x_{2}\right)-L_{1}\left(x_{2}\right)=y_{2}-y_{2}=0
\end{array}
$$
即 $ x_{0}, x_{2} $ 为 $ r(x) $ 的零点.于是 $ r(x)=A\left(x-x_{0}\right)\left(x-x_{2}\right) $, 其中 $ A $ 为待定常数.于是
$$
P_{2}(x)=L_{1}(x)+r(x)=\frac{y_{1}\left(x-x_{2}\right)-y_{2}\left(x-x_{0}\right)}{x_{0}-x_{2}}+A\left(x-x_{0}\right)\left(x-x_{2}\right)
$$
对 $ P_{2}(x) $ 求导得
$$
P_{2}^{\prime}(x)=\frac{\left(y_{1}-y_{2}\right)}{x_{0}-x_{2}}+A\left(2 x-x_{0}-x_{2}\right)
$$
由 $ P_{2}^{\prime}\left(x_{1}\right)=y_{1}^{\prime} $ 得
$$
\frac{\left(y_{1}-y_{2}\right)}{x_{0}-x_{2}}+A\left(2 x_{1}-x_{0}-x_{2}\right)=y_{1}^{\prime}
$$
所以
$$
A=\frac{y_{1}^{\prime}\left(x_{0}-x_{2}\right)-\left(y_{1}-y_{2}\right)}{\left(2 x_{1}-x_{0}-x_{2}\right)\left(x_{0}-x_{2}\right)}
$$
因而 
$$ P_{2}(x)=y_{1}+\frac{y_{2}-y_{1}}{x_{2}-x_{0}}\left(x-x_{0}\right)+\frac{y_{1}^{\prime}\left(x_{0}-x_{2}\right)-\left(y_{1}-y_{2}\right)}{\left(2 x_{1}-x_{0}-x_{2}\right)\left(x_{0}-x_{2}\right)}\left(x-x_{0}\right)\left(x-x_{2}\right) $$

\end{tcolorbox}


  \begin{tcolorbox}[enhanced,colback=10,colframe=9,breakable,coltitle=green!25!black,title=2024]
  
在区间 $ [a, b] $ 上任取插值节点
$$
a \leqslant x_{0}<x_{1}<\cdots<x_{n} \leqslant b,
$$
$$
L(x)=\frac{\left(x-x_{1}\right)\left(x-x_{2}\right) \cdots\left(x-x_{n}\right)}{\left(x_{0}-x_{1}\right)\left(x_{0}-x_{2}\right) \cdots\left(x_{0}-x_{n}\right)},
$$
求证:
$$
L(x)= 1+\frac{x-x_{0}}{x_{0}-x_{1}}+\frac{\left(x-x_{0}\right)\left(x-x_{1}\right)}{\left(x_{0}-x_{1}\right)\left(x_{0}-x_{2}\right)}+\cdots  +\frac{\left(x-x_{0}\right)\left(x-x_{1}\right) \cdots\left(x-x_{n-1}\right)}{\left(x_{0}-x_{1}\right)\left(x_{0}-x_{2}\right) \cdots\left(x_{0}-x_{n}\right)} .
$$

 \tcblower

解: $ L(x) $ 为 $ n $ 次多项式, 且
$$
L\left(x_{0}\right)=1, L\left(x_{1}\right)=0, L\left(x_{2}\right)=0, \cdots, L\left(x_{n}\right)=0,
$$
于是$L\left[x_{0}\right]=L\left(x_{0}\right)=1,$
$$
L\left[x_{0}, x_{1}, \cdots, x_{k}\right]=\sum_{i=0}^{k} \frac{L\left(x_{i}\right)}{\prod\limits_{\substack{j=0 \\ j \neq i}}^{k}\left(x_{i}-x_{j}\right)}=\frac{1}{\prod\limits_{j=1}^{k}\left(x_{0}-x_{j}\right)}, \quad 1 \leqslant k \leqslant n .
$$
记 $ N(x) $ 为 $ L(x) $ 的 $ n $ 次 Newton 插值多项式, 则
$$
\begin{aligned}
N(x)= & L\left[x_{0}\right]+L\left[x_{0}, x_{1}\right]\left(x-x_{0}\right)+L\left[x_{0}, x_{1}, x_{2}\right]\left(x-x_{0}\right)\left(x-x_{1}\right) \\
& +\cdots+L\left[x_{0}, x_{1}, \cdots, x_{n}\right]\left(x-x_{0}\right)\left(x-x_{1}\right) \cdots\left(x-x_{n-1}\right),
\end{aligned}
$$
由于 $ L(x) $ 为 $ n $ 次多项式, 因而
$$
\begin{aligned}
L(x)=  N(x) 
=  1+\frac{x-x_{0}}{x_{0}-x_{1}}+\frac{\left(x-x_{0}\right)\left(x-x_{1}\right)}{\left(x_{0}-x_{1}\right)\left(x_{0}-x_{2}\right)}  +\cdots+\frac{\left(x-x_{0}\right)\left(x-x_{1}\right) \cdots\left(x-x_{n-1}\right)}{\left(x_{0}-x_{1}\right)\left(x_{0}-x_{2}\right) \cdots\left(x_{0}-x_{n}\right)}
\end{aligned}
$$

\end{tcolorbox}


  \begin{tcolorbox}[enhanced,colback=10,colframe=9,breakable,coltitle=green!25!black,title=2024]
  
设 $s(x)$是 $[0,2]$ 上满足自然边界条件的三次样条函数, 试确定
$$
\begin{array}{l}
s(x)=\left\{\begin{array}{ll}
1+2 x-x^{3} & 0 \leqslant x \leqslant 1 \\
2+b(x-1)+c(x-1)^{2}+d(x-1)^{3} & 1 \leqslant x \leqslant 2
\end{array}\right. \\
\end{array}
$$
 中的参数  $b, c$ 和 $ d$ .


 \tcblower
取 $ x_{0}=0, x_{1}=1, x_{2}=2 $, 根据三次样条函数的定义, 有 $ S(x) \in C^{2}[0,2] $, 由 $ S(x) $ 及其导数的连续性, 即有: 
$$ S\left(x_{1}-0\right)=S\left(x_{1}+0\right), S^{\prime}\left(x_{1}-0\right)=S^{\prime}\left(x_{1}+0\right), S^{\prime \prime}\left(x_{1}-0\right)=S^{\prime \prime}\left(x_{1}+0\right) $$ 

而$ \begin{array}{l}S^{\prime}(x)=\left\{\begin{array}{ll}-3x^{2}+2 , &0 \leqslant x<1 \\ 3d(x-1)^{2}+2 c(x-1)+b,  &1 \leqslant x \leqslant 2\end{array}\right. \quad S^{\prime \prime}(x)=\left\{\begin{array}{ll}-6 x & 0 \leqslant x<1 \\ 6d(x-1)+2 c & 1 \leqslant x \leqslant 2\end{array}\right. \end{array} $

根据满足自然边界条件的要求,应有下述等式成立:
$$
\left\{\begin{array}{ll}
2-3=\mathrm{b} \quad&(s(x) \text { 在 } x=1 \text { 处的左右一阶导数相等 }) \\
-6=2 \mathrm{c} \quad&(s(x) \text { 在 } x=1 \text { 处的左右二阶导数相等 }) \\
0=2 \mathrm{c}+6 d \quad&(s(x) \text { 在 } x=2 \text { 处及 } x=0 \text { 处的左右二阶导数均等于 } 0)
\end{array}\right.
$$

解得$b=-1, c=-3,d=1$

\end{tcolorbox}

 \begin{tcolorbox}[enhanced,colback=10,colframe=9,breakable,coltitle=green!25!black,title=2024]
  证明对于 $ f(x) $ 的以 $ x_{0}, x_{1} $ 为节点的一次插值多项式 $ L(x) $, 插值误差
$$
|f(x)-L(x)| \leqslant \frac{1}{8}\left(x_{1}-x_{0}\right)^{2} \max _{x_{0} \leqslant x \leqslant x}\left|f^{\prime \prime}(x)\right| \quad x_{0} \leqslant x \leqslant x_{1}
$$
\tcblower
 一次插值余项
$$
R(x)=f(x)-L(x)=\frac{f^{\prime \prime}(\xi)}{2} \omega(x)
$$
其中 $ \omega(x)=\left(x-x_{0}\right)\left(x-x_{1}\right) $.
对 $ \omega(x) $ 求极值, $ \omega^{\prime}(x)=\left(x-x_{0}\right)+\left(x-x_{1}\right)=0 $, 得 $ x=\frac{1}{2}\left(x_{0}+x_{1}\right), \omega^{\prime \prime}(x)=2>0 $, 有极小值, 即
$$
\left(x-x_{0}\right)\left(x-x_{1}\right) \geqslant \frac{1}{4}\left(x_{1}-x_{0}\right)^{2}
$$
取 $ x \in\left[x_{0}, x_{1}\right] $, 则 $ \left(x-x_{0}\right)\left(x-x_{1}\right) \leqslant 0 $, 有
$$
\left|\left(x-x_{0}\right)\left(x-x_{1}\right)\right| \leqslant \frac{1}{4}\left(x_{1}-x_{0}\right)^{2}
$$
所以
$$
|R(x)|=|f(x)-L(x)| \leqslant \frac{1}{8}\left(x_{1}-x_{0}\right)^{2} \max _{x_{0} \leqslant x \leqslant x_{1}}\left|f^{\prime \prime}(x)\right|, \quad x_{0} \leqslant x \leqslant x_{1}
$$
\end{tcolorbox}



  \begin{tcolorbox}[enhanced,colback=10,colframe=9,breakable,coltitle=green!25!black,title=2024]
  
已知 $ f(x) $ 在区间 $ [0,2] $ 上具有三阶连续导数,设
$$
g(x)=f(0)+(f(1)-f(0)) x+\frac{f(2)-2 f(1)+f(0)}{2}\left(x^{2}-x\right)
$$
试证明: $ \max\limits _{x \in[0,2]}|f(x)-g(x)| \leq \dfrac{\sqrt{3}}{27} \max\limits _{x \in[0,2]}\left|f^{\prime \prime \prime}(x)\right| $.
 \tcblower

对 $ g(x) $ 变形得:
$$
\begin{aligned}
g(x) & =f(0)+\frac{f(1)-f(0)}{1-0} \cdot x+\frac{f(2)-f(1)-[f(1)-f(0)]}{2-0} \cdot x \cdot(x-1) \\
& =f[0]+f[0,1] x+f[0,1,2] \cdot x \cdot(x-1) .
\end{aligned}
$$
易知 $ g(x) $ 为二次牛顿插值多项式. 因为 $ f(x) $ 为定义在 $ [0,2] $ 上且具有三阶连续导数的函数, $ x_{0}=0 , x_{1}=1 , x_{2}=2 $ 是 $[0.2]$上的 $3$ 个互异插值节点,
则 
$$
R_{2}(x)=f(x)-g(x)=f[0,1,2,x] \cdot(x-0)(x-1)(x-2)
$$
而又知 $ f[0,1,2,x]=\dfrac{f^{(3)}(\xi)}{3 !} $,则 
$$ \max _{x \in[0,2]}|f(x)-g(x)|=\max _{x \in[0,2]}\left|\frac{f^{(3)}(\xi)}{6} \cdot x(x-1)(x-2)\right| $$
令$h(x)=x \cdot(x-1)(x-2)=x^{3}-3 x^{2}+2 x,x \in[0,2]$.
由$ h^{\prime}(x)=3 x^{2}-6 x+2=0 $得$ x_{1}=1-\frac{\sqrt{3}}{3}, x_{2}=1+\frac{\sqrt{3}}{3} $.计算得到
$$
h(0)=0, \quad h(2)=0 . \quad h\left(x_{1}\right)=\frac{2 \sqrt{3}}{9} \quad h\left(x_{2}\right)=-\frac{2 \sqrt{3}}{2}
$$
于是 $ \max\limits _{x \in[0, 2]}|h(x)|=\frac{2}{9} \sqrt{3} $.


$$
\begin{aligned}
\max _{x \in[0,2]}|f(x)-g(x)|&=\max _{x \in[0,2]}\left|\frac{f^{\prime \prime \prime}(\xi)}{6} \cdot x(x-1)(x-2)\right|  \\
&\leqslant \frac{1}{6} \max _{x \in[0,2]}\left|f^{\prime \prime \prime}(x)\right| \cdot \max _{x \in[0,2]}\left|\left(x-{0}\right)\left(x-{1}\right)\left(x-{2}\right)\right| \\
&=\frac{1}{6} \cdot\frac{2 \sqrt{3}}{9} \max _{x \in[0,2]}\left|f^{\prime \prime \prime}(x)\right|\\&=\frac{\sqrt{3}}{27}   \max _{x \in[0,2]}\left|f^{\prime \prime \prime}(x)\right|
\end{aligned}
$$
得证.

\end{tcolorbox}


  \begin{tcolorbox}[enhanced,colback=10,colframe=9,breakable,coltitle=green!25!black,title=2024]
 设 $ f(x) \in C^{2}(\mathbf{R}) $, 证明:
$%
\dfrac{\mathrm{d}}{\mathrm{d} x} f[a, x]=f[a, x, x] .
$%
 \tcblower
 方法 1 : 当 $ x \neq a $ 时, 有
$$
\begin{aligned}
f  [a, x]&=\frac{f(x)-f(a)}{x-a},  \\
\frac{\mathrm{d}}{\mathrm{d} x} f[a, x] & =\frac{f^{\prime}(x)(x-a)-[f(x)-f(a)]}{(x-a)^{2}}, \\
& =\frac{f^{\prime}(x)-f[a, x]}{x-a} \\
& =\frac{f[x, x]-f[a, x]}{x-a}=f[a, x, x] .
\end{aligned}
$$
上式右边为 $ x $ 的连续函数, 因而
$$
\lim _{x \rightarrow a} \frac{\mathrm{d}}{\mathrm{d} x} f[a, x]=\lim _{x \rightarrow a} f[a, x, x]=\frac{f^{\prime \prime}(a)}{2},
$$

即对一切 $ x $ 有
$$
\frac{\mathrm{d}}{\mathrm{d} x} f[a, x]=f[a, x, x] .
$$

方法 2: 根据题意, 可得
$$
\begin{aligned}
\frac{\mathrm{d} f[a, x]}{\mathrm{d} x} & =\lim _{\Delta x \rightarrow 0} \frac{f[a, x+\Delta x]-f[a, x]}{\Delta x} \\
& =\lim _{\Delta x \rightarrow 0} f[a, x, x+\Delta x] \\
& =f[a, x, x] .
\end{aligned}
$$
 \end{tcolorbox}


    \begin{tcolorbox}[enhanced,colback=10,colframe=9,breakable,coltitle=green!25!black,title=2024]
 设 $ f \in C^{4}[a, a+2] $, 求一个 3 次多项式 $ H(x) $, 使之满足
$$
H(a)=f(a), \quad H(a+1)=f(a+1), \quad H(a+2)=f(a+2), \quad H^{\prime}(a)=f^{\prime}(a),
$$
并写出插值余项 $ f(x)-H(x) $ 的表达式.
 \tcblower
 由 Hermite 插值, 有
$$
\begin{aligned}
H(x)= & f(a)+f[a, a](x-a)+f[a, a, a+1](x-a)^{2} \\
& +f[a, a, a+1, a+2](x-a)^{2}[x-(a+1)],
\end{aligned}
$$
又
$$
\begin{array}{c}
f[a, a]=f^{\prime}(a), \quad f[a, a+1]=f(a+1)-f(a), \\
f[a+1, a+2]=f(a+2)-f(a+1), \\
f[a, a, a+1]=f(a+1)-f(a)-f^{\prime}(a), \\
f[a, a+1, a+2]=\frac{1}{2}[f(a+2)-2 f(a+1)+f(a)], \\
f[a, a, a+1, a+2]=\frac{1}{4}\left[f(a+2)-4 f(a+1)+3 f(a)+2 f^{\prime}(a)\right],
\end{array}
$$
因此
$$
\begin{aligned}
H(x)= & f(a)+f^{\prime}(a)(x-a)+\left[f(a+1)-f(a)-f^{\prime}(a)\right](x-a)^{2} \\
& +\frac{1}{4}\left[f(a+2)-4 f(a+1)+3 f(a)+2 f^{\prime}(a)\right](x-a)^{2}[x-(a+1)],
\end{aligned}
$$
且
$$
f(x)-H(x)=\frac{f^{(4)}(\xi)}{4!}(x-a)^{2}(x-a-1)(x-a-2), \quad \xi \in(a, a+2) .
$$
 \end{tcolorbox}


    \begin{tcolorbox}[enhanced,colback=10,colframe=9,breakable,coltitle=green!25!black,title=2024]
 设函数 $ f(x) \in C^{3}[a, b] $, 并且 $ f(a)=f(b)=0 $.
 
(1) 求一个 2 次多项式 $ p(x) $, 使其满足 $ p(a)=f(a), p^{\prime}(a)=f^{\prime}(a), p(b)=f(b) $;

(2) 求一个 2 次多项式 $ q(x) $, 使其满足 $ q(a)=f(a), q(b)=f(b), q^{\prime}(b)=f^{\prime}(b) $;

(3) 证明: $\displaystyle \max _{a \leqslant x \leqslant b}\left|f(x)-\frac{f^{\prime}(b)-f^{\prime}(a)}{2(b-a)}(x-a)(x-b)\right| \leqslant \frac{1}{48}(b-a)^{3} \max _{a \leqslant x \leqslant b}\left|f^{\prime \prime \prime}(x)\right| $.

 \tcblower
 (1) 由 Hermite 插值知
$$
p(x)=f(a)+f[a, a](x-a)+f[a, a, b](x-a)^{2} .
$$
由 $ f[a, a]=f^{\prime}(a), f(a)=f(b)=0 $, 得
$$
f[a, a, b]=\frac{f[a, b]-f[a, a]}{b-a}=-\frac{f^{\prime}(a)}{b-a},
$$
所以
$$
p(x)=f^{\prime}(a)(x-a)-\frac{f^{\prime}(a)}{b-a}(x-a)^{2}=-\frac{f^{\prime}(a)}{b-a}(x-a)(x-b) .
$$
(2) 由 Hermite 插值知
$$
q(x)=f(a)+f[a, b](x-a)+f[a, b, b](x-a)(x-b),
$$
又 $ f[a, b]=0, f[a, b, b]=f^{\prime}(b) /(b-a) $, 所以
$$
q(x)=\frac{f^{\prime}(b)}{b-a}(x-a)(x-b) .
$$
(3) 对任意 $ x \in[a, b] $, 有
$$
\begin{aligned}
 &\left|f(x)-\frac{f^{\prime}(b)-f^{\prime}(a)}{2(b-a)}(x-a)(x-b)\right| 
=  \left|f(x)-\frac{1}{2} p(x)-\frac{1}{2} q(x)\right| \\
\leqslant & \frac{1}{2}|f(x)-p(x)|+\frac{1}{2}|f(x)-q(x)| \\
= & \frac{1}{2}\left|\frac{f^{\prime \prime \prime}(\xi)}{3!}(x-a)^{2}(x-b)\right|+\frac{1}{2}\left|\frac{f^{\prime \prime \prime}(\eta)}{3!}(x-a)(x-b)^{2}\right|, \quad \xi, \eta \in(a, b) \\
\leqslant & \frac{1}{12} \max _{a \leqslant x \leqslant b}\left|f^{\prime \prime \prime}(x)\right|\left[(x-a)^{2}(b-x)+(x-a)(x-b)^{2}\right] \\
= & \frac{1}{12} \max _{a \leqslant x \leqslant b}\left|f^{\prime \prime \prime}(x)\right|(x-a)(b-x)(b-a) \\
\leqslant & \frac{1}{12} \max _{a \leqslant x \leqslant b}\left|f^{\prime \prime \prime}(x)\right| \frac{(b-a)^{2}}{4}(b-a) =  \frac{1}{48}(b-a)^{3} \max _{a \leqslant x \leqslant b}\left|f^{\prime \prime \prime}(x)\right| .
\end{aligned}
$$
 \end{tcolorbox}


    \begin{tcolorbox}[enhanced,colback=10,colframe=9,breakable,coltitle=green!25!black,title=2024]
 设 $ f(x) \in C^{2}[1,3] $, 求 3 次多项式 $ H(x) $, 使得
$$
H(1)=f(1), \quad H(2)=f(2), \quad H(3)=f(3), \quad H^{\prime \prime}(1)=f^{\prime \prime}(1) .
$$
 \tcblower

设 $ p(x) $ 为 2 次多项式, 满足
$$
p(1)=f(1), \quad p(2)=f(2), \quad p(3)=f(3),
$$
则
$$
p(x)=f(1)+f[1,2](x-1)+f[1,2,3](x-1)(x-2),
$$
其中
$$
\begin{array}{c}
f[1,2]=f(2)-f(1), \\
f[1,2,3]=\frac{1}{2}[f(3)-2 f(2)+f(1)] .
\end{array}
$$
易知
$ H(x)=p(x)+A(x-1)(x-2)(x-3) $, 其中 $ A $ 为待定常数,由条件 $ H^{\prime \prime}(1)=f^{\prime \prime}(1) $ 可得
$$
\begin{array}{c}
2 f[1,2,3]-6 A=f^{\prime \prime}(1), \\
A=\frac{1}{6}\left\{2 f[1,2,3]-f^{\prime \prime}(1)\right\}=\frac{1}{6}\left[f(3)-2 f(2)+f(1)-f^{\prime \prime}(1)\right],
\end{array}
$$
所以
$$
\begin{aligned}
H(x)= & f(1)+[f(2)-f(1)](x-1)+\frac{1}{2}[f(3)-2 f(2)+f(1)](x-1)(x-2) \\
& +\frac{1}{6}\left[f(3)-2 f(2)+f(1)-f^{\prime \prime}(1)\right](x-1)(x-2)(x-3) .
\end{aligned}
$$
 \end{tcolorbox}

   \begin{tcolorbox}[enhanced,colback=10,colframe=9,breakable,coltitle=green!25!black,title=2024]
 设 $ f(x) \in C^{1}[0,1] $, 求 4 次多项式 $ P(x) $, 使得
$$
\begin{array}{ll}
P(0)=f(0), & P(1)=f(1), \\
P^{\prime}(0)=f^{\prime}(0), P^{\prime}\left(\frac{1}{3}\right)=f^{\prime}\left(\frac{1}{3}\right), & P^{\prime}(1)=f^{\prime}(1) .
\end{array}
$$
 \tcblower
作 3 次多项式 $ H(x) $, 使其满足
$$
\begin{array}{cc}
H(0)=f(0), & H(1)=f(1), \\
H^{\prime}(0)=f^{\prime}(0), & H^{\prime}(1)=f^{\prime}(1),
\end{array}
$$
则
$$
H(x)=f(0)+f[0,0] x+f[0,0,1] x^{2}+f[0,0,1,1] x^{2}(x-1) .
$$
列表求差商:
\begin{center}
\begin{tabular}{c|cccc}
\hline$ x_{i} $ & $ f\left(x_{i}\right) $ & & & \\
\hline 0 & $ f(0) $ & $ f^{\prime}(0) $ & $ f[0,0,1] $ & $ f[0,0,1,1] $ \\
0 & $ f(0) $ & $ f[0,1] $ & $ f[0,1,1] $ & \\
1 & $ f(1) $ & $ f^{\prime}(1) $ & & \\
1 & $ f(1) $ & & & \\
\hline
\end{tabular}
\end{center}

其中
$$
\begin{array}{c}
f[0,1]=f(1)-f(0), \quad f[0,0,1]=f(1)-f(0)-f^{\prime}(0), \\
f[0,1,1]=f^{\prime}(1)-f(1)+f(0), \\
f[0,0,1,1]=f^{\prime}(1)-2 f(1)+2 f(0)+f^{\prime}(0),
\end{array}
$$
所以
$$
\begin{aligned}
H(x)= & f(0)+f^{\prime}(0) x+\left[f(1)-f(0)-f^{\prime}(0)\right] x^{2} \\
& +\left[f^{\prime}(1)-2 f(1)+2 f(0)+f^{\prime}(0)\right] x^{2}(x-1) .
\end{aligned}
$$
易知$P(x)=H(x)+A x^{2}(x-1)^{2}$, 其中 $A$  为待定常数, $P^{\prime}(x)=H^{\prime}(x)+2 A x(x-1)(2 x-1),$ 由条件$P^{\prime}\left(\frac{1}{3}\right)=f^{\prime}\left(\frac{1}{3}\right)$ 可得
$$
A=\frac{9}{4}\left[3 f^{\prime}\left(\frac{1}{3}\right)-4 f(1)+4 f(0)+f^{\prime}(1)\right],
$$
因此
$$
P(x)=H(x)+\frac{9}{4}\left[3 f^{\prime}\left(\frac{1}{3}\right)-4 f(1)+4 f(0)+f^{\prime}(1)\right] x^{2}(x-1)^{2} .
$$

 \end{tcolorbox}

   \begin{tcolorbox}[enhanced,colback=10,colframe=9,breakable,coltitle=green!25!black,title=2024]
 设 $ f(x) \in C^{1}[0,2] $.
 
(1) 求 4 次多项式 $ H(x) $, 使得
$$
H(0)=f(0), H(1)=f(1), H^{\prime}(1)=f^{\prime}(1), H(2)=f(2), H^{\prime}(2)=f^{\prime}(2) \text {; }
$$
(2) 证明: 满足以上插值条件的 4 次多项式是唯一的.
\tcblower
 (1) 构造差商表:
 \begin{center}
\begin{tabular}{c|ccccc}
\hline$ x_{i} $ & $ f\left(x_{i}\right) $ & & & & \\
\hline $ \mathbf{0} $ & $ f(0) $ & $ f[0,1] $ & $ f[0,1,1] $ & $ f[0,1,1,2] $ & $ f[0,1,1,2,2] $ \\
1 & $ f(1) $ & $ f^{\prime}(1) $ & $ f[1,1,2] $ & $ f[1,1,2,2] $ & \\
1 & $ f(1) $ & $ f[1,2] $ & $ f[1,2,2] $ & & \\
2 & $ f(2) $ & $ f^{\prime}(2) $ & & & \\
2 & $ f(2) $ & & & & \\
\hline
\end{tabular}
 \end{center}
 其中
$$
\begin{array}{c}
f[0,1]=f(1)-f(0), f[1,2]=f(2)-f(1), \\
f[0,1,1]=f^{\prime}(1)-f(1)+f(0), \\
f[1,1,2]=f(2)-f(1)-f^{\prime}(1), \\
f[1,2,2]=f^{\prime}(2)-f(2)+f(1), \\
f[0,1,1,2]=\frac{1}{2}\left[f(2)-f(0)-2 f^{\prime}(1)\right], \\
f[1,1,2,2]=f^{\prime}(2)-2 f(2)+2 f(1)+f^{\prime}(1), \\
f[0,1,1,2,2]=\frac{1}{4}\left[2 f^{\prime}(2)+4 f^{\prime}(1)-5 f(2)+4 f(1)+f(0)\right],
\end{array}
$$
所以
$$
\begin{aligned}
H(x)= & f(0)+[f(1)-f(0)] x+\left[f^{\prime}(1)-f(1)+f(0)\right] x(x-1) \\
& +\frac{1}{2}\left[f(2)-f(0)-2 f^{\prime}(1)\right] x(x-1)^{2} \\
& +\frac{1}{4}\left[2 f^{\prime}(2)+4 f^{\prime}(1)-5 f(2)+4 f(1)+f(0)\right] x(x-1)^{2}(x-2) .
\end{aligned}
$$
(2) 设另有 4 次多项式 $ G(x) $ 满足
$$
G(0)=f(0), G(1)=f(1), G^{\prime}(1)=f^{\prime}(1), G(2)=f(2), G^{\prime}(2)=f^{\prime}(2) \text {, }
$$
令$P(x)=H(x)-G(x)$,
则 $ P(x) $ 为 4 次多项式, 且满足
$$
P(0)=0, P(1)=0, P^{\prime}(1)=0, P(2)=0, P^{\prime}(2)=0,
$$
即 0 为 $ P(x) $ 的单零点, 1 和 2 为 $ P(x) $ 的二重零点, 因而 $ P(x) $ 有 5 个零点, 从而 $ P(x) \equiv 0 $, 即 $ G(x)=H(x) $.
   \end{tcolorbox}

   
   \begin{tcolorbox}[enhanced,colback=10,colframe=9,breakable,coltitle=green!25!black,title=2024]
 设 $ f(x) \in C^{2}[a, b] $. 作一个 3 次多项式 $ H(x) $, 使其满足如下条件:
$$
H(a)=f(a), H^{\prime}(a)=f^{\prime}(a), H^{\prime \prime}(a)=f^{\prime \prime}(a), H^{\prime}(b)=f^{\prime}(b) \text {. }
$$
\tcblower
设 $ p(x) $ 为一个 2 次多项式, 满足
$$
p(a)=f(a), p^{\prime}(a)=f^{\prime}(a), p^{\prime \prime}(a)=f^{\prime \prime}(a),
$$
则
$$
\begin{aligned}
p(x) & =f(a)+f[a, a](x-a)+f[a, a, a](x-a)^{2} \\
& =f(a)+f^{\prime}(a)(x-a)+\frac{f^{\prime \prime}(a)}{2}(x-a)^{2},
\end{aligned}
$$
记 $ R(x)=H(x)-p(x) $, 则 $ R(x) $ 为 3 次多项式, 且有
$$
R(a)=R^{\prime}(a)=R^{\prime \prime}(a)=0,
$$
因此
$R(x)=A(x-a)^{3},$ 其中 $ A $ 为待定常数. 所以
$$
\begin{aligned}
H(x) & =p(x)+A(x-a)^{3} \\
& =f(a)+f^{\prime}(a)(x-a)+\frac{f^{\prime \prime}(a)}{2}(x-a)^{2}+A(x-a)^{3},
\end{aligned}
$$
求导得
$$
H^{\prime}(x)=f^{\prime}(a)+f^{\prime \prime}(a)(x-a)+3 A(x-a)^{2},
$$
由条件 $ H^{\prime}(b)=f^{\prime}(b) $ 可得
$$
f^{\prime}(a)+f^{\prime \prime}(a)(b-a)+3 A(b-a)^{2}=f^{\prime}(b),
$$
求得
$$
A=\frac{f^{\prime}(b)-f^{\prime}(a)-(b-a) f^{\prime \prime}(a)}{3(b-a)^{2}},
$$
所以
$$
\begin{aligned}
H(x)= & f(a)+f^{\prime}(a)(x-a)+\frac{f^{\prime \prime}(a)}{2}(x-a)^{2} \\
& +\frac{f^{\prime}(b)-f^{\prime}(a)-(b-a) f^{\prime \prime}(a)}{3(b-a)^{2}}(x-a)^{3} .
\end{aligned}
$$
   \end{tcolorbox}

\begin{tcolorbox}[enhanced,colback=10,colframe=9,breakable,coltitle=green!25!black,title=2024]

 证明两点三次埃尔米特插值余项是
$$
R_{3}(x)=\frac{f^{(4)}(\xi)}{4!}\left(x-x_{k}\right)^{2}\left(x-x_{k+1}\right)^{2}, \quad \xi \in\left(x_{k}, x_{k+1}\right),
$$
并由此求出分段三次埃尔米特插值的误差限.
\tcblower
【证明】利用 $ \left[x_{k}, x_{k+1}\right] $ 上两点三次埃尔米特插值条件
$$
\begin{array}{l}
H_{3}\left(x_{k}\right)=f\left(x_{k}\right), \quad H_{3}\left(x_{k+1}\right)=f\left(x_{k+1}\right) \\
H_{3}^{\prime}\left(x_{k}\right)=f^{\prime}\left(x_{k}\right), \quad H_{3}^{\prime}\left(x_{k+1}\right)=f^{\prime}\left(x_{k+1}\right)
\end{array}
$$
可知 $ R_{3}(x)=f(x)-H_{3}(x) $ 有二重零点 $ x_{k} $ 和 $ x_{k+1} $.
故设插值余项为 $ R_{3}(x)=k(x)\left(x-x_{k}\right)^{2}\left(x-x_{k+1}\right)^{2} $.
当 $ x=x_{k} $ 或 $ x_{k+1} $ 时 $ k(x) $ 取任何有限值均可;
当 $ x \neq x_{k}, x_{k+1} $ 时, $ x \in\left(x_{k}, x_{k+1}\right) $, 构造关于变量 $ t $ 的函数
$$
g(t)=f(t)-H_{3}(t)-k(x)\left(t-x_{k}\right)^{2}\left(t-x_{k+1}\right)^{2}
$$
则 $ g\left(x_{k}\right)=0, g(x)=0, g\left(x_{k+1}\right)=0, g^{\prime}\left(x_{k}\right)=0, g^{\prime}\left(x_{k+1}\right)=0 $.
在 $ \left[x_{k}, x\right] $ 和 $ \left[x, x_{k+1}\right] $ 上对 $ g(x) $ 使用 Rolle 定理, 存在 $ \eta_{1} \in\left(x_{k}, x\right) $ 及 $ \eta_{k} \in\left(x, x_{k+1}\right) $ 使得
$$
g^{\prime}\left(\eta_{1}\right)=0, \quad g^{\prime}\left(\eta_{2}\right)=0 .
$$
在 $ \left(x_{k}, \eta_{1}\right),\left(\eta_{1}, \eta_{2}\right),\left(\eta_{2}, x_{k+1}\right) $ 上对 $ g^{\prime}(x) $ 使用 Rolle 定理, 存在 $ \eta_{11} \in\left(x_{k}, \eta_{1}\right), \eta_{12} \in\left(\eta_{1}, \eta_{2}\right) $ 和 $ \eta_{13} \in\left(\eta_{2}, x_{k+1}\right) $ 使得
$$
g^{\prime \prime}\left(\eta_{11}\right)=g^{\prime \prime}\left(\eta_{12}\right)=g^{\prime \prime}\left(\eta_{13}\right)=0
$$
再依次对 $ g^{\prime \prime}(t) $ 和 $ g^{\prime \prime \prime}(t) $ 使用 Rolle 定理, 知至少存在 $ \xi \in\left(x_{k}, x_{k+1}\right) $ 使得
$$
g^{(4)}(\xi)=0
$$
而 $ g^{(4)}(t)=f^{(4)}(t)-k(x) 4 $ !, 将 $ \xi $ 代入, 得到
$$
k(x)=\frac{1}{4!} f^{(4)}(\xi), \quad \xi \in\left(x_{k}, x_{k+1}\right)
$$
推导过程表明 $ \xi $ 依赖于 $ x_{k}, x_{k+1} $ 及 $ x $. 综上可得
$$
R_{3}(x)=\frac{1}{4!} f^{(4)}(\xi)\left(x-x_{k}\right)^{2}\left(x-x_{k+1}\right)^{2} .
$$
下面建立分段三次埃尔米特插值的误差限.
设 $ x_{k}=a+k h \quad(k=0,1, \cdots, n), \quad h=\frac{b-a}{n} $
则分段三次埃尔米特插值的误差为
$$
\begin{aligned}
|R(x)| & =\frac{1}{4!}\left|f^{(4)}(\xi)\right|\left(x-x_{k}\right)^{2}\left(x-x_{k+1}\right)^{2} \\
& \leqslant \frac{1}{4!} \max _{a \leqslant s \leqslant b}\left|f^{(4)}(x)\right| \max _{x_{k} \leqslant x \leqslant x_{k+1}}\left(x-x_{k}\right)^{2}\left(x-x_{k+1}\right)^{2} \\
& \leqslant \frac{1}{384} h^{4} \max _{a \leqslant \leqslant \leqslant b}\left|f^{(4)}(x)\right| .
\end{aligned}
$$
\end{tcolorbox}

\begin{tcolorbox}[enhanced,colback=10,colframe=9,breakable,coltitle=green!25!black,title=2024]

设 $ f(x) \in C^{3}[a, b] $, 作一个 2 次多项式 $ p(x) $ 使得
$$
p(a)=f(a), p^{\prime}(a)=f(a), p(b)=f(b)
$$
并证明
$$
f(x)-p(x)=\frac{1}{6} f^{\prime \prime \prime}(\xi)(x-a)^{2}(x-b)
$$
其中 $ \xi \in(\min \{a, x\}, \max \{b, x\}) $
\tcblower
求插值多项式,记
$
L_{1}(x)=f(a) \dfrac{x-a}{b-a}+f(b) \dfrac{x-b}{a-b},
$
并令 $ p(x)=L_{1}(x)+Q(x) $,
则易知 $ Q(x)=p(x)-L_{1}(x) $
为不超过 2 次的多项式, 且 $ Q(a)=Q(b)=0 $,
于是
$$
\begin{aligned}
Q(x) & =A(x-a)(x-b) \\
p(x) & =f(a) \frac{x-b}{a-b}+f(b) \frac{x-a}{b-a}+A(x-a)(x-b)
\end{aligned}
$$
其中 $ A $ 为待定常数.由 $p^{\prime}(x)=\frac{f(b)-f(a)}{b-a}+A(2 x-a-b)$ 及
$p^{\prime}(a)= f^{\prime}(a)$  得 $\frac{f(b)-f(a)}{b-a}+A(a-b)=f^{\prime}(a)$.
于是
$$
A=\frac{1}{b-a}\left[\frac{f(b)-f(a)}{b-a}-f^{\prime}(a)\right]
$$
因而
$$
p(x)=f(a) \frac{x-b}{a-b}+f(b) \frac{x-a}{b-a}+ \frac{1}{b-a}\left[\frac{f(b)-f(a)}{b-a}-f^{\prime}(a)\right](x-a)(x-b)
$$
求余项.记 $R(x)=f(x)-p(x)$, 则 $ a $ 为 $ R(x) $ 的 2 重零点, $ b $ 为 $ R(x) $ 的 1 重零点.因而可设 $ R(x) $ 具有如下形式
$$
R(x)=K(x)(x-a)^{2}(x-b)
$$
其中 $ K(x) $ 待定.当 $ x=a, b $ 时上式两端均为 0 , 因而 $ k(x) $ 取任意常数均成立.现考虑 $ x \neq a, b $ 的情形.暂时固定 $ x $, 作铺助函数
$$
\varphi(t)=R(t)-k(x)(t-a)^{2}(t-b)
$$
显然 $ \varphi(t) $ 以 $ a $ 为 2 重零点,以 $  b $ 为 1 重零点.反复应用 Rolle 定理可知 $ \varphi^{\prime \prime \prime}(t) $ 至少有一个零点 $ \xi_{0} $. 由  $\varphi^{\prime \prime \prime}(t)=R^{\prime \prime \prime}(t)-6 K(x)=f^{\prime \prime \prime}(t)-6 K(x)$ 得 $K(x)=f^{\prime \prime}(\xi) / 6
$
因而
$$
f(x)-p(x)=\frac{1}{6} f^{\prime \prime \prime}(\xi)(x-a)^{2}(x-b)
$$
\end{tcolorbox}